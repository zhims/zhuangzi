%%%%%%%%%%%%%%%%%%%%%%%%%%%%%%%%%%%%%%%%%%%%%%%%%%%%%%%%%%%%%%%%%%%%%%%%%%%%%%%%%%%%%%%%%%

%%%%%%%%%%%%%%%%%%%%%%%%%%%%%%%%%%% Author:Yao Zhang  %%%%%%%%%%%%%%%%%%%%%%%%%%%%%%%%%%%%
%%%%%%%%%%%%%%%%%%%%%%%%%%%%% Email: jaafar_zhang@163.com  %%%%%%%%%%%%%%%%%%%%%%%%%%%%%%%

%%%%%%%%%%%%%%%%%%%%%%%%%%%%%%%%%%%%%%%%%%%%%%%%%%%%%%%%%%%%%%%%%%%%%%%%%%%%%%%%%%%%%%%%%%

\documentclass[11pt]{article}
\usepackage[utf8]{inputenc} 
\usepackage[table]{xcolor}
\usepackage[most]{tcolorbox}
\usepackage[left=2.50cm, right=1.50cm, top=2.0cm, bottom=2.50cm]{geometry}
\usepackage{xcolor,url,cite}
\usepackage{amsmath,amsthm,amsfonts,amssymb,amscd,multirow,booktabs,fullpage,calc,multicol}
\usepackage{lastpage,enumitem,fancyhdr,mathrsfs,wrapfig,setspace,cancel,amsmath,empheq,framed}
\usepackage[retainorgcmds]{IEEEtrantools}
\usepackage{algorithm}
\usepackage{algorithmic}
\newlength{\tabcont}
\setlength{\parindent}{0.0in}
\setlength{\parskip}{0.05in}
\colorlet{shadecolor}{orange!15}
\parindent 0in
\parskip 12pt
\geometry{margin=1in, headsep=0.25in}
\usepackage{subfig,graphicx,framed}
\graphicspath{ {img1/} }
\usepackage{ctex}
\usepackage{multirow, tabularx}
%%%%%%%%%%%%%%%%%%%%%%%%%%%%%%%%%%%%%%%%%%%%%%%%%%%%%%%%%%%%%%%%%%%%%%%%%%%%%%%%%%%%%%%%%%

\newtheorem{theorem}{Theorem}[section]
\newtheorem{definition}{Definition}[section]
\newtheorem{exercise}{Exercise}[section]
\newtheorem{note}{Note}[section]
\newtheorem{notation}{Notation}
\newtheorem{lemma}{Lemma}[subsection]
\newtheorem{proposition}{Proposition}[section]
\newtheorem{example}{Example}[section]
\newtheorem{homework}{Homework}[section]
\newtheorem{summary}{Summary}[section]
\newtheorem{corollary}{Corollary}[section]
\newtheorem*{remark}{Remark}
\makeatletter
\@addtoreset{equation}{section}
\makeatother
\renewcommand{\theequation}{\arabic{section}.\arabic{equation}}
\usepackage[colorlinks,linkcolor=blue, anchorcolor=green,citecolor=red,urlcolor=blue]{hyperref}


%%%%%%%%%%%%%%%%%%%%%%%%%%%%%%%%%%%%%%%%%%%%%%%%%%%%%%%%%%%%%%%%%%%%%%%%%%%%%%%%%%%%%%%%%%

\def\beginrefs{\begin{list}%
		{[\arabic{equation}]}{\usecounter{equation}
			\setlength{\leftmargin}{0.8truecm}\setlength{\labelsep}{0.4truecm}%
			\setlength{\labelwidth}{1.6truecm}}}
	\def\endrefs{\end{list}}
\def\bibentry#1{\item[\hbox{[#1]}]}

\def\UrlBreaks{\do\A\do\B\do\C\do\D\do\E\do\F\do\G\do\H\do\I\do\J
	\do\K\do\L\do\M\do\N\do\O\do\P\do\Q\do\R\do\S\do\T\do\U\do\V
	\do\W\do\X\do\Y\do\Z\do\[\do\\\do\]\do\^\do\_\do\`\do\a\do\b
	\do\c\do\d\do\e\do\f\do\g\do\h\do\i\do\j\do\k\do\l\do\m\do\n
	\do\o\do\p\do\q\do\r\do\s\do\t\do\u\do\v\do\w\do\x\do\y\do\z
	\do\.\do\@\do\\\do\/\do\!\do\_\do\|\do\;\do\>\do\]\do\)\do\,
	\do\?\do\'\do+\do\=\do\#}

\renewcommand{\algorithmicrequire}{\textbf{Input:}}  % Use Input in the format of Algorithm
\renewcommand{\algorithmicensure}{\textbf{Output:}}
%\renewcommand{\figurename}{\kaishu 图}
%%%%%%%%%%%%%%%%%%%%%%%%%%%%%%%%%%%%%%%%%%%%%%%%%%%%%%%%%%%%%%%%%%%%%%%%%%%%%%%%%%%%%%%%%%
\begin{document}
\kaishu 
	
\setcounter{section}{6}
\title{title}
\thispagestyle{empty}

\begin{center}
	{\Large \kaishu 齐物论 \ \ 莫若以明 \ 下}
	
	%{\large \kaishu  告别负面情绪的方法}
	
	{\vspace{-0.2cm}}
	
	{\kaishu 蔡璧名}
\end{center}

%\section{}
% \subsection{\kaishu 告别负面情绪的方法}

{\Large {\color{purple} 朝三暮四---可曾发现让你欢喜与愤怒的主张, 到头来并无不同?}}

\begin{center}
	{\color{magenta} 劳神明为一, 而不知其同也. 谓之朝三. 何谓朝三?}
\end{center}

\vspace{-0.5cm}

接下来庄子要说一个故事, 可是我们在世界上我们都不这么想, 所以我们有烦恼, {\color{blue} 劳神明为一,} 我们常常会坚持一件事, 然后非常劳神, 你觉得非这样不可.

谈个恋爱, 你觉得我非他不可, 其实这都是一种偏执. 你不知道, 其实有变化跟没有变化结果是一样的, 还说不定变化更好呢. 庄子说: 这就是朝三. 我问他什么叫朝三, 他告诉我们, 那个大家都知道的朝三暮四的典故就是从这儿来.

\begin{center}
	{\color{magenta} 狙公赋芧(xu), 曰: 朝三而莫四. 众狙皆怒.}
\end{center}

\vspace{-0.5cm}

有一个养猴的人, 这个赋就是赋予,给予, 他给予他的猴子们, {\color{blue} 狙公赋芧}, 这个芧是什么? 是橡食, 橡树的果实. 狙公他每天喂猴儿的时候也会说话, 他就和猴儿说: 猴啊猴, 我就今天早上喂你们三升橡食, 今晚喂四升, 怎么样? 哎呀, 猴子们觉得太少,太少了, 三升哪吃得饱? 于是就非常生气地集体暴动了. 
\begin{center}
	{\color{magenta} 曰: 然则朝四而莫三. 众狙皆悦. 名实未亏, 而喜,怒为用, 亦因是也.}
\end{center}

\vspace{-0.5cm}

狙公见状就说了: 不然这样好了, 我今天早上喂给你们四升, 黄昏喂给你们三升怎样? 那猴子就很高兴地说: 好啊, 好啊, 谢谢狙公, 谢谢, 就很开心. 这群傻猴, 它们怎么那么傻呢? 这给的食物, 名义都是芧, 都橡食, 实质也是橡食, 都没有增减, 一天都是七升, 猴子们一下好开心, 一下好生气. 可是我们人在日常生活中, 常常也是这样.

在台湾, 我们今天应该最不敢谈的一个话题, 就是蓝绿的对立, 对不对? 蓝绿的对立如果再换成一个语言, 它背后可能或多或少, 在整个历史里面有统独意识存在, 我们觉得蓝才是统, 绿代表独. 那这个时候就有一个人, 后来他的书有一度被说成是禁书, 好像是殷海光先生吧, 因为他在台湾的戒严时代他说了一句话, 他说: 蒋介石是C型台独, 台独跟感冒一样, 跟流感一样有ABC型, 怎么说呢? 因为他不断说, 要反攻大陆都还没有反攻, 说三民主义要统一中国, 也没有统一, 就这样偏安江南, 所以是C型台独. 哇, 说完这句话他的书就变禁书了, 可想而知在戒严时代. 可说他说的话, 为什么要被抓起来? 因为如果听了大家不会觉得有一点道理, 可能也不用抓起来, 懂吧? 因为好像有一点道理. 

那我们说民进党执政, 陈水扁应该是比较偏独得吧, 可是独是不是个假性台独呢, 不然怎么当了八年总统也没看他真的喊台湾独立呢? 所以如果统也不是真统, 独也不是真独. 那选民为什么要因为他们是统独, 这点意识形态, 这么点蓝绿而去支持他们呢? 

当你执著于这个意识形态, 你就可以完全不管你的生活, 你的食衣住行是不是被照顾得好, 其实我们最需要的不就是管理我们衣食住行的人吗? 我记得我以前在课堂上讲这样的话, 别人都听了没感觉.

 那时我举的例子是, 我的老家在现在的新北市, 然后我常会走路回家, 从台大, 以前, 然后因为走路回家就要避开车流, 所以会走一些比较偏僻的地方, 会经过一个有点可怕的隧道, 那隧道里有可怕的壁画, 不管谁执政, 一直换人, 那壁画上很可怕的文字跟图像都没有被清掉, 当然我也没有很尽责打电话去反映啦. 然后呢, 那个天桥, 在上下学走路的时候, 只要是雨天, 噗啾, 那地方永远都噗啾, 所以我在想, 其实谁执政我并不在乎, 但我总希望这个地下道干净一点, 明亮一点, 走过的时候, 路人不必觉得害怕, 然后人行道好一点, 不是吗?

我们谁不想, 我们的住, 房价很高,很高, 这些东西都可以不用管理, 就用一个统跟独, 一个蓝跟绿就可以赢得选票, 那是多傻? 我们还这么傻的那一天, 我们就是庄子笔下的一只猴子喔. 




\begin{center}
	{\color{magenta} 是以圣人和之以是非, 而休乎天钧, 是之谓两行.}
\end{center}

\vspace{-0.5cm}

所以呢, {\color{blue} 圣人和之以是非}, 会把这些冲突调和, 冲突都能够化解, 怎么样化解? 直捣问题的核心, 让争议停止在该停止的地方. {\color{blue} 是之谓两行}, 都是其存在的理由, 这叫做两可, 什么叫都有其存在的理由? 

我们今天说发展经济当然对一个国家,对一个社区都是重要的, 环境保育, 当然也是重要的, 那两个冲突怎么办? 难道真的没有办法找出既能保护好环境, 甚至于发挥好环境优势的一种经济方式吗? 一定有的, 那如果还没有, 那是不是最对的人还没有在最对的位置? 有时候我看到台湾发展观光的一些建议, 这在庄子的角度都觉得很悲伤. 比方说, 我们这样的一个亚热带地区, 居然要搞一个企鹅馆. 比方说, 南投这样一个有这么多喝茶文化跟资源的地方, 居然不是要盖一个茶业博物馆, 而是要盖一个侏罗纪公园, 要摆很多的恐龙在这个茶山之间, 难道台湾也是一个华人文化园的地方, 我们除了黄色小鸭跟侏罗纪公园. 

我们没有文化吗? 我们没有观光文化吗? 我们能阅读的书难道除了<<魔戒>>>跟<<哈利波特>>, 我们这一代没有共同的语言了吗? 我觉得这是我们要深思的喔. 这时候你再去看很多的政策, 你会有跟之前不同的眼光. 你到了世界各地去旅行, 你会觉得最精彩的地方就是它有在地特色, 对不对? 它能发挥它的优势, 那这一点我倒是觉得我们真的很爱这块土地, 尤其是知识分子我们要注意的, 也许有一天你会到一个可以有决定权的位置,尤其是贵班, 你们将来随便弄一块地, 发展一个什么农园, 就可能弄出很有在地特色的东西. 


{\Large {\color{purple} 可谓成乎---什么是你心目中的成就?}}

那接下来, 我们刚刚天南地北地聊, 我们现在要来聊, 什么是这个世界上最珍贵的知识? 那关于这一点, 庄子有很特别的说法. 

来, 请同学念一下: {\color{blue} 古之人, 其知有所至矣. 恶乎至?}

{\color{blue} 古之人, 其知有所至矣. 恶乎至? 有以为未始有物者. 至矣, 尽矣, 不可以加矣. 其次以为有物矣, 而未始有封也. 其次以为有封焉, 而未始有是非也. 是非之彰也, 道之所以亏也. 道之所亏, 爱之所以成. 果且有成与亏乎哉? 果且无成与亏乎哉? 有成与亏, 故昭氏之鼓琴也; 无成与亏, 故昭氏之不鼓琴也. 昭文之鼓琴也, 师旷之枝策也, 惠子之据梧也, 三子之知几乎! 皆其盛者也, 故载之末年. 唯其好之也, 以异于彼; 其好之也, 欲以明之彼. 非所明而明之, 故以坚白之昧终. 而其子又以文之纶终, 终身无成. 若是而可谓之成乎? 虽我亦成也. 若是而不可谓成乎? 物与我无成也. 是故滑疑之耀, 圣人之所图也. 为是不用而寓诸庸, 此之谓以明.}

好. 
%%%%%%%%%%%%%%%%%%%%%%%%%%%%%%%%%%%%%%%%%%%%%%%%%%%%%%%%%%%%%%%%%%%%%%%%%%%%%%%%%%%%%%%%%%%%%%%%%%%%%%%%%%%%%%%%%%%%%%%%%%%%%%%%%%%%%%%%%%%%%%%%%%%%%%%%%%%
\begin{center}
	{\color{magenta} 古之人, 其知有所至矣. 恶乎至?}
\end{center}

\vspace{-0.5cm}

古之人, 古时候的人啊, 他的所知到达极致了, 什么是极致呢? 什么是庄子认为的极致呢? 

\begin{center}
	{\color{magenta} 有以为未始有物者. 至矣, 尽矣, 不可以加矣.}
\end{center}

\vspace{-0.5cm}

居然是{\color{blue} 有以为未始有物者}, 在任何的物质现象还不存在的时候, 有一种知识, 就是在探索那是什么呢? 讲的就是心神灵魂, 优先于一切具体事物存在, 是没有形体, 没办法看见的. 可是关于这样的一种知识, 也就是能提升我们的心神灵魂的知识, 在庄子的学说里面, 认为是最重要的,最高的知识了, 无以复加. 

\begin{center}
	{\color{magenta} 其次以为有物矣, 而未始有封也. 其次以为有封焉, 而未始有是非也.}
\end{center}

\vspace{-0.5cm}

其次呢, {\color{blue} 以为有物矣, 而未始有封也}. 刚刚讲, 大家看了会觉得很奇怪. 为什么跟心神灵魂相关的知识是会最重要的知识? 在<<庄子>>里面他会讲到真跟假, 什么叫假? 讲的假不是说它是假的而是说它是凭借的, 它是短暂的, 它是要告别的. 

我们的一生, 我们很多人的追求, 一个很好的房子,一辆觉得很酷的车子, 一个你觉得很满意的对象, 情感对象, 但是这些东西呢, 庄子不是不珍惜, 但是有一种拥有, 时间更长, 那就是自己的生命,永恒的生命,自己的灵魂, 那因为他永恒, 所以会更重视他, 而不会为了换房子,车子,以及情感的对象而把他卖了, 让他失去人应该有的人性跟灵性, 这在<<庄>>学里面是不可能的因为他最重视的, 就是这一块喔! 

他说第二个层级的知识呢? 是把焦点放在这个世界上五官能感受的具体事物, 那当然, 我们活在这个世界, 我们能看到的一切, 它有它独具的学问, 那你非常重视, 但你不认为它们有分类和分界. 比方说, 你今天对待人的学问跟对待动物的学问, 你觉得都是很重要的, 或者说你今天爱养人跟爱养我们房子旁边的一片竹林或者树, 你觉得都是非常重要的, 你没有这样的一个界限之别. 

可是接着呢, 你以为有界限之别了, 可是你还没有那种是非观念, 那我想举一个比较我们平常不会留意的例子, 我们都知道在公领域要保持整洁不要随便丢东西, 可是你回到家, 你回到你房间呢? 你的外套要落地的时候, 你的袜子要落地的时候, 你的书本要落地的时候, 你是不是就这样丢? 唉, 这我家, 我房间耶, 我爱怎么样, 我还要遵守你学校规矩那一套? 其实这是不是一种分别心呢? 

有一次我读丰子恺的散文, 它里面讲到他跟他房间之间的感情, 然后他说啊, 他房间一段时间要整理一遍, 然后排列好之后, 他有一种感觉喔, 就是他觉得他的房间又恢复了宇宙秩序, 然后他会坐个椅子在房间的中间, 慢慢欣赏, 像欣赏星罗棋布一样. 我每次看到这篇文字的时候, 我就好感动, 我想哪一田我们家能整理到这个境界, 不知道多美?  

我有时到一个朋友的研究室, 那个朋友就到达这种境界, 他研究室里任何一个盆栽或什么, 他放的角度都没有更好的了, 就那个盆栽就觉得美极了, 每个地方都这样. 我觉得很好奇, 有一天我终于机会来了, 我到他们家去玩, 也认识了女主人, 我那天的角色是办外烩的, 因为有一个外国的学者来台湾, 然后他们要在家里请那个人吃饭, 他老婆做菜的经验比我少一点, 所以我就过去他们家帮助做菜. 可是我那天学到很多, 虽然我是主厨, 他老婆是二厨, 我学到什么? 我发现即使在厨房里, 每个东西要落地, 他老婆都没有过渡的地点, 懂这意思吗? 就这个垃圾就是直接到垃圾桶, 碗盘就是直接到哪里, 没有那种先随便放这种概念, 我看了非常震惊, 然后他们家的每一个地方, 那简直就是一个中华民国唯一找不到蟑螂的地方, 你会有那种感觉, 就是森罗棋步啊.

后来, 我就问他太太, 我说: 你怎么办到的? 我那时才发现, 难怪她老公房间也这样, 她告诉我原来他们上过一个课, 什么课你们知道吗? 茶道. 他们学过茶道, 她说茶道老师会教, 所有东西要落点, 绝对不放在过渡的地方, 就是回到它该放的地方, 所以她已经养成习惯了. 他们学过茶道后, 每样东西都变成这样, 各位同学, 这样的东西, 你也许觉得你在谈恋爱,找对象,或是工作的时候没有人会管你, 可是这却影响你一生的生活品质. 

很多欧洲人来台湾玩, 他们回去写了很不友善的文章, 然后我们的政府都很气, 他们说: 台湾人住的地方有的很像猪圈, 就很乱. 可是你想, 确实我们是不是没有养成好习惯? 可是这习惯是谁要养成啊? 大家都有责任, 我们自己要去管理我们自己, 一放假, 去买本 <<断舍离>> 来读读, 你们家马上东西会丢掉很多, 会整齐很多, 丢了很多, 你会发现地变很多, 所以你富有很多. 因为你本来住二十坪的房子, 你乱塞乱塞, 你只剩无坪的财产, 懂吧? 另外哪里有责任啊? 任何人到日本都知道, 日本的小学, 老师就会训练他们每一个人怎么样做一些自己的餐,怎么样做一些家事. 可是在台湾很奇怪, 万般皆下品, 只有读书高, 每个家长都不会炫耀我儿子好会做家事喔,好会做菜喔, 都说我儿子又考第一名了, 我女儿会弹钢琴耶, 所以每个会考第一名跟弹钢琴的最后都住在猪圈里, 是这样吗? 可是刚刚讲了为什么会这样, 因为没有是非嘛, 因为在家房间乱不用罚钱, 懂吧? 不犯法. 在台大校园乱丢纸屑, 刚好遇到我就, 咔嚓, 给你拍下来, 马上找校警.

\begin{center}
	{\color{magenta} 是非之彰也, 道之所以亏也.}	
\end{center}

\vspace{-0.5cm}

可这世界上有更多的是非观念它其实不一定好, 它其实不一定好, 就好像所谓的多元价值, 多元价值真的那么好吗? 

我们有很多我们觉得对的事情, 比方说, 大家不要给学生什么框架, 打到偶像,打倒圣贤, 让孩子自由发展, 发展成他自己喜欢的样子, 真的是这样吗?

我们没有共通的价值吗? 健康不是我们共通的价值吗? 心灵的平和不是我们共通的价值吗? 所以很多东西是可以进一步思考的喔.



\begin{center}
	{\color{magenta} 道之所亏, 爱之所以成. 果且有成与亏乎哉? 果且无成与亏乎哉?}
\end{center}

\vspace{-0.5cm}

道之所以亏损, 就是因为一己有这么多的偏私跟喜爱. 

最后庄子要问的就是: 有绝对的成就跟亏损吗?

在这个世界, 我们刚刚讲的, 所以人世间认为的成败是什么? 很多人看<<贾伯斯传>>, 我一大堆创业感兴或有成的学生最不爱看<<贾伯斯传>>的就是那个结尾, 他们没有办法忍受贾伯斯怎样在死前, 好像对于他的丰功伟业觉得有点后悔, 觉得好像人生不应该那么努力地工作, 因为我学生崇拜他不就是他的工作吗? 可是很多人临死的时候会有这么一遭, 我觉得我在追求什么? 所以人间的成败到底是什么? 还是没有所谓的成败呢? 

到底一个本本分分的, 为别人服务, 哪怕你的职业是计程车司机, 还是一个小菜贩, 你这一生最后是比较有光辉的? 还是一个有很高的位置, 可以毒杀很多人的人呢? 

\begin{center}
	{\color{magenta} 有成与亏, 故昭氏之鼓琴也; 无成与亏, 故昭氏之不鼓琴也.}
\end{center}

\vspace{-0.5cm}

{\color{blue} 有成有亏}, 我们为什么说会有成败? 当我们听昭文谈琴的时候, 我们说有成败了, 因为好听和难听嘛. 可是今天如果昭文不弹琴了, 我们知道他的心吗? 

中国说: 大孝论行不论心, 论心自古无完人呐. 我不知道他心底想些什么.

我不知道你们是不是跟我一样看过一部日本电影叫<<砂之器>>, 里面演一个非常杰出的音乐家, 想要娶一个家世非常显赫的女子, 后来有一天, 他好害怕这女子知道他有一个麻疯病的爸爸, 结果他毒杀了他的爸爸, 那如果这样这个人有成就吗? 这个人算人吗? 

所以我觉得庄子这个学问让我们重新反省了, 在世俗价值里面我们觉得的成, 真的是成吗? 我们觉得的败, 真的是败吗? 他帮助我们去思考, 我们过去没有去思考的事情喔.

\begin{center}
	{\color{magenta} 昭文之鼓琴也, 师旷之枝策也, 惠子之据梧也, 三子之知几乎! 皆其盛者也, 故载之末年.}
\end{center}

\vspace{-0.5cm}

{\color{blue} 昭文之鼓琴也}, 不管是昭文的琴艺非常杰出也好, 师旷他的打击乐, 枝策, 枝就是拄, 就是拿着, 这个策, 就是打鼓棒, 就是击节枝, 师旷做打击乐非常地出色, 惠子呢, 伏案苦思, 有他有名地逻辑思想. 

{\color{blue} 三子之知几乎}, 这三个人的知识或者技艺, 各位同学, 技艺也是一种知识, 当代非常重视的一个默会之知,具身认知就是一种技术的知识喔. 他们都是表现得非常杰出得, 都登峰造极了, 所以{\color{blue} 故载之末年}, 所以才会流传后世嘛. 

\begin{center}
	{\color{magenta} 唯其好之也, 以异于彼; 其好之也, 欲以明之彼. 非所明而明之, 故以坚白之昧终. 而其子又以文之纶终, 终身无成.}
\end{center}

\vspace{-0.5cm}

可是, 各位同学, 当一个技艺这么出色, 他就会显得与众不同, 他就会很想彰显, 让别人知道, {\color{blue} 欲以明之彼, 非所明而明之}, 可是在庄子看来, 这不是一个人的一生最值得努力的,最值得提升的,最值得在意的, 所以他觉得可惜了, 有人呢就在石头到底是白的,还是硬的?  这样的一个论辩里面度过一生. 

有的人爸爸弹了琴, 儿子就觉得把琴弹好, 弹到台北第一,台湾第一,世界第一就是他这辈子最重要的任务了. 但是有时候没才份, 搞得一辈子没有成就, 或者有成就, 可是当他不弹琴的时候, 他有可能是一败涂地的.

\begin{center}
	{\color{magenta} 若是而可谓之成乎? 虽我亦成也. 若是而不可谓成乎? 物与我无成也.}
\end{center}

\vspace{-0.5cm}

庄子问了: 如果这样的话, 这些人可以说是有成就的吗? 那我, 这个我就是泛指之我, 不是说只有庄子, 那我们也可以算有成就了. 那如果说这样不能算有成就了, 那世间万物跟我们有可能都不算有成就了. 当我们觉得非常有成就的人, 我们去想想他所有的产出对这个世界人的影响, 比如说脖子驼了, 比方说眼睛花了, 比方说造成光害了, 那真的是成就吗? 

\begin{center}
	{\color{magenta} 是故滑疑之耀, 圣人之所图也.}
\end{center}

\vspace{-0.5cm}

很多的成就或是非, 我们都可以重新思考一次, 所以庄子说: {\color{blue} 滑疑之耀, } 滑就是乱, 是水流, 光照水流非常混乱的纹路, 水流一样紊乱涌现的眩目光芒, 那是<<圣人之所图也>>, 圣人轻视的, 圣人不在乎, 也不想追求的. 

很拽啊, 一大堆镁光灯的焦点, 然后呢? 你对这世界真的做出贡献了吗? 
 
\begin{center}
	{\color{magenta} 为是不用而寓诸庸, 此之谓以明.}
\end{center}

\vspace{-0.5cm}

{\color{blue} 为是不用}, 所以今天我们未必要汲汲营营地让自己成为一个工具, {\color{blue} 而寓诸庸}, 是将生命寄托在一个日常职业里, 在里面陶冶自己地心灵.

各位同学, 我不知道为什么这样的思想, 好像深刻地影响日本人, 日本人来台湾表演相扑, 就相扑嘛, 两个很壮地人推来推去, 当然是日本的国计, 不可小看, 可是他们就会写: 技,道,心, 相扑这个技术里有道, 道就是万物共通的道理, 可以值得一生追求, 最后扣紧最重要的是心, {\color{blue} 此之谓之明}, 这样的境界就是庄子追求的生命的光明, 各位同学, 莫若以明, 莫若以明, 以明, 以明, 此之谓以明, 明照, 照之于天. 

{\Large {\color{purple} 万物为一---自我,亲人,朋友,故乡,国家,世界,哪里是你关怀所及的边界?}}

\begin{center}
	{\color{magenta} 今且有言于此, 不知其与是类乎, 其与是不类乎?}
\end{center}

\vspace{-0.5cm}

接着庄子说, {\color{blue} 今且有言于此}, 很有意思, 其实到刚刚为止, <<齐物论>> 的庄子已经把儒墨的是非稍微评价一番, 对不对? 最后就是像一阵鸟鸣嘛, 道隐于小成嘛, 还没有达到道的最高境界嘛. 那儒家,墨家头上敢动土, 太岁爷头上敢动土, 动完土还敢讲你自己的言论, 这不是太冒险了吗? 所以我们看他怎么开口. 

他说啊: {\color{blue} 今且有言于此}, 也许我今天在这儿也要提出我的学说了, {\color{blue} 不知其与是类乎}, 不知道我讲的学说是刚刚属于对的那一类? {\color{blue} 其与是不类乎}, 还是跟刚刚我说对的那一类不一样的一类呢? 

\begin{center}
	{\color{magenta} 类与不类, 相与为类, 则与彼无以类矣.}
\end{center}

\vspace{-0.5cm}

最后庄子说了: 不管我是属于刚刚说过的对的那一类, 还是不对的那一类, {\color{blue} 相与为类}, 我都是其中的一类啊. {\color{blue} 则与彼无以类矣}, 其实我跟刚刚用小鸟的鸣叫声来譬喻的那些先秦诸子没什么不同. 

其实各位同学, 你听到这儿一定很讶异, 他怎么没有一幅我要超越他们, 这是自家所独创, 他为什么没有讲出这样的语言, 这么谦卑呢? 你待会会看得出来, 谦卑的智慧了.

\begin{center}
	{\color{magenta} 虽然, 请尝言之, 有始也者, 有未始有始也者, 有未始有夫未始有始也者.}
\end{center}

\vspace{-0.5cm}

他虽然这样, 我讲的话可能也不值一听, {\color{blue}请尝言之}, 但还是容我试着跟大家说说吧, 那他要说什么呢? 他这么谦卑的态度, 他接下来第一个要批判的你们认为是谁呢?

{\color{blue} 有始也者}, 有人探讨宇宙的开端, {\color{blue} 有未始有始也者}, 有人探讨的呢, 是在宇宙开始之前, 还没有开始, 开始之前的状态. {\color{blue} 有未始有夫未始有始也者}, 甚至有人研究那个连开始都谈不上的状态.

其实我相信你听到这儿觉得有点昏了. 如果你觉得有点昏, 有点无聊, 那就达到庄子的目的了, 因为他就是不要走上形上学这条路, 他怕你还不够昏, 怕你的脑子挺形上学的, 所以他讲另一个面相.

\begin{center}
	{\color{magenta} 有有也者, 有无也者, 有未始有无也者, 有未始有夫未始有无也者.}
\end{center}

\vspace{-0.5cm}

他说啊: {\color{blue} 有有也者}, 有人穷究万有, 有的人呢, 探究在万有存在以前, 那个空无一物得阶段, 然后呢, {\color{blue} 有未始有夫未始有无也者}, 有人探究得是连空无一物这个概念, 都还没有得状态. 

你看, 我连念都要打结了, 你们知道这个形上学多难解啊. 甚至更讨论更早于连无都不存在之前得假说. 

\begin{center}
	{\color{magenta} 俄而有无矣, 而未知有无之果孰有孰无也.}
\end{center}

\vspace{-0.5cm}

其实他在这边要达到得一个论述的目的是, {\color{blue} 俄而有,无矣}, 在这些不断往形上发展的研究之后, 我们忽然察觉, 这个世间有,有跟无, 天下万物生于有, 有生于无, 道生一, 一生二, 二生三, 三生万物. 

这时候在你心里开始 OS, 你读过老子喔. 

{\color{blue} 而未知有无之果孰有孰无也}, 可庄子说: 我们又不知道我们认定的有, 是不是真有? 

我有这个人的爱情了吗? 如果我真的拥有, 我怎么会失去? 

你真的拥有你的手, 所以你的手才不会就剁下来借给别人了吗? 

你就问说: 那我真的拥有我自己了吗? 

我真的拥有我自己, 那怎么有一天这个形躯会被你最爱的人烧成灰, 放到骨灰坛里去? 会不会连这个情绪我们也不曾拥有? 那我们认定的有, 真的没有吗? 那我们认定的无呢? 

子不语怪力乱神, 可是好像我说了, 又会有一些办案人员告诉我们他为什么办案, 甚至托梦啦, 什么之类的, 莫非世间真的有在生命, 你出生之前就有你这个人的灵魄, 死后又有这个人的魂魄呢? 还是这是个科学不讨论的命题呢? 还是它是科学可以否定的命题呢? 



\begin{center}
	{\color{magenta} 今我则已有有谓矣, 而未知吾所谓之其果有谓乎? 其果无谓乎? 夫天下莫大于秋豪之末, 而太山为小;}
\end{center}

所以最后庄子说的是, 现在我说话了, {\color{blue} 而未知吾所谓之其果有谓乎? } 我根本不知道我说这些话是不是有意义. {\color{blue} 其果无谓乎? }, 还是根本没有意义? 

他最后做了这样一个惊人的结论,{\color{blue} 夫天下莫大于秋豪之末, 而太山为小}, 养宠物的人都知道, 宠物在秋天长的毛, 超好摸的对不对? 细细软软的, 天底下没有比这个细小的, 庄子说天底下没有什么比这个巨大, {\color{blue} 天下莫大于秋豪之末, 而太山为小}, 所以什么是重要的? 什么不重要的? 什么是大? 什么是小? 其实都是相对的. 
\vspace{-0.5cm}

\begin{center}
	{\color{magenta} 莫寿乎殇子, 而彭祖为夭. 天地与我并生, 而万物与我为一. }
\end{center}

\vspace{-0.5cm}

{\color{blue} 莫若乎殇子}, 还没有成年就死, 中国古代叫殇, 可是庄子却说没有人比他长寿, 我们当然就会想起, 这世界上很多有名的人, 他可能很短就结束了生命了.

著名的音乐家 Mozart, 或者说中国最有名的诗人李白吧, 就现在而言, 也不过活个五六十岁, 哪算长寿啊? 可是他很长寿, 李白活五六十年就过去了, 可是世世代代,古今中外, 李白的诗已经被翻成二十六种语言了. 有这么多的人, 像我的老师他研究李白, 他觉得他就用他有限的人生, 又陪李白走完一辈子.

哇, 你忽然觉得李白超长寿的, 对不对? 而且他旅游的空间好远喔, 有二十六个国家, 都听他作品发表.

{\color{blue} 而彭祖为夭}, 彭祖相传活八百岁, 八百岁怎么会短命? 来, 我们说说吧, 各位同学, 你对彭祖有什么了解? 你说, 老师, 不就活八百岁吗? 再多说一点吧, 喔, 没有, 活八百岁, 翻过去吧, 彭祖不就是史书这样一页一句, 那真是个短命鬼啊. 

各位同学, 那到底什么叫长命? 什么叫短命呢? 

有一个日本著名诗人的俳句, 他的内容就是这样: 人的一生你以为有几年? 当你扣掉你襁褓时期在母亲怀里, 扣掉你穿着开裆裤, 留着鼻涕在那边追跑, 还不知道人生要干嘛, 扣掉你生病的时间, 扣掉你睡眠的时间, 哎呀, 最后一句, 他说, 原来我的人生只有十五年哪.

哇, 听了觉得好震撼啊, 我好小的时候读这个俳句的时候就觉得好震撼. 在现在读来, 人是黄昏, 我现在是生命的黄昏了, 读来更觉得震撼, 我们的时间是这么有限, 怎么还有时间去计较那些无关痛痒的是非呢? 你生命中最核心该致力的致力了吗?

其实这就是庄子要跟我们讲的, 如果你能免除寿夭跟死生的分别, 视死生如一, 我们之后会读到非常多庄子告诉我们, 生死一如的桥段. 比方说, 一个人的生跟死就像四季, 就像一天的白天跟黑夜, 如果死亡只是黑夜, 明天天还会亮.

你还存在在宇宙中, 只是不是用这个形躯不是这个样态, 那么一己的生命就跟天地等长了, {\color{blue} 万物与我为一}, 如果你连看你们家的宠物, 你们家周边的一棵树, 或者不只你们家附近, 这个城市, 这个土地,这个地球,一切资源, 就像你最爱的, 你刚买的 3C 产品一样的话, 或者你就像古代的诗人白居易, 他写的对联: 心中怀念农桑苦, 耳里如闻饥冻声. 所有困苦的人都在他的胸怀, 而不只是开心自己银行存折多少? 自己的成绩怎么样? 如果你能达到这样一个没有人我分别心的境界的话, 他人的成就那就好像自己的成就. 

我有很多很好的朋友, 在以前我常常熬夜很忙的日子, 他们都会给我一些人生的指点. 他们告诉我: 你就做你觉得你来做最合适的事吧, 如果别人也能做的, 你就让给别人, 这样就不会太忙. 好, 你让给别人了, 一件你很想做的事, 就像我小时候到底要念美术系, 还是要念中午呢下考虑好久. 后来, 我高中跟她一起画画的最好的朋友, 她考上师大美术系, 我读了中文系, 她大三那年, 邀我去她宿舍, 就我们见面, 去她宿舍聊天, 我看到她的画, 我记得我离开的时候心里有淡淡的哀愁, 高中的时候, 我美术成绩并不是不如她的, 可今天我狠狠被甩在后面了.



\begin{center}
	{\color{magenta} 既已为一矣, 且得有言乎? 既已谓之一矣, 且得无言乎?}
\end{center}

\vspace{-0.5cm}

可是如果今天你是读<<庄子>>的人, 你好开心喔. 那个我来不及涂满的画纸, 就是我最好的朋友帮我涂满了, 那其实我们对任何事情如果能这么想, 而他人亏损, 你今天有房子, 你今天不是农夫, 可是你看到一块土地莫名其妙被征收了, 有人一天内失去他的家园, 甚至于他是一个在他的农业已经得到国家的荣誉,国家奖项的农夫, 他就这样不再能务农了, 你觉得非常悲伤, 那样的悲伤就好像你自己经历一样, 庄子说: 如果说你达到这样的境界, 那这个世间万物就是你, 你就是万物, 你就跟万物一体了. 

\begin{center}
	{\color{magenta} 一与言为二, 二与一为三. 自此以往, 巧历不能, 而况其凡乎! 故自无适有, 以至于三, 而况自有适有乎!}
\end{center}

\vspace{-0.5cm}

如果有一天, 我们真的能跟道合一, 不再有分别了, 那我们还得用话语去诠释吗? 可是, 一旦有哲人高士用语言为我们说了, 说出像天地万物跟我是共生一体这样的语言来, 那又怎么能说是没有说呢? 好像开始说了: 道生一, 一生二, 二生三, 三生万物, 我们一旦说出道是什么, 那道跟道德诠释就分别成两个了, 是两个不同了, 有了一根二就产生三德概念. 哎呀, 从此以往不断推算下去. 

各位同学, 我们所有的学问就是这样衍生出来的. 在东汉的时候, 我们中文系的人, 我们读到很多的知识分子,儒生, 皓首穷经啊, 他想把一本书读通, 读得头发都白了. 各位同学, 我觉得读通<<庄子>>是一件很美好得事, 每次备课心情都很好, 来上课心情也很好, 有时候如果觉得今天心情没太好, 就是太久没教<<庄子>>了.

可是世界上很多学问不是这样, 你越读越觉得焦虑, 你读到后来不知道自己为什么要读, 不知道世界为什么要产生这样的东西. 我们开始去思考, 人生哪些忙碌是白忙呢? 

{\color{blue}  故自无适有, 以至于三,} 从无到三都这么复杂, 这么累了, 更何况要从适有, 然后还要推到万有, 人最后都被淹死了, 识的追求, 是值得我们反省的.

\begin{center}
	{\color{magenta} 无适焉, 因是已?}
\end{center}

\vspace{-0.5cm}

最后庄子说: {\color{blue} 无适焉}, 别再说了, 适, 就是前往, 说得越多, 离道越远, 不要再往情推进了, {\color{blue} 因是已}, 到这里就好, 到这就好. 

\newpage 


{\Large {\color{purple} 圣人怀之---学习体谅与包容, 才能理解对立得彼端,照见事情得全貌.}}

\vspace{0.25cm}

那什么是道呢? 庄子重新去反省一次. 同学念一下, 我们加把劲, 越过这个山头:  夫道未始有封... 

好, 先念到这, 我们先讲

\begin{center}
	{\color{magenta} 夫道未始有封, 言未始有常.}
\end{center}

\vspace{-0.5cm}

{\color{blue} 夫道未始有封,} 他说一开始, 大道还没有被分门别类. 

各位同学, 讲这话在先秦诸子时代你很能了解嘛. 

你翻开每一家都谈道, 可是他们认为宇宙最重要得道理是不一样的, {\color{blue} 言未始有常}, 刚开始呢, 真理也没有被说定.
\begin{center}
	{\color{magenta} 为是而有畛(zhen)也. 请言其畛: 有左,有右, 有伦,有义, 有分,有辩, 有竞,有争, 此之谓八德.}
\end{center}

\vspace{-0.5cm}

可是后来人的心开始有了分别, 有了界限, 并诉诸言辞, {\color{blue} 请言其畛}, 庄子说: 让我来说说这些分别吧.

各位同学, 庄子要开始叙述所谓的八德.

我第一次读觉得真好笑, 这怎么会是八德? 八德, 忠孝仁爱信义和平啊, 庄子说八德是什么? 有左,有右,有伦,有议,有分,有辩,有竟,有争, 此之谓八德.

你说感情他是开儒家的玩笑?   

可是你听完会有另一番想法喔, 

有尚左, 有尚右, 我亲爱的同学, 彭美玲教授, 她的博士论文就是写, 在中国的古代, 有多少在儒家文明里面, 哪些东西是尚左的, 哪些礼仪是尚右的, 各位同学, 这是厚厚的一本博士论文哪, 所以他讲的儒家, 有尚左,有尚右, 这是八德, 也不能说他错. 

有论,有议, 有时候人会不带价值地论说一些事, 有时候会讲一些, 这样是对的, 这样是错的, 跟庄子同一个时代, 从来没有碰过面, 我们一直觉得他们交锋一定非常精彩的孟子说, 无父无君, 是禽兽也, 对不对?

有分,有辩, 我们开始对很多的事情有了区分, 什么叫华夏? 什么叫蛮夷? 什么叫君子? 什么叫小人? 

有竞, 有争, 竟是竞逐, 争是争辩.

庄子说: 这就是八德.  

后来我们想想, 整个儒家文化对于后代的影响, 不要讲后代, 讲庄子的时代吧, 演门有亲死者, 有个地方有亲人死了, 这孝子苦得很伤心, 哭到肋骨都凸出来了, 哎, 君王知道了这样的孝子应该要表扬啊, 给他盖一个表扬他的牌坊吧. 不得了, 这消息一出, 从此家里死了爹娘的小孩就一直哭, 可不知道为什么, 他们哭的时候一直看肋骨, 怎么还不凸出来?  不然再做一点仰卧起坐就好了, 那已经失去了她孝顺的本质了.

这世界上很多的德性, 到最后有时候好像只是一场表演似的, 所以这是针对既有文化的一些流弊的一个论述喔. 那真正的圣人到底应该怎么样? 

到底是要像墨家的明鬼还是儒家的远鬼呢? 

\begin{center}
	{\color{magenta} 六合之外, 圣人存而不论; 六合之内, 圣人论而不议; <<春秋>>经世, 先王之志, 圣人议而不辩. 故分也者; 有不分也; 辩也者, 有不辩也.}
\end{center}

\vspace{-0.5cm}

我们来看, 庄子说: {\color{blue} 六合之外, 圣人存而不论}, 太秒了, 他说: 我们身处在东南西北上下的宇宙之中, 在这个可知,可感的经验现象世界之外的存在. 我们承认它存在, 但是我们不讨论.

庄子的身心技术人人可学, 为什么人人可学? 因为这个不是宗教, 他说{\color{blue} 六合之外, 圣人存而不论,} 喔不去论, 一论就打架了, 懂吧? 

我们就不知道基督教的教堂跟佛教的西方极乐世界, 怎么长得不太一样? 那怎么会这样呢? 所以我们不谈天堂地狱, 一谈, 那可能就流于空疏.

然后呢, {\color{blue} 六合之内}, 在天地之间, 我们的感官可感的, {\color{blue} 圣人论而不议}.

我们说: 唉, 这同学, 他讲话总是会夹带一些有趣的,比较粗糙的语言, 让我们在台湾大学可以感受各种阶层次的生活, 就这样, 就好了. 你不用去说: 这样讲话真不好, 真失格啊! 你也不用去说: 这样讲话超好的, 超好的, 超台的, 评价不要太多. 

所以各位同学, 你看电视的时候就要判断了, 这个新闻台, 到底是在新闻报道还是新闻评论呢? 是不是有时候很多的名嘴或者是新闻人, 会像上帝一样告诉我们那个人的动机呢? 这时候你就要问: 他真的知道吗? 还是他自己想到? 

庄子说: 既然对于这个世界的一切这么地谨慎与客观. 只说了一件事: 当你每天都要注意其神凝, 你要注意形如槁木,心如死灰, 你要做很多有意义的事, 你的人生真的不要花在这些无聊的事情上.

好, 可是我们既有的风俗习惯呢? 我们有既有的地方文化呢?

{\color{blue} <<春秋>>经世}, <<春秋>> 这本书记载的是先王治理天下的遗志啊, 搞儒家的听到我讲地方文化会气死喔. 他们觉得东海有圣人出焉, 此心同也, 此理同也. <<春秋>> 里面的是经世之道是可以放诸四海而皆准, 对不起我们是上<<庄子>>喔, 所以请暂时容许喔这么说. 庄子说啊: <<春秋>> 这部书里面记载的先王治理天下的遗志, 身为这个文化的后人, 对于这个是非评论, 圣人已经议论的, 我就不与之争辩, 直接接受了. 

哎呀, 多可爱,多亲切的一个学说啊, 不跟任何人打架. 我们的风俗习惯认同了, 我就跟大家一样, 和光同呈, 这里该是一夫一妻, 我就一夫一妻, 不该劈腿就不要劈腿, 懂吧? 当你人生最重要的目的是提升你心身境界, 哪还有那么多的时间拿来劈腿? 一个人有几条腿呢?

{\color{blue} 故分也者, 有不分也}, 庄子最后说: 这个天下是不必分的, 你分到最后, 一定有分辨不清楚的地方, 你是蓝的还是绿的呢? 我学生以前偷偷叫我六八九, 那听说是蓝的, 可是听说蓝的如果不一直选蓝, 就会被叫浅蓝, 可, 浅蓝再进化一点会被叫中间选民, 拜托! 当你去这样讲的时候, 听说哨子一吹, 蓝绿就会归队, 绝对不要相信这样的话, 也不要这样对待这块土地的人, 我们是同胞百姓, 这里不是动物园, 不要把别人当动物. 人都是有思想的,有价值判断,有眼睛的, 你相信自己的同时也要相信别人喔. 你一直去分, 你一定有分不清楚.

{\color{blue} 辩也者, 有不辩也}, 它到底是飞鸟? 还是老鼠? 你们看蝙蝠, 所以不要把人作太简单的归类, 相信每个灵魂每个人都是非常独特的.

\begin{center}
	{\color{magenta} 曰: 何也? 圣人怀之, 众人辩之以相示也. 故曰: 辩也者, 有不见也.}
\end{center}

\vspace{-0.5cm}

{\color{blue} 曰: 何也?}, 为什么我们不去分辨呢? 不去分辨那么仔细呢? 

{\color{blue} 圣人怀之}, 庄子说圣人的胸怀, 他能够包容一切的分别, 如果这一切都是你所关爱的, 你有缘分际会的, 那你不必再去分, 他是不是你的人, 是不是跟你同一党, 选举的时候是不是跟你投同一个人, 有必要分那么清楚吗?

{\color{blue} 故曰辩也者, 有不见也}, 当你觉得你分得非常清楚的时候, 你一定有看不清楚的地方, 因为每一次的分类, 我们会找到分类的原则, 换一个原则, 同类的就变不同类了. 

比如说, 一对情侣, 我们咋看, 哇, 好登对喔, 帅哥美女. 可是你后来了解了他们, 哇, 价值观差好多喔, 这两人实在太不合了. 所以不要以为自己分清楚了, 还是有很多分不清楚的的地方.

{\Large {\color{purple} 酌焉不竭---如何陶养心灵, 使拥有源源不绝的灵感,丰沛焕发的生机?}}

\vspace{-0.5cm}

\begin{center}
	{\color{magenta} 夫大道不称, 大辩不言, 大仁不仁(亲), 大廉不嗛(qian), 大勇不忮(zhi).}
\end{center}

\vspace{-0.5cm}

最后庄子说: 最伟大的道理, 是没有办法用言语说尽的, {\color{blue} 大道不称}, 甚至于辩论就想要服人吗? 可是最能服人的人, 居然靠的不是言辞. {\color{blue} 大仁不仁}, 最伟大的仁德大仁不亲, 他不会对某一个仁特别仁厚, 不会对某一个人特别亲近. 只要你今天还很有私心, 这个人找我帮他, 我好好帮, 这个人不错, 念大学就是要开拓人脉, 现在这社会上也有这样的知识在书店贩卖, 不是吗? 什么几岁以前你一直请别人吃饭, 因为为了培养人脉, 以后事业才会成功, 可是庄子的价值是, 真的达到很高境界的人, 当有人找你帮忙, 你是一视同仁的, 为什么一视同仁?  因为那五分钟是你生命中的五分钟, 你都尽全力在帮助一个人, 不会因为是甲或是乙, 不会是你心仪的对象或是另一个人, 而有所不同, 因为你都尽力了.

{\color{blue} 大廉不嗛}, 一个真正廉洁的人, 也不一定让别人看到他谦虚的身影, 他可能就这样走过去了, 你没有看到他在孔融让梨, 在那边推, 把红包往外推, 没有看到这些姿势. 

{\color{blue} 大勇不忮}, 真正勇敢的人呢? 他也不会让你看到逞凶斗狠.

\begin{center}
	{\color{magenta} 道昭而不道, 言辩而不及, 仁常而不成, 廉清而不信, 勇忮(zhi)而不成.}
\end{center}

\vspace{-0.5cm}

{\color{blue} 道昭而不道}, 道被说明白了, 就跟道不同了. {\color{blue} 言辩而不及}, 言语说出来了, 总有欠缺完整的地方. 

一个人仁德, 很规律, 这个男朋友, 实践儒家的仁, 对待他的家人,他的女友都很好, 对待女友很好, 每天早上送早点, 这样怎么会是仁呢? 如果有天你上学的路上遇到一个老婆婆, 很需要救助, 所以那天你不能给你女朋友送早点, 那你也该去啊, 可是你为了这个早点已经送了一年了, 刚好是三百六十五天的第三百六十五天, 你不想有一天的缺憾, 所以你就不管那个老太婆了, 那你的仁就不周全了. 

{\color{blue} 廉清而不信}, 一个人整天穿的破破的, 别人送他什么礼他都不收, 你会觉得有点可疑喔, 在这个时代, 所以当老师有时候我们真的遇到的学生很可爱, 老师这是我自己做的两个面包, 请你收下, 你就要收了, 如果你说: 喔, 不行, 这是一种贿赂, 你还在我的修课任内, 你不可以这样, 这也太不符合人情了吧. 你不要因为两个面包影响你的判断不就好了.

{\color{blue} 勇忮而不成}, 你一旦逞凶斗狠, 可能一事无成.

\begin{center}
	{\color{magenta} 五者圆而几向方矣.}
\end{center}

\vspace{-0.5cm}

庄子说: 上述这五个境界都达到圆的境界, 就非常高了, 但还没有达到最高的, 我们用来表述道德方德境界.

\begin{center}
	{\color{magenta} 故知止其所不知, 至矣! }
\end{center}

\vspace{-0.5cm}

很多事情不一定要往前探求, 要懂得在不知道, 或没有必要追究的地方停止那就行了. 很多的是非, 如果跟你没有太多密切相关, 不要浪费太多时间.

\begin{center}
	{\color{magenta} 孰知不言之辩, 不道之道? 若有能知, 此之谓天府.}
\end{center}

\vspace{-0.5cm}

接着庄子就要说, 可是有谁知道呢? {\color{blue} 孰知不言之辩}, 一种不需要言辞就能说服别人的言论到底是什么? 不用讲很多, 就一两句话, 甚至于没有声音, 你就被说服了.

{\color{blue}不道之道}, 有一种道理, 这世界上一定有很多的道理, 它还没有被说出来, 我们在不同的年代, 我们觉得世纪绝症是不一样的, 因为有时候会发明新药, 对不对. 同样我们觉得最高阶的物理学, 能量不灭, 什么量子力学, 它也会随着时代的迁移而改变, 所以谁能真正了解那些还没有被说出来的大道呢? 

庄子说: {\color{blue} 若有能知, 此之谓天府}. 如果有能知道, 哎呀, 那就是所谓的天府了.  什么是天府? 各位同学, 我们现在一看到府, 这个府是脏腹. 你说: 每天买菜都会看到的, 就是我们台湾人很多人是吃内脏的, 对不对? 这个猪心,猪肝, 就是人的脏腑嘛, 可是如果你有读过 <<黄帝内经>>, 你知道东方对脏腑的概念. 

我的博士论文 <<身体与自然>> 谈的就是 <<黄帝内经>> 这本书里面有一个章节, 是写脏腑的. 那脏腑在东方的概念里面, 不是只是一个器官, 有些功能, 这个脏腑里面也是会藏着精藏着气, 藏着血, 还有藏着神的.

简单讲, 我们的整个灵魂, 在东方的观念里面, 在中国传统医学里面, 灵魂也跟身体一样, 我不会说: 各位同学举起你的身体来, 我必须说举起手来, 对不对? 那当我们去指称脏腑的某一部分的时候, 在东方的身体会说, 跟这个心重合的这个部分的灵魂叫做神, 那跟肺, 在肺这部分那就叫魄了, 在肝这部分叫魂了, 在肾脏这部分我们称他为志, 就变成同样一个灵魂, 他在各个不同的脏腑也有不同的称谓, 

我在这边为什么要强调这边, 这边讲的天府, 不是说: 哎呀, 怎样的肝脏是天府吗? 不是这意思, 是讲我们的心灵, 他有一个最原初的样态. 这个最原初样态的心灵, 其实是可以让你去参透这个世间更高深更高妙的道理的, 这样的一种心灵境界, 你不管往里面增加什么, 增加什么, 它都不会满出来. 

这点是让我们了解, 在庄子的观念里面, 人的心, 是何等辽阔的存在喔, 可能会让同学想起, 我们上次在讲到, 这个瑜伽修炼, 瑜伽修炼里面, 他在做冥想做 meditation 的时候, 他要我们想象: 我们的灵魂超越了我们这个空间, 包住了这栋大楼, 包住了整个台湾大学, 包住他根本不管他是不是快要竞选的台北市, 然后包住整个台湾, 包住亚洲,包住地球, 甚至于可以怀抱整个太阳系.

如果我们的心灵境界能扩及至此的话, 你不用担心我一直扩充会不会像气球爆炸, 不会.

\begin{center}
	{\color{magenta} 注焉而不满, 酌焉而不竭, 而不知其所由来, 此之谓葆光.}
\end{center}

\vspace{-0.5cm}

他说啊: {\color{blue} 注焉而不满}, 我学太多东西, 我的记忆体会不会有限? 不会, 它是无穷无尽的喔, 你要取用它里面的智慧. 以前我们一些写诗的朋友都会聊, 唉, 你这一季怎么样书写呢? 写哪些东西呢? 有几首作品呢? 这时有人会开玩笑: 不能写太多, 写太多脑汁就用尽了.

可是庄子告诉我们, 所谓的天府的心灵, 你不管怎样取用, 那智慧, 那点子, 那创意, 好像是无穷的, 当我们这样去了解庄子的心灵, 大家对于学习 <<庄>> 学可能有更高的意愿, 对不对? 

如果你喜欢设计, 那你更要多读 <<庄子>> 了, 因为它可以让你的心灵变得非常澄明, 你有很多的创意永远都用不完, 这样的一个状态, 而且, {\color{blue} 而不知其所由来}, 我也不知道喔是为什么有这样的点子. 前些时候, 我一个诗人朋友曾淑美, 她出版了她的新书 <<无愁君>>, 那我跟她聚会的时候, 我就问起她写诗的感觉, 她告诉我, 她说有时候, 她是在心情不好的时候去写一首诗, 可是这写诗的内容并不是去记载她心情不好的这个事件, 她好像想提升自己到另一个想法, 到另一个情感的层次. 

她说: 说都抽象, 又具体一点, 我觉得我在接收这个宇宙意识. 我觉得用一个非常现代的语言去解释刘勰(xie)讲的神思, 我们的灵感从哪里来? 其实我们不知道, 我有时候也会有这种感觉, 这个礼拜就是要写出你对 <<庄子>> 这个段落的心得, 可能备课的时候要用到, 那你会非常努力地让自己地心声, 可能我身体先做穴道导引, 我的心注意在我地眉心,印堂或者丹田, 我让我自己进入一种完全没有烦恼, 没有杂念地状态. 嘿, 我觉得这时候我写稿最顺利, 会觉得, 哇, 很有灵感. 然后有时候写出来觉得: 这真的是我想的吗? 我真的能想出这么好的东西吗? 你开始有点怀疑, 你懂吧, 所以我想这就是我们从事 <<庄子>> 身心修炼的人, 大概能体会的一点点道理喔, 就是说你真的拥有某一种智慧, 你觉得你在进步, 可是你好像不是你用你的理性去想出来的东西, 庄子说: <<此之谓葆光>>, 他说那就像一种若有似无, 明亮又不耀眼的光芒.


{\Large {\color{purple} 万物皆照---如何放下文化沙文主义?}}

这个段落, 我们知道 <<齐物论>> 非常地长, 我常觉得上 <<庄子>> 最难熬地就是 <<齐物论>> 了, 如果你的 <<齐物论>> 能够不打瞌睡地听完, 那你绝对后面就是轻舟已过万重山了, 非常轻松喔, 那 <<齐物论>> 我们分成四个大段落, 这四个大段落是我为了教学方便而分的, 第一大段落是南郭子綦, 对不对? 你们看过洪金宝的电影吗? 我们那时代的一个谐星, 每次看他的电影, 就在欣赏一个人为什么这么胖, 但这么灵活? 我看他的电影回来, 我住宿舍, 我住在下铺, 我的同学要爬上二楼, 每次都把我吵醒, 如果我比她早睡, 所以培养我晚睡的习惯, 当然是个借口啦. 我都会跟她说: 嘿, 轻点学洪金宝跳天鹅湖那一段, 轻点上去. 可是很难, 有的人步履就是很沉重, 那如果可以我们也好想要, 就算体重再重, 但很轻灵, 那一定脂肪很低嘛, 一定身体气血状况很好嘛. 那重有何不可呢? 所以我们在南郭子綦这段, 我们看到一个 <<庄>> 学圣人的典范.

然后第二段呢, 我们就想那我们为什么不能跟他一样? 我们进入莫知所萌, 对不对? 那莫知所萌, 哇, 人之生也, 固若是芒乎? 其我独芒, 而人亦有不芒者乎?  我们大家都陷溺在这个可怕的世界里了.

然后到第三段, 最长的一段, 庄子告诉我们的就是莫若以明, 其实你可以的, 你只要让你的心灵到一个太阳跟月亮, 我们每天透过窗, 透过抬头, 我们看到的太阳跟月亮, 你真的有一件想不开的事, 你就到太阳, 月亮的高度, 你假想, 你就是王国维诗里偶开天眼觑红尘, 可怜身是眼中人, 那个眼中人, 你就不只是可以体谅你自己, 你可以体谅你身边的每一个人, 也许前一秒钟觉得他带给你麻烦, 你下一秒钟又觉得他好无助, 他愿意撑到今天才带给你麻烦, 已经很忍耐, 很委屈了. 你真的对世界可以有完全不同的想法, 你就不觉得今天什么事耽搁了, 你觉得没关系, 这也是一个生命很真实人生的一部分, 然后你该做的事赶快继续做, 然后你对于以前会动心的现在不会, 你会很开心喔, 那莫若以明就教我们这个. 他用不同的语言, 什么得其环中啊, 道枢啊, 照之于天啊, 其他跟我们讲得两行啊, 跟我们讲的无物不然, 无物不可啊, 其实他讲的就是同一件事嘛. 就是好像已经可以帮我们抚平生活大小事里面可能有的情绪,烦恼,创伤, 可是最后一件事, 其实也是很困难的, 就是我们是怎样去看待异文化, 我们是怎么样去看待不同的国家的? 

喔记得有一年, 那时是李登辉总统, 然后跟总统夫人曾文惠女士, 他们到我们的友邦参访, 那当然电视是会播出的. 我记得我那时候看到画面, 就觉得: 啥? 怎么会这样呢? 总统下飞机了, 当然跟那块土地的总统握手, 可是我们曾文惠女士好忙啊, 为什么? 总统有很多的老婆排成一排, 所以曾文惠得一个一个握, 我那天看电视觉得: 啊? 我们的友邦, 我觉得我的心里就有一种文化沙文主义, 你们知道吗?  怎么这友邦, 这老婆怎么这么多啊? 而且怎么我们的总统来, 你们还穿一个大露肩呢? 

其实这就是一种, 你面对异文化很容易有的姿态, 有一种人类学家, 他们比一般人类学家还要人类学家, 因为人类学家是什么呢? 就是我们会拿一个, 我们去关怀土著嘛, 就是西方世界当它发展得非常好得时候, 我们要关怀土著文化啊, 所以我们不只有了解普世得价值, universal 的价值, 我们要了解一些 local, 一些第三世界,一些土著的观点, 因此有人类学的诞生, 听起来很有爱心对不对? 是你再转念一想, 他已经界定了他自己的文化就是普世的价值, 别人的文化, 你那地方观念, 我了解一下, 所以其实人要摆脱这样文化沙文, 真的非常困难, 你自己要非常小心, 我自己要非常地小心. 

儒家文化你们懂的, 华夏跟蛮夷, 对不对? 你想你知书达理的, 然后我偶尔最近到韩国旅行, 一看到韩国人还穿着汉服, 在重要的仪节, 看来挺开心的, 感觉我们华夏文明被广远, 然后到了日本, 日本的成年礼, 那也是受到我们中国加冠礼的影响, 其实当你看了很感动,很开心的时候, 你的文化沙文主义在当中已经萌芽了, 你懂吗? 然后你就去看什么 <<上帝也疯狂>> 这样的电影, 去了解非洲的部落, 他们怎么样面对生活的真实, 可是有一天你真的到山林里去了, 你还真发现, 你认不得这些动物的大便真的还挺困扰的, 因为你不知道靠近你的将是那一头野兽, 对不对? 然后, 我看我台大的学生, 喜欢爬山的学生, 他们参加好多野生户外营, 我说你们学什么? 这次营队的诉求就是认动物的大便, 我咋听觉得非常好笑, 可是后来觉得, 这也是耶, 这到森林里不认得那还得了, 难道你到了非洲, 怎么样不要被鳄鱼咬死, 不要被食人鱼吃了, 不要中什么毒, 难道这件事比怎样打躬作揖来得不重要吗? 其实人可以不断去思考这非常多的问题.

接下来这个小小的段落, 其实就是庄子对于异文化, 跟异国, 跟大国, 跟小国的关系的一个思考, 那当然我们华人文化圈有很多的处境, 需要有新的思想,新的观念, 来帮我们突破这个困境, 但这个新观念不一定来自于新的时代, 有可能它来自古典的一个新价值, 这一段很重要, 同学念一下:

{\color{blue} 故昔者尧问于舜曰: "我欲伐宗,脍,胥敖, 南面而不释然, 其故何也?" 舜曰:"夫三子者, 犹存乎蓬艾之间. 若不释然何哉! 昔者十日并出, 万物皆照, 而况德之进乎日者乎!" }

我觉得庄子这段议论非常有意思. 因为你去熟悉儒家经典的人, 你就知道尧的地位, 尧的地位你要看古典就会更清楚, 因为我们当代大家都知道孔子的地位, 对不对? 可是你去搜寻尧, 在所有古典文献, 他所有的治绩, 你会发现, 哇! 如果你这样看孔子, 你可能要这样看才能看到尧, 你懂吧? 你会觉得所有的儒家文化, 什么孝悌,仁义, 每一个他都告诉你, 是尧教我们的, 敢情孔子是个传播者, 所以尧这样一个人物呢, 我们来看庄子怎样对待他. 所以庄子这些你说 <<庄子>>, 寓言十九, 重言十七, 他挑选这些角色是有含意的, 你看从 <<逍遥游>> 开始对不对? 尧不就给我们跑了三次龙套了吗? 然后他现在又出现了. 

\begin{center}
	{\color{magenta} 故昔者尧问于舜曰: "我欲伐宗,脍,胥敖, 南面而不释然, 其故何也?"}
\end{center}

\vspace{-0.5cm}

有一天啊, 尧就问舜了, 问了他的未来的继承人, 他说啊: {\color{blue} 我欲伐宗,脍,胥敖}, 我之前一直想, 一直想攻打宗, 脍, 胥敖, 这三个小小的国家, 可是有一天呢, 你们看他的德行啊, 对不起这样讲好像失敬了, 我应该说, 你看他的尊容, 有一天他已经做到了, 他已经收了那三个小国家了, 真的坐在北方南面称王了, 该打躬作揖的, 他们都照做了, 该戴帽子的, 他们也戴了, 不该纹身的, 也都尽量穿衣服遮起来了, 怎么我还不开心? 他觉得很奇怪喔. 这个释然, 我们说释怀, 我们的心揪在一起当然不好过, 就放开嘛. 那放开就是怡悦, 就是欢乐啊, 为什么我已经战胜了, 华夏文化已经衣被了蛮夷之邦, 我怎么还不开心啊? 请问这是为什么? 你看这段话多么重要, 在这个时代.

 
\begin{center}
	{\color{magenta} 舜曰:"夫三子者, 犹存乎蓬艾之间. 若不释然何哉! 昔者十日并出, 万物皆照, 而况德之进乎日者乎!"}
\end{center}

\vspace{-0.5cm}

舜回答: {\color{blue} 夫三子者}, 他说这三个小小国家啊, {\color{blue} 犹存乎蓬艾之间}, 他们就安然生活在那个小小的蓬蒿艾草丛中.

各位同学, 你看到这, 你马上想到, <<逍遥游>> 翱翔于蓬蒿之间, 那些小鸟, 对不对? 为什么小国譬喻成小鸟. 那尧呢? 大国呢? 儒家文化矗立眼前, 懂吧. 

所以我们要说的是, 他这边其实就是对儒家文化的一些反省喔. 他说这个小小的国家, 在蓬蒿艾草丛中这些卑贱之地, 你打赢他们, 你为什么不开心呢? 

我们来探索一下, 他就说了, 他说啊: {\color{blue} 昔者十日并出, 万物皆照}, 他说听说遥远的时代, 有十个太阳, 这十个太阳一起高挂天上的时候, 世间万物通通被照耀了, 不是因为你今天给了太阳政治献金, 不是因为你今天帮太阳做了一首竞选歌, 也不是因为你在他最需要的时候去帮他喊, 冻蒜或是帮他站台, 甚至于你有时候还骂了太阳几句, 可是它还是照耀你了, 他说这是太阳啊, {\color{blue} 十日并出, 万物皆照}, 这是太阳的德行啊. 可是我们知道儒家的教育, 儒家圣人的自我要求, 太阳再伟大也只是个物, 而人呢? 人是万物之灵啊, 所以你的德行,你的包容,你的普照, 应该要超越太阳的, 那怎么会这样呢? 什么叫怎么会这样? 那你怎么还会有非拿下不可得国家呢? 为什么要那个国家一定要跟你一样? 用一样得制度, 贴一样得标签, 或是一样得国徽? 这样你才可以呢? 这是舜对尧的一个反问喔, 就是 {\color{blue} 十日并出, 万物皆照}. 所以你知道这个照之于天的概念, 如果真的每一个大国有这样的胸襟, 这个世界会减少很多杀戮喔. 

那也可以教导我们怎么样去看一个异文化, 当你真的一个人类学家, 你要去探访一个土著观点的时候, 你不应该是以一个当代文明的,普遍性的尺, 拿着去丈量你们这些到底合不合这个尺的规格, 而是你放掉这把尺, 你重新认识一个文明, 有没有可能这个文明它的高度, 比你以为的西方当代的文明还具有普世价值? 

你要抱着这个可能性, 你才能充分地了解, 我看一些搞人类学的人研究中医, 他们也会说: 中国古代医生他们想象人有多少条经络, 你懂吗? 你们知道他这么叫 local 观念, 你们一听就听得出来, 那其实人要没有成见真的很难, 我觉得这是一生学习得道路, 但你没放掉一点成见,你会更开心, 因为你觉得你的世界更开阔了, 那这就是我们讲 <<莫若以明>>, <<齐物论>> 我们分五个大段, 最难通过的一个大段, 因为它文字最长, 所以通过这段, 后面就轻松了. 

{\Large {\color{purple} 问题与思考}}
\vspace{-0.5cm}
\begin{enumerate}
	\item 试着在你原本缺乏好感的人身上, 发现一件令人喜欢或值得欣赏,尊敬的事.
	\item 如果可以一年三百六十五天都是你最喜爱的春天而没有夏,秋和冬, 如果二十四小时都是你最爱的夜晚而没有白昼, 你觉得好吗? 
	
	那么在下一次面对即将引爆的争执时, 你能否像习惯黑夜与白天, 像面对春夏秋冬更迭般, 安然自在地接受不同地观点? 
\end{enumerate}
%

	 



%%%%%%%%%%%%%%%%%%%%%%%%%%%%%%%%%%%%%%%%%%%%%%%%%%%%%%%%%%%%%%%%%%%%%%%%%%%%%%%%%%%%%%%%%%%%
\beginrefs
\bibentry{1}{ 蔡璧名}, 
``正是时候读庄子: 庄子的姿势,意识与感情'',
{\it 天下杂志出版 },
2015.
\endrefs

\begin{flushright}
	\tiny \kaishu \today  \  北京.
\end{flushright}

\end{document}

              
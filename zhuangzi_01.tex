%%%%%%%%%%%%%%%%%%%%%%%%%%%%%%%%%%%%%%%%%%%%%%%%%%%%%%%%%%%%%%%%%%%%%%%%%%%%%%%%%%%%%%%%%%

%%%%%%%%%%%%%%%%%%%%%%%%%%%%%%%%%%% Author:Yao Zhang  %%%%%%%%%%%%%%%%%%%%%%%%%%%%%%%%%%%%
%%%%%%%%%%%%%%%%%%%%%%%%%%%%% Email: jaafar_zhang@163.com  %%%%%%%%%%%%%%%%%%%%%%%%%%%%%%%

%%%%%%%%%%%%%%%%%%%%%%%%%%%%%%%%%%%%%%%%%%%%%%%%%%%%%%%%%%%%%%%%%%%%%%%%%%%%%%%%%%%%%%%%%%

\documentclass[11pt]{article}
\usepackage[utf8]{inputenc} 
\usepackage[table]{xcolor}
\usepackage[most]{tcolorbox}
\usepackage[left=2.50cm, right=1.50cm, top=2.0cm, bottom=2.50cm]{geometry}
\usepackage{xcolor,url,cite}
\usepackage{amsmath,amsthm,amsfonts,amssymb,amscd,multirow,booktabs,fullpage,calc,multicol}
\usepackage{lastpage,enumitem,fancyhdr,mathrsfs,wrapfig,setspace,cancel,amsmath,empheq,framed}
\usepackage[retainorgcmds]{IEEEtrantools}
\usepackage{algorithm}
\usepackage{algorithmic}
\newlength{\tabcont}
\setlength{\parindent}{0.0in}
\setlength{\parskip}{0.05in}
\colorlet{shadecolor}{orange!15}
\parindent 0in
\parskip 12pt
\geometry{margin=1in, headsep=0.25in}
\usepackage{subfig,graphicx,framed}
\graphicspath{ {img1/} }
\usepackage{ctex}
\usepackage{multirow, tabularx}
%%%%%%%%%%%%%%%%%%%%%%%%%%%%%%%%%%%%%%%%%%%%%%%%%%%%%%%%%%%%%%%%%%%%%%%%%%%%%%%%%%%%%%%%%%

\newtheorem{theorem}{Theorem}[section]
\newtheorem{definition}{Definition}[section]
\newtheorem{exercise}{Exercise}[section]
\newtheorem{note}{Note}[section]
\newtheorem{notation}{Notation}
\newtheorem{lemma}{Lemma}[subsection]
\newtheorem{proposition}{Proposition}[section]
\newtheorem{example}{Example}[section]
\newtheorem{homework}{Homework}[section]
\newtheorem{summary}{Summary}[section]
\newtheorem{corollary}{Corollary}[section]
\newtheorem*{remark}{Remark}
\makeatletter
\@addtoreset{equation}{section}
\makeatother
\renewcommand{\theequation}{\arabic{section}.\arabic{equation}}
\usepackage[colorlinks,linkcolor=blue, anchorcolor=green,citecolor=red,urlcolor=blue]{hyperref}


%%%%%%%%%%%%%%%%%%%%%%%%%%%%%%%%%%%%%%%%%%%%%%%%%%%%%%%%%%%%%%%%%%%%%%%%%%%%%%%%%%%%%%%%%%

\def\beginrefs{\begin{list}%
		{[\arabic{equation}]}{\usecounter{equation}
			\setlength{\leftmargin}{0.8truecm}\setlength{\labelsep}{0.4truecm}%
			\setlength{\labelwidth}{1.6truecm}}}
	\def\endrefs{\end{list}}
\def\bibentry#1{\item[\hbox{[#1]}]}

\def\UrlBreaks{\do\A\do\B\do\C\do\D\do\E\do\F\do\G\do\H\do\I\do\J
	\do\K\do\L\do\M\do\N\do\O\do\P\do\Q\do\R\do\S\do\T\do\U\do\V
	\do\W\do\X\do\Y\do\Z\do\[\do\\\do\]\do\^\do\_\do\`\do\a\do\b
	\do\c\do\d\do\e\do\f\do\g\do\h\do\i\do\j\do\k\do\l\do\m\do\n
	\do\o\do\p\do\q\do\r\do\s\do\t\do\u\do\v\do\w\do\x\do\y\do\z
	\do\.\do\@\do\\\do\/\do\!\do\_\do\|\do\;\do\>\do\]\do\)\do\,
	\do\?\do\'\do+\do\=\do\#}

\renewcommand{\algorithmicrequire}{\textbf{Input:}}  % Use Input in the format of Algorithm
\renewcommand{\algorithmicensure}{\textbf{Output:}}
%\renewcommand{\figurename}{\kaishu 图}
%%%%%%%%%%%%%%%%%%%%%%%%%%%%%%%%%%%%%%%%%%%%%%%%%%%%%%%%%%%%%%%%%%%%%%%%%%%%%%%%%%%%%%%%%%
\begin{document}
\kaishu 
	
\setcounter{section}{0}


\begin{center}
	
	{\Large \kaishu 正是时候读庄子 }
	
	{\kaishu 讲者: 蔡璧名} \quad { \small  Scribe: Yao Zhang}
		
\end{center}

\vspace{-0.75cm}

\section{\kaishu 逍遥游 \ \ 北冥有鱼}

设定怎样的人生目标才能幸福逍遥?

\setcounter{subsection}{-1}

\subsection{\kaishu 前言}

\vspace{-0.5cm}

驴子的葫芦卜, 在外面.

鸟儿的虫子,果子在外面.

凤凰的枝头, 在外面.

大鹏的梦土, 也在外面.

爱情,财富,房子,车子,位子,人的一生啊, 更是充满了在外面的憧憬, 在外面的向往, 在外面的诱惑, 在外面的目标...

在外面的, 等待.

等待他迎向你. 等待你追上她,或他,或它. 等待终于挣得,拥有! 彷佛人生原本充满缺憾, 正凭籍完遂诸般有待于外的追求, 才能过于完整.

树者不然. 鸟儿飞行的目的, 可能是树. 但树的目的, 欲是它自己. 星霜夜露, 四时雨风, 有生之年, 所有的风景, 都助成它年复一年壮大的年轮, 茂盛它渐得庇阴相逢者的浓阴. 那树, 它孤独吗? 抬头一看, 才发现不知何时起, 树, 与同样朝天空缓缓伸展的邻树, 啊, 林树 --- 十年? 百年? 千年?  --- 早已连理,交枝,合抱了. 

\subsection{\kaishu 其正色邪---标杆典范,谁说了算}

来, <<逍遥游>>. 同学念一下: {\color{blue} 北冥有鱼}

{\color{blue}  北冥有鱼, 其名为鲲. 鲲之大, 不知其几千里也. 化而为鸟, 其名为鹏. 鹏之背, 不知其几千里也. 怒而飞, 其翼若垂天之云. 是鸟也, 海运则将徙于南冥. 南冥者, 天池也.}

我们应该先看这个题目, <<逍遥游>>. 

各位同学, 如果你能是你生命的主宰, 而和林徽因不一样, 林徽因在不能跟徐志摩在一起的时候, 她曾经感慨她觉得人只是上帝手中的一枚棋. 如果你们都是你们生命的主宰, 你希望你的台大四年, 是逍遥游四年还是水深火热四年? 

其实正常我们都要逍遥游, 所以庄子的第一篇就讲出一个所有人类的人生目标, 我们都希望好快乐地过一辈子, 而他让他第一个出场地人是大鹏, 其实你要知道大鹏, 是怎么样一只鸟? 最好地办法就是你跟它一起飞. 

然后你在跟它一起飞, 你开始就觉得很怪, 你不知道它为什么一直飞, 因为你习惯读 <<诗经>>, <<诗经>> 开始, 关关雎鸠, 在河之洲. 窈窕淑女, 君子好逑. 为什么叫关关你知道吗? 我读中文系第一次念到关关的注解, 我笑了. 关关的意思啊, 他说就是公鸟叫一声关, 母鸟就应和一声关, 所以关关雎鸠是非常浪漫的事. 

那你就不懂, 为什么庄子那么不会抓住观众的心, 我们知道漫画要吸引人, 就要有爱情,有热血,有反派, 它怎么一开始就一只孤零零地在那飞啊. 可是有可能, 这就是我们生命的本质吧, 一个人的长跑, 孑然一身, 一个人的长跑可能是我们一生的样子, 而孑然一身有可能是我们来到世界跟有一天要离开世界, 最初跟最后的样子, 所以我们就别怪大鹏鸟是单飞了.


\begin{center}
	{\color{magenta} 北冥有鱼, 其名为鲲.}
\end{center}

\vspace{-0.5cm}

{\color{blue} 北冥有鱼}, 北当然是北方, 冥就是看不清楚. 各位同学, 你有没有想过, 你每一天的追求, 你追求的那个东西, 是别人看得到的吗? 还是看不到? 我们在学术圈有时候同事之间开玩笑. 我说, 我们在学术圈的老师, 每个人都好像一只梅花鹿, 我们的价值好像被我们身上的点点给决定了. 那你呢? 你生活当中有没有花时间在一种追求, 其实你追求之后, 它是很难量化, 很难数据化, 很难在聚光灯下被看到, 那就叫做冥, 就叫做冥.

其实我们的一生, 在庄子的整个理论学说里面, 不管是心灵的提升, 或是身体的放松, 你一旦往这个工夫行去, 你在你的专业上的结果, 应该就是会像庄子书中出现的实际的职人一样, 都成为那个行业的翘楚哦. 那样可以让你的心灵非常地平静, 然后整个人变得比较清朗.

那接着呢, {\color{blue} 北冥有鱼, 其名为鲲}, 故事开始了. 一开始, 就是个滑稽的开端. 为什么? 因为你看不出来, 因为你不知道鲲是谁. 鲲啊, 简言之, 小鱼之名, 就是一只很小的鱼. 鱼子, 鱼的孩子, 小鱼, 如果用现代的语言来讲叫吻仔鱼(小鱼苗). 

你觉得非常地荒谬, 这么大的一只鱼, 它叫做吻仔鱼. 这有什么荒谬? 我若跟你们说, 哎呀, 老师前几天跑去登山, 遇到一个很荒凉的地方, 很天然的地方, 居然看到一只怪兽, 你知道吗. 这时候你们就突然问老师, 是什么动物啊? 我说它的名字叫蚂蚁, 你们一定笑翻了, 怪兽很大只, 蚂蚁这么小. 

这就是庄子刻意搞的, 让你去想, 为什么这么大的动物, 它的名字这么小, 你们待会会知道为什么, 有一段揭牌的时候, 我现在先不要爆雷.

\begin{center}
	{\color{magenta} 鲲之大, 不知其几千里也.}
\end{center}

\vspace{-0.5cm}

{\color{blue} 鲲之大, 不知其几千里也.} 这鲲实在太巨大了, 好几千里, 没有人看到它的全貌. 

\begin{center}
	{\color{magenta} 化而为鸟, 其名为鹏.}
\end{center}

\vspace{-0.5cm}

有一天, 它化为一只鸟, 它的名字叫做鹏. 各位同学, 你们知道什么叫做鲲化为鸟吗? 这绝对不是你有一天告诉别人说, 我大一的时候五十公斤, 大二的时候六十公斤, 大三七十公斤, 大四八十公斤, 这是个体重的变化, 是个物理的变化. 可是鲲化为鹏是什么? 是个体质的变化, 是化学的变化, 是一个脱胎换骨的变化. 

各位同学, 我好希望你们这一年庄子都是一个脱胎换骨的变化, 你一年之后回头看, 觉得这一年的我有了一些长进, 你喜欢现在的自己, 希望是这样的一个结果哦. 

那我们说的是, {\color{blue} 化而为鸟, 其名为鹏}, 它的名字叫做鹏. 这鱼大, 化为鸟, 鸟也大.

\begin{center}
	{\color{magenta} 鹏之背, 不知其几千里也. 怒而飞, 其翼若垂天之云.}
\end{center}

\vspace{-0.5cm}

{\color{blue} 鹏之背, 不知其几千里也.} 它的背好几千里那么长, 甚至更长.

{\color{blue} 怒而飞}, 各位同学, 这个怒写成粉嫩的怒, 但这个字是努力的意思. 

你说, 大鹏鸟, 都这么大了, 还要努力飞哦? 那不是随便翅膀一拍, 咻就很远了吗, 但不是的, 不是这样的, 其实天才这个东西是什么? 

我现在不敢随便赞美一个学生. 我多年前教诗, 班上一个男生, 第一次交出来的作业, 我惊为天人. 我就告诉他, 这位同学, 你是天才啊, 你的才分远远超过老师在你这个年龄的时候. 那个男生笑得好灿烂, 他对我灿烂一笑, 我永远记得他的笑容, 就坐在这里, 我今天还记得他的名字以及那份作业的内容. 

天才耶, 连老师都不如我, 我上她的课干嘛呀? 他再也没有来, 以致于他在期末考当他交出他的期末作业, 那绝对不是平仄不合而已, 他的整个布局,结构以及抒情, 他的情感的酝酿与蓄积远远不如班上二分之一的同学.

这个故事告诉我们什么, 告诉我们天才啊, 你只要能够持续发挥你的才能, 珍惜它, 你就有机会绽放你的光芒, 达到某个成就, 只要,只要, 其实不是只要, 是必要. 也就是如果你没有办法珍惜,发挥你的才分而且持续地让它成长的话, 那最后天才将隐没在人群当中成为庸才. 

所以{\color{blue} 怒而飞}这个怒字, 让我们看到, 即便它是这么大的一只鸟, 它也要非常地努力来飞翔, 它地翅膀像从天垂下来地云这么地长, 而这只鸟, 这样就够了吗? 还不止. 

\begin{center}
	{\color{magenta} 是鸟也, 海运则将徙于南冥.}
\end{center}

\vspace{-0.5cm}

{\color{blue} 是鸟也, 海运则将从徙于南冥.} 什么叫海运啊? 当大风海动地时候, 它就要迁徙到遥远地南冥去了. 那大风海动是什么呢? 各位同学, 你想, 一艘船要出航, 它要涨潮地时候, 对不对? 退潮就水不够深啊. 那这只大鸟它要起飞, 它要有足够地水量跟风, 它才能往南走.

那南冥是哪里呢? 我们讲这个冥就是一个悠远的,看不清楚的, 这个字, 甲骨文身上这样写的, 有一块布, 盖住了一个东西, 这个东西被两只手拿着, 这叫冥, 有另一个说法, 说这个冥字是分娩的娩的意思.

冥跟娩之间其实有共通的意涵. 在中国的两大思想--儒家跟到家, 他要达到的最高境界, 一个是专气致柔, 能婴儿乎? 一个是赤子之心, 人皆有之. 每一个人要从这个大人者, 不失其赤子之心者也, 是非之心,辞让之心,羞恶之心,四端之心, 每一个人都有, 你把天生都有的东西扩充之, 所以飞向那个出生的地方. 这样好像也能解释, 如果大鹏是代表儒家, 是尧或者孔子的话.

\begin{center}
	{\color{magenta} 南冥者, 天池也.}
\end{center}

\vspace{-0.5cm}

{\color{blue} 南冥者, 天池也}, 它就是要飞到一个广阔无垠的地方, 那里叫做天池.

我觉得这个 <逍遥游> 很有意思, 这么大的鸟, 这么大的鱼, 变成这么大的鸟. 其实我们会用神话来看待这个故事, 可是庄子不希望这样, 庄子希望, 好像它是只尼斯湖大海怪, 不只一个人看过,两个人看过,有三个人看过, 所以叙述了三次, 想要增加你相信这个故事.

好, 同学往下念: {\color{blue} 齐谐者, 志怪者也}

{\color{blue} 齐谐者, 志怪者也. <<谐>> 之言曰:“鹏之徙于南冥也, 水击三千里, 抟扶摇而上者九万里, 去以六月一息者也.”  野马也, 尘埃也, 生物之以息相吹也. 天之苍苍, 其正色邪? 其远而无所至极邪? 其视下也, 亦若是则已矣.}



\begin{center}
	{\color{magenta} 齐谐者, 志怪者也. 谐之言曰:“ 鹏之徙于南冥也, 水击三千里, 抟扶摇而上者九万里, 去以六月一息者也.”}
\end{center}

\vspace{-0.5cm}

{\color{blue} 齐谐} 是一个人的名字, 他专门记录一些光怪陆离的事, 其实庄子讲这段话好像让我们相信真的有这只大鹏鸟. 

他说{\color{blue} 鹏之徙于南冥也}, 当大鹏要飞往遥远,遥远的南方, 那这个徙字我想着墨地说一下哦. 

我常常听到, 谈恋爱的人会说: 我就喜欢你原来的样子, 你什么也不要改变, 你做你自己就好, 这是我们现代人最爱讲的话, 对不对?  

这就表示这个人绝对不是中国人, 我说那个文化中国的中国人, 什么意思呢? 因为整个文化, 中国文化的精髓哦, 不管是儒家或道家, 儒家说: 吾日三省吾身, 为人谋而不忠乎? 与朋友交而不信乎? 传不习乎? 你每天透过不断反省自己, 不断反省自己, 然后不断提升自己, 不断提升自己, 你希望你不断在变化当中, 而且这变化是不断往好的方向进行. 道家思想, 我们刚讲真阳之气, 我今天借一点真阳之气, 我希望我不断地提升, 不断地充实, 每天都比昨天更好.

那我们为什么要跟一个人说: 我们相爱, 你就用原来地样子就好. 为什么不是我们地相遇, 有一天我们回头, 在看这样一段, 师生之谊也好, 朋友之谊也好, 情人也好, 我们觉得: 谢谢老天让我遇见这个人, 我觉得因为遇见他, 我的生命更好. 我觉得人最好的缘分应该是这样, 那最好的自己也是像吃甘蔗一样, 每天都渐入佳境哦.

也就是说, 我们会发现在东方的文化, 我们第一堂课讲的, 我们的生理, 我们的心理, 我们都不只是让自己没病就够了, 我们可以从零, 朝着那个数线不断往正向走, 我们的脾气, 我们的个性, 我们的身体, 我们透过内省, 每天都可以更好, 所以这个徙字, 我觉得他是写得很用力的.

{\color{blue} 鹏之徙于南冥也, 水击三千里}, 但庄子要强调的是, 这大鹏要飞往南冥去, 当中如果没有深达三千里地大洋供它起飞, 没有高达九万里地飙风, 什么叫扶摇, 扶摇是疾风, 非常快速地风, 上行风. 我们看古书的注解会想起美国的电影 <<龙卷风>>, 可以就是, 哇, 非常地强, 可以把大鹏鸟一下子卷道天空去.

{\color{blue} 抟(tuan)扶摇而上者九万里, 去以六月一息者也}, 刚刚讲海运, 这边讲六月一息, 凭借着相隔六个月才会碰上一次地大风海动, 它才有办法到那遥远遥远的南方.

各位同学, 如果你大学不管是联招也好, 基测也好, 你那天考坏了, 你能跟考试单位说: 对不起, 我昨天拉肚子, 可不可以今天再帮我出一份卷? 不行, 考期过了就过了. 请等明年吧, 这就是这世界的残酷.

\begin{center}
	{\color{magenta} 野马也, 尘埃也, 生物之以息相吹也. 天之苍苍, 其正色邪? 其远而无所至极邪?}
\end{center}

\vspace{-0.5cm}

如果你要飞的一个目标, 是有待于外在世界的, 那你就要等待一些机缘的配合. 那我们说大鹏, 它飞到高空, 它看见的是什么呢? 这里非常重要, 它说它看到云既像野马一样地奔腾, 像尘埃一般地漂浮, 更重要地是它看到了什么? 我觉得这一点是庄子要提醒我们的, 也是老庄, 道家思想不断阐述的, 为什么功成能弗居, 能够不居功, 因为知道{\color{blue} 生物之以息相吹也.} 我们在这个教室其实我们彼此受着彼此气息的影响, 不相信? 如果今天老师重感冒, 一边讲一边咳, 一边咳, 坐在前面的同学就被我影响了, 那你通常被传染你才被影响, 可是你没有发现, 就在你大一的那一年, 迎新宿营的那一年, 同学说要组织一个什么那一年, 你参加你栽进去就是四年了. 有限的四年, 你一生有几个四年? 可是你就在一个非常不经意的源起, 你就这样进去了, 这就是{\color{blue} 生物之以息相吹也.}

以下讲更重要的一件事, 庄子问:{\color{blue} 天之苍苍, 其正色邪?}  他说天空的颜色苍苍, 真的就是天空真正的颜色吗? 各位同学, 这句话你读起来我不知道你什么感觉, 我读起来就觉得庄子是天下第一骂人高手啊! 他怎么能写得这么不着痕迹, 不着痕迹到我读庄子十年以后才发现这个, 不是达文西密码是 <逍遥游> 密码. 

中国的先秦诸子有一家很喜欢讲正, 人心那要正心, 位子要正, 名分要正名, 肉... 席不正不坐, 割不正不食, 这是哪一家啊? 很熟嘛, 儒家. 为什么这边说, 我跟你们讲哦, 那个苍苍这个天色是真的天色吗? 我跟你说, 那要有一个前提成立才算, 就是这只大鹏鸟它飞到的地方, 远到不能再远.

\begin{center}
	{\color{magenta} 其视下也, 亦若是则已矣.}
\end{center}

\vspace{-0.5cm}

我们再往下看, 告诉我们的天空真的是苍苍, 我就信了你大鹏鸟, 真的是苍苍. 这句话很严厉耶, 你知道吗? 他说如果你飞到的地方, 不是人类的至境, 请不要来给我定义什么叫正.

其实我觉得我们读 <<庄子>> 之后, 会让我们对于过去坚持的不那么固执, 为什么? 因为你对于正就是你觉得这样就是最对的,最好的, 你会问自己, 我达到最高境界了吗? 我是完人吗? 

所以我觉得学 <<庄子>> 的我, 有时候学生会跟我讲一些话. 他说: 老师, 我们同学非常可恶, 他明明知道怎么样, 可是他又怎么样, 可是... 我说: 你怎么知道他明明知道呢? 你是上帝吗? 我常会问这句话, 他就楞住了, 我不是. 我说那你就不要用上帝的口气说话. 

我想这句话不只用在问我问题的学生, 可以用在你在新闻媒体上看到百分之九十的人, 就是我们讲... 其实我们面对很多政治的问题, 乃至于宗教的律法, 你会去想, 到底什么东西不能吃, 是有气血的掐得出血来得不能吃, 还是有些东西, 有些植物, 它所谓得荤腥得不能吃, 那为什么一样是佛教, 这个流派可以吃, 这个流派不可以吃, 然后达赖喇嘛来台湾的时候, 最爱吃的是晶华饭店的牛肉丸. 

这一切到底怎么回事? 那个正到底是什么? 其实你很想,你很想, 去挑战它, 去质疑它, 但是我觉得我们学 <<庄子>>, 最重要的不一定是挑战质疑别人定义的正, 而是挑战自己的成见, 这是更重要的.

\subsection{\kaishu 聚粮待风--成功人士的必备条件?}

我们接着往下看. 你说老师刚强调的, 大鹏鸟要飞那么高, 还不是靠水,靠风, 庄子真的有这些意涵吗? 对, 他一次一次, 一层一层写得非常清楚. 

\begin{center}
	{\color{magenta} 且夫水之积也不厚, 则其负大舟也无力.}
\end{center}

\vspace{-0.5cm}

他说啊, {\color{blue} 且夫水之积也不厚, 则其负大舟也无力.} 如果今天水的蓄积不够深厚, 你就不能乘载起大船了.

\begin{center}
	{\color{magenta} 覆杯水于坳堂之上, 则芥为之舟. 置杯焉则胶, 水浅而舟大也.}
\end{center}

\vspace{-0.5cm}

{\color{blue} 覆杯水于坳堂之上}, 我今天如果倒一杯水, 这教室实在太平整了, 如果有个凹洞我倒下去, 同学, 你在旁边树上摘片叶子, 那不就变成一条船了吗? 

可是有同学说, 哦, 放纸船啊, 那我把我的杯子也放上去, 哇, 整个就卡住了, 这就是因为水浅舟大. 所以那么大的船, 可需要很多的水呢. 

\begin{center}
	{\color{magenta} 风之积也不厚, 则其负大翼也无力. 故九万里则风斯在下矣, 而后乃今培风; 背负青天, 而莫之夭阏(e)者, 而后乃今将图南. }
\end{center}

\vspace{-0.5cm}

{\color{blue} 风之积也不厚}, 水是这样, 风也是这样, 你今天如果风不够强劲, 你连放风筝都无法放, 尤其是大鹏鸟这么大, 要怎么飞得起来, 所以要有九万里得翼下之风拖着它, 支持它在旅途中得每一次展翅, 它才能飞上高峰, 所以它是需要九万里风的, 因为它真的太大只了.

{\color{blue} 而后乃今培风}, 这个培就是凭借得凭, 因此它在风上面, 它才能凭借着风, 背负起广阔得蓝天. 为什么这边要荡开这一笔呢? 我不知道是不是因为大鹏鸟象征得是儒家得缘故, 儒家胸怀天下, 比起汲汲营营于个人得私立跟欲望, 儒家当然是伟大的, 他为了社稷苍生, 为了更多人着想, 我们不就希望有这样的胸怀的人出来管理众人之事吗? 所以他想要拯饥解溺, 想要家齐,治国,平天下, 那样的胸怀他要背负起的不就是这一片天吗? 

{\color{blue} 而莫之夭阏者}, 你看中国历史上多少受到儒家思想影响的人, 岳飞也好, 文天祥也好, 更不要讲, 可能我们在戏剧节目里看到的赵氏孤儿, 他真的想要完成一个他觉得一个臣子该做的, 呀, 他牺牲自己的生命, 牺牲自己子女的生命, 在所不惜.

{\color{blue} 莫之夭阏}, 夭和閼都是停止, 各位同学, 你有做什么事情, 是别人怎么反对你都要做的吗? 我通常在台大学生看到最明显的都是他的恋情, 她的爸妈发对, 她觉得爸妈没眼光, 我非嫁不可, 因为这样来找我的人最多, 很少那种我觉得我非念什么科系不可我爸妈却不同意, 我要坚持下去, 都讲算了吧, 挥挥眼泪就转了系就转了, 通常都是这样. 

所以我们知道大鹏鸟飞得这么远, 它除了要有先天庞大的体型, 它要有坚强的意志力, 还有, 一阵又一阵帮助它的海跟风, {\color{blue}  而后乃今将图南}, 它才能往它梦想的南方飞去哦. 

各位同学, 你刚刚看到大鹏鸟这样的飞, 我不知道你有没有问你自己, 今年, 就要快达到二十岁, 是不是, 弱冠之年的你, 你人生的目标在哪里? 我通常每次在跟同学聊这个问题, 同学都没有把重点放在人生目标, 都在跟我讲, 我二十岁, 我还没, 我才十八, 我十九, 年轻的嘛. 那你的路要在哪里? 

我们刚刚讲完大鹏的目标, 我们现在来看看其他飞禽的目标.

\begin{center}
	{\color{magenta} 蜩(tiao)与学鸠(jiu)笑之曰:“我决起而飞, 抢榆枋而止, 时则不至, 而控于地而已矣, 奚以之九万里而南为?” }
\end{center}

\vspace{-0.5cm}

{\color{blue}蜩与学鸠笑之曰}, 这个蜩就是蝉啊, 这学鸠呢, 就是小雀, 小山雀. 笑之曰, 它们扑哧一笑, 它们笑什么呢?

决起, 决是快飞, 我轻松地很快地往上一腾跃, {\color{blue} 抢榆枋而止}, 这个抢是停止, 我们就停止聚集在榆枋枝头了.

各位同学, 我不知道你们有没有发现, 大鹏鸟地飞翔是一只孤独地飞, 这些小只得都是一群一群得, 你们懂吧。

{\color{blue} 时则不至}, 有时候连这么矮的枝头都没飞好, 掉了下来, 掉下来没关系啊. {\color{blue} 而控于地而已矣}, 这个控就是投啊啊, 就好像一颗球, 被丢在地上而已, 这有什么关系呢, 就像我的猫跌倒一样, 一只麻雀跌倒到泥土,不会,不碍事的.

{\color{blue}奚以之九万里而南为!} 它想这大鹏鸟你有病吗? 这近郊的果子就这么甜美, 你飞到九万里外的高空没得吃, 你要干什么? 其实在每个时代都有这样的人, 我们就想他为什么要往艰难的路走, 为什么要选一条明明没有什么赚头的路哦.

那我们往下看, 蜩与学鸠跟大鹏鸟, 可能是飞翔目标远近的两个极端, 当中还有很多不同的阶段.

\begin{center}
	{\color{magenta} 适莽苍者, 三餐而反, 腹犹果然; 适百里者, 宿舂(chong)粮; 适千里者, 三月聚粮. 之二虫又何知!}
\end{center}

\vspace{-0.5cm}

{\color{blue} 适莽苍者}, 今天如果我们要飞到近郊, 那你要怎么样, 刚刚你发现没有, 讲蜩与学鸠完全不必讲要带多少粮食的问题.

如果你参加联考的时候, 你从小你爸妈就觉得, 啊, 考上了也好, 没考上也好啦, 那就不用去补习了, 你懂吧, 可是如果至少要考上什么学校.

唉, {\color{blue}适莽苍者}, 你要到近郊的草野, 那可能要准备一个餐盒, 要带上三餐的粮食. 这样子你当天回来, 肚子才能, 果然, 肚子才能长得跟水果一样, 你们这个年龄看到这个字还不觉得忧患对不对, 就肚子圆滚滚的, 吃得很饱. 

可是, {\color{blue} 适百里者, 宿舂粮}, 如果你要飞到百里外, 很抱歉, 你可能要花一个晚上的时间在那边边舂捣粮食, 另一个解释是, 因为你要在外面过夜, 所以你就要舂捣更足够的粮食.

{\color{blue} 适千里者}, 那百里都要这样准备了, 何况是千里之外? {\color{blue} 三月聚粮}, 那要花三个月聚集粮食. 

各位同学, 这时候你很理性的不断动脑筋在下面听着, 你觉得老师, 这段话问题很大呀.

请问, 赴千里之外的鸟儿, 它的粮食要放在哪个行李箱里呢? 它要怎么把它叼到千里之外, 你不要这么理性, 这时譬喻懂不懂. 至少我们在这个譬喻里面我们了解到, 你要达到的目标越高, 你要付出的越多. 你的家长要帮你准备的教育基金可能也越多, 他们这一切都不是蜩与学鸠能了解的, 何况是大鹏鸟.

那我想看到这里, 我们一定会想, 我们的人生, 我们飞行的方向的问题, 你会发现他的第一个讲的, 蜩与学鸠, 其实它从零点并没有往外飞, 它就在附近, 想要飞啊, 掉下来就算了, 这时候我们就要想人的一生的飞翔. 

如果你读 <<世说新语>>, 你会认识一个人叫王导, 他是个宰相. 宰相, 了不起吧, 宰相相当于台湾现在的院长, 是吧. 可是后代的人怎么看王导呢?  <<朱子语类>> 说: 王导为相, 只周旋人过一生. 讲得更简单一点, 王导一生, 周旋而已.

以前我在报纸上看到一个小新闻, 他说有一个人他本身当一个庙公,庙祝, 后来有人就鼓励他出来选议员, 因为他当庙祝, 好多人来这进香, 他认识好多人, 人脉可好着, 选着选着就当了个小议员, 可是这个人很知道怎么样带大家去旅游, 票越来越多, 后来就好像当到议长, 当到议长到最后涉到什么案, 最后就没官当了, 别人问那你现在怎么办? 没关系啊, 我回去当庙祝. 你懂吧. 

就是你会发现, 其实, 如果你今天忠肝义胆, 你为了提升那块土地上的人的生活, 你出来管理众人之事, 那你当然很像大鹏鸟一样, 你可能自己过得有点窘迫, 你这条路很艰难, 你要揭发很多的弊案, 你还是得走下去. 可是如果你只是出来混混, 其实还是蛮容易的哦.

所以王导最后他的一生就被人认为说周旋而已, 所以王导当到宰相, 你如果看看我们介绍过的这群飞鸟, 好像跟蜩与学鸠最接近, 对不对? 
 
然后呢, 你再看看刚上大学的你们, 你们这两个礼拜以来, 是不是很忙呀, 你忙啥? 大一最忙了, 忙完迎新营, 忙新生杯, 忙完新生杯就要忙期中考了, 忙完期中考就要加入之夜的演出来送学姐离开了, 然后才送完学姐离开, 就要大二就要办营队来迎新了, 大一就这样匆匆告别. 

我在讲你们, 大一生活的 tempo (速度) 跟王导其实非常地相似, 就我们人有时候觉得我们不断地往前跑, 可是其实我们一直在绕圈圈,绕圈圈, 只是大家一起跑, 你觉得很热闹, 就像海德格讲, 他说我们人德一生很少人活在 I, 活在自己, 大部分都活在 they, 活在他者之中.

有的同学, 我认识一些学生, 在我认识他德岁月里面, 因为他实在太聪明了, 随便一考就很容易考上很好的科系, 所以人生主要的飞行就在魔兽世界里面, 从七级,十几级变打到七,八十级, 变打到无级可上, 再等新的游戏出现. 可是我总想, 他在里面对荧幕不断往前的这些岁月, 他有没有意识到他有可能是背对幸福的? 他有可能背驼了, 也没时间认识女生, 也没时间培养更多的兴趣嗜好.

所以呢, 你的飞行到底在哪里? 不要讲我的学生打魔兽, 我们站在这个讲台, 这个位置, 我如果没有不断地提醒自己, 我很可能也会为了我身为一只梅花鹿, 我上面的斑点, 或许我的教学成绩等等, 而我就这样在讲台上,在文章里,在字迹间, 就把我的背搞驼了, 把我的心牢禁了, 其实这是非常容易的. 

如果我们不读 <<庄子>>, 不去注意时时刻刻都要知道心感受幸福的不是别的, 是心. 享受健康人生的, 必须有健康的身体, 你随时提醒自己: 提升自己的身心技能,身心境界, 不然你很快可能就在一个潮流里, 憔悴了, 老了. 

那讲完这一段以后呢, 庄子更明白地跟我们点出: 你说, 如果小只的跟大只的, 外面都要风风雨雨, 那小只跟大只是不是一样呢? 庄子说不一样. 

 
\begin{center}
	{\color{magenta} 小知不及大知, 小年不及大年. 奚以知其然也? 朝菌不知晦朔, 蟪蛄(gu)不知春秋, 此小年也. }
\end{center}

\vspace{-0.5cm}

{\color{blue} 小知不及大知, 小年不及大年.} 庄子清楚地告诉我们, 不一样. 他说为什么呢? 我们来看, 小的智慧为什么比不上大智慧? 寿命短的为什么比不上寿命长的? 怎么知道是这样呢? 因为朝生暮死的虫子, 这就是朝菌. 朝菌有两个解释, 一个是大芝, 大芝就是灵芝, 那一个是小虫, 我们选小虫的解释, 跟随王叔岷先生的注. {\color{blue} 朝菌不知晦朔}, 它的生命很短, 没几天就死了, 所以它永远不知道什么叫月初, 什么叫月尾, 什么叫上弦月, 什么叫下弦月.

而蟪蛄, 蝉, 我们知道蝉破土而出就只剩下十三天左右的生命, 它如果是春蝉, 这是刚讲的那只虫, 你看这只春蝉, 它看了一个秋天的照片, 它很困惑, 这是什么? 这是它一辈子来不及见到的光景, 这就是年纪小的限制.

\begin{center}
	{\color{magenta} 楚之南有冥灵者, 以五百岁为春, 五百岁为秋;}
\end{center}

\vspace{-0.5cm}

而楚国的南方, 有一个叫冥灵的大树, 它叶子长完整, 叶生需要五百年, 叶枯也需要五百年.

\begin{center}
	{\color{magenta} 上古有大椿者, 以八千岁为春, 八千岁为秋, 此大年也.}
\end{center}

\vspace{-0.5cm}

上古有一棵大椿更不得了, 它叶生叶落都要八千年, 所以它一年就有一万六千年那么长. 你看这棵大树, 所以它遇到什么乱子, 它都老神在在, 它会告诉你这件事古代也发生过, 没什么, 小青年, 冷静下来. 就像我们一样, 道家者流, 盖出于史官, 不是吗? 所以呢, 我们人如果你的眼界不到这, 你不知道有活到这么长寿的大树.

\begin{center}
	{\color{magenta} 而彭祖乃今以久特闻, 众人匹之, 不亦悲乎!}
\end{center}

\vspace{-0.5cm}

你就一直想, 我要跟彭祖一样, 我要跟彭祖一样, 在古籍的记载有一种说法, 活了七百八十岁, 你觉得这样就够了, 你想跟他一样. 

庄子说: {\color{blue} 众人匹之, 不亦悲乎!}, 为什么这样? 因为可能在庄子得时代, 那时候儒家思想是主流, 大家都想要齐家,治国,平天下. 然后呢, 立德,立功,立言, 流芳万世, 好像这样就不得了了. 

你说: 老师, 是不得了呀, 还能更不得了吗? 将来我们 <齐物论>, 当庄子揭牌德时候, 他要达到德境界, 不只是齐家,治国,平天下, 是天地与我并生. 如果在这地球上的每一棵树, 每一只野生动物, 哪怕只是你脚下的泥土, 你对待它就像对待自己生命的感觉, 那我们台大走到哪里, 就不会看到很多筷子套啊, 丢在地上, 当然不一定是台大学生丢的, 有时候假日有很多外面的人, 因为你怎么可能把垃圾丢到你的嘴里呢? 你不会丢到你嘴里就不会丢到泥土里, 因为天地与我并生嘛. 所以你爱这个世界, 是用一种更普及的不只是对人类, 对整个环境, 都有更深的关怀. 然后呢, 你岂止要流芳万世, 你不求流芳万世了, 因为天地与我并生, 所以你就知道, 那个庄子要为我们打开的生命境界, 可能更高远而辽阔的哦. 


\subsection{\kaishu 小大之辩---你怀抱着小确幸还是大梦想?}

我们又说了, 不知道为什么大鹏鸟鲲鹏的故事, 庄子很怕你们以为是假的, 所以他现在要开始在 <逍遥游> 里面叙述他的第三遍, 

{\color{blue} 汤之问棘也是已:} 

{\color{blue} 汤之问棘也是已: 穷发之北, 有冥海者, 天池也. 有鱼焉, 其广数千里, 未有知其修者, 其名为鲲. 有鸟焉, 其名为鹏, 背若泰山, 翼若垂天之云, 抟扶摇羊角而上者九万里, 绝云气, 负青天, 然后图南, 且适南冥也. 斥鴳笑之曰:“彼且奚适也? 我腾跃而上, 不过数仞而下, 翱翔蓬蒿之间, 此亦飞之至也, 而彼且奚适也?” 此小大之辩也.}

当这个故事被叙述第三次, 刚刚有一次, 第一次是庄子直接说, 对不对? 第二次他说, 不是只有我这样说, 齐谐也这么说. 

\begin{center}
	{\color{magenta} 汤之问棘也是已: 穷发之北, 有冥海者, 天池也. 有鱼焉, 其广数千里, 未有知其修者, 其名为鲲. 有鸟焉, 其名为鹏, 背若泰山, 翼若垂天之云, 抟扶摇羊角而上者九万里, 绝云气, 负青天, 然后图南, 且适南冥也.}
\end{center}

\vspace{-0.5cm}

他第三次说, 商汤问他的贤臣棘, 棘也说了这个故事, 这个故事大家知道, 所以相信我吧. 穷发, 什么叫做穷发? 人叫做头发, 大地的头发是什么? 是树木. 在完全没有绿色植物的地方, 在地球的最北, 有个冥海, 什么叫做冥? 我们说冥就是远, 远到看不清楚的地方, 那儿有个天池. {\color{blue} 有鱼焉}, 那鱼好大呀, {\color{blue} 其广数千里, 未有知其修者, 其名为鲲.}  这前面讲过, 我们就快速跳过了.

{\color{blue} 有鸟焉, 其名为鹏, 背若泰山}, 呀, 它的背像泰山一样地雄伟, {\color{blue} 翼若垂天之云, 抟扶摇羊角而上者九万里}, 刚刚是说, {\color{blue} 搏扶摇而上}, 我们说扶摇是上行风, 对不对, 这里讲扶摇羊角, jue 是读音, 我们读一个字我们用读音, 羊 jue. 我们平常聊天, 你看那个人犄(ji)角, 我们不会说你看那个人犄 jue, 不会. 牛角面包, 我们不会说牛 jue 面包. 可是如果今天我们读这篇文章, 我们就要念牛 jue, 所以这地方念羊 jue, 搏就是击, 拍打上行风, 像羊角一样地上行风, 就这样往上飞行, 乘着形如羊角腾卷而上的旋风, 穿越层层云气, 飞上九万里高空. 

有多高? {\color{blue} 绝云气}, 已经完全没有云的地方了, 各位我们搭飞机都知道, 飞到够高的地方就没有云了对不对. {\color{blue} 负青天}, 刚刚讲过了, 为什么要用背负青天来讲大鹏鸟, 齐家治国平天下. {\color{blue} 然后图南}, 最后的目标是南方, {\color{blue} 且适南冥也}, 让我往南方去吧. 

各位同学, 我觉得如果你现在还乘坐在大鹏鸟的背上, 你开始想大鹏鸟要把我载到哪里? 可是我要问你们的是, 不是大鹏鸟要把你带到哪里, 是你自己要把自己带到哪里呢? 我觉得这是每一个同学都要深刻思考的问题, 或者每一个学 <<庄子>> 的同学一定要深刻思考的问题. 

其实人有时候一定阶段会有一个憧憬, 我之前看我班上一个女生的作业, 她可能在某一个阶段她最爱的就是漂亮, 所以那阵子她最爱的就是买衣服. 可是有一天她发现, 她有点失落, 她觉得买衣服, 如果把身体练健康了, 穿什么都好看, 她就开始不喜欢买衣服了, 她就跑去学瑜伽了, 我看一个女生, 她在讲她整个心里历程, 然后她后来遇见庄子, 她讲了一些三个不同阶段的变化.

那其实有时候, 我觉得当你在整理你的房子, 我想这不只是整理衣柜, 我们人一生都会花钱买过一些东西. 我不知道你是否有这样的经验, 当你在整理你的抽屉或者衣柜的时候, 你有没有动过这样的念头想: 我当初怎么会买这个? 我要讲的是, 其实我们在这个年龄, 有一天你会发现就像你觉得我当初怎么会买这个, 你会想当初我为什么花时间在哪里? 可是在那个当下, 你是很陶醉的.

我永远记得有一天, 我忘了我几岁, 我那时候好像, 圣诞节吧, 我忘了哪来这么多圣诞老人, 不知道是吃什么糖果,还是什么拥有的, 我就被一排圣诞老人, 然后用粘土把他做成一整排, 然后在那边精雕细琢那种. 那时候我父亲就走到我身边, 他说: 璧名, 你在做什么? 在玩圣诞老人, 在制作我的手工艺. 我爸笑了: 有一天, 你回头看这件事会觉得非常无聊. 我那一刹那, 我受伤了, 我想: 这么美的雕塑, 明天拿到班上跟同学炫耀一下, 他们觉得酷得不得了, 哎呀, 代沟呀, 你不知道欣赏. 后来哪一年, 当我整理抽屉的时候, 好丑的圣诞老人, 我一把它丢了.

其实人生常是这样, 就看你在什么时候找到你生命那一件真的值得做一辈子的事, 那件很重要的事, 甚至于最重要的事, 那我们人生其实一直在面对选择, 你每一次选一次就错过一个, 其实你不知道你选的那一个更重要的, 错过的是次要的. 

我们面对选择, 我们非常地谨慎, 我现在是面对怎样面对我的每一天? 比方说我今天可以这样,也可以那样, 那我就要想:怎么样的选择, 我的心是最安定的,最静定的,最没有烦恼的, 我的身体感觉是最好的, 我就做这个选择, 如果你不这样去选择, 你会发现, 你永远有一天做最多的运动, 永远有一天做得,过得最好,最充实,最积极, 你们知道是哪一天? 就是明天嘛. 比方说今天应该要念书, 没关系啦, 明天再开始. 今天应该早起, 没关系, 我真的再晚睡一天, 我明天就早起了. 你就会变所有东西推到明天, 这是个非常负面得一个思维哦. 

所以我觉得我们在解释{\color{blue} 南冥也}, 我们跟着大鹏鸟飞到这的时候, 你真的好好想一想, 当然我为什么讲这么多, 因为我就是一个年纪很老的史官嘛. 因为我有时候会接到, 因为我研究医家思想, 我常会接到这样的电话, 我来上课以前, 有朋友打电话给我. 他说: 璧名, 这个我的父亲肝癌第四期, 你可不可以建议一下, 我现在该怎么办. 我每次接到这种电话, 我就会想: 为什么要等到这一天? 包括我自己, 为什么要等到那一天, 你才会彻底反省你的生命, 然后把你生命的本末先后重新布局, 我们可不可以早一点? 我们可不可以早一点点, 在下坡的开端就回头? 其实这就是老庄思想, 持盈保泰的思想.

\begin{center}
	{\color{magenta} 斥鴳笑之曰:“ 彼且奚适也? 我腾跃而上, 不过数仞而下, 翱翔蓬蒿之间, 此亦飞之至也, 而彼且奚适也?” 此小大之辩也.}
\end{center}

\vspace{-0.5cm}

那这时候小只要出场了, 这样的故事讲三遍, 如果小只的都同样两个小只, 那也太无趣了, 就换一只吧. 换斥鷃, 什么叫斥? 斥就是尺泽, 什么是尺泽呢? 大概只有一尺的一个池塘, 小泽, 小泽上的一只小雀鸟, 鶠是小雀鸟. 它看到了大鹏, 你看这只小雀鸟, 我觉得这幅画画得很好, 她把小只的画的真么大, 大只的画得这么小, 因为在小只的心目中, 那大只的就是这么小. 

{\color{blue} 彼且奚适也?}, 唉, 你要飞到哪呢? {\color{blue} 我腾跃而上}, 我们就这样往上一跳, {\color{blue} 不过数仞而下}, 飞个十来尺我就掉下来了. {\color{blue}  翱翔蓬蒿之间}, 我就在矮矮的蓬草, 蒿草之间遨游, 这不就是飞翔的极致了吗? 

各位同学, 你看到这, 你马上知道, 我知道庄子是哪一种人! 其实你不知道, 因为你现在想的一定是那种胸无大志, 你未来干嘛? 都可以, 钱多事少离家近, 最好是上班的时候可以喝茶看报纸, 你想一定是这种, 可是不是哦. 到底是什么? 这点要翻过去才能说, 这只小小鸟, 我们这时候才意识到一件事, 并不是大鹏鸟飞到最远的地方, 小小鸟也觉得它的地方就是最远的地方, 懂吧. 那小小鸟就不会了解, 大鹏你为什么要这么累飞到那么远呢?  

庄子说, 这就是小跟大的差别, 其实你永远不要忘记, 虽然庄子提出他理想的生命境界跟目标, 但是他告诉我们: 小不及大. 在他心目中大鹏鸟还是有崇高的地位的, 所以我有时候蛮佩服庄子的, 他会把他对手的一个角色, 用大鹏鸟这么地俊帅,潇洒, 能够在中国飞行千年, 为这么多文人墨客所喜爱, 你看李白的 <<大鹏赋>>. 他去形塑一个对手的角色, 你就知道他是个多么能看到别人优点的人, 我想这是读庄学我们最能学到的一件事.

揭牌喽, 永远不要忘记, 刚刚讲的 {\color{blue} 此亦飞之至也}, 这只鸟是哪一只? 是斥鶠. 那斥鶠是庄子讲的人类世界里的谁呢? 我们往下看, 同学念一下: {\color{blue} 故夫 ...}

{\color{blue} 故夫知效一官, 行比一乡, 德合一君, 而征一国者, 其自视也, 亦若此矣. 而宋荣子犹然笑之. 且举世而誉之而不加劝, 举世而非之而不加沮, 定乎内外之分, 辩乎荣辱之境, 斯已矣. 彼其于世, 未数数然也. 虽然, 犹有未树也.}

\begin{center}
	{\color{magenta} 故夫知效一官, 行比一乡, 德合一君, 而征一国者, 其自视也, 亦若此矣.}
\end{center}

\vspace{-0.5cm}

{\color{blue} 故夫知效一官 }, 各位同学, 我们做什么事情总是要有一定程度的聪明嘛, 对不对. 所以他说他的智能啊, 能够胜任任一官职, {\color{blue} 行比一乡}, 他的德行, 他的言行, 能够庇荫一个乡里, 这个比是当作庇荫的庇字, 是个假借字. 

{\color{blue} 德合一君}, 他的品德操守适合当一个国君, 这个而念能力的能. 他的能力能成为全国征信什么叫做能力为全国征信? 在我们这个民主有选举可以投票的时代, 那当然他们拿到最多选票, 就等于被最多人信任嘛. 在上一次的总统大选, 就是马英九对不对, 在上一次台北市大选就是郝龙斌对不对, 可是庄子下一句接什么? 我刚刚讲了两个名字, 现在是因为时代不同了, 所以你们听了居然都没有坐直一点肃然起敬, 像听到国歌一样. 在我们小时候, 那听到蒋介石,听到蒋经国都要马上站直,坐直的, 所以我刚到台大第一天, 张亨老师在课堂上, 我们不知道为什么, 就讲到一代完人, 然后张老师就微笑说, 不就蒋介石蒋经国嘛. 那一刹那同学居然敢笑了, 我才发现时代不同了, 懂吗? 

好, 我要讲的是这些大人物, 你非常讶异的是在那只小小鸟的下一段. 庄子说这些位阶已经在天下人, 当然在{\color{blue} 而征一国}, 在一国之上的一个人, 他说他看他自己其实就是像这只小小鸟一样, 他以为他自己已经到达鸟类的极致, 所以我每次看到这段, 我就要提醒我的学生, 我的学生有时候上完 <<庄子>> 就没搞清楚. 他说: 老师, 聪明才智能胜任一个国家的官职, 言谈举止能庇荫一个乡里, 品德操守适合当一个国君, 能力为全国征信. 他明明是听我这样讲的, 这就是小小鸟, 你看, 不是吗? 小鸟的脸跟君王一定完全一样, 你们没看出来, 他就是小小鸟嘛. 可是呢, 没上完 <<庄子>>, 他们就会遗忘这一段, 然后就跟我说: 老师, 我不要, 我可不可以不要当宋荣子,或列子,或是我可不可以就当小小鸟就好, 我觉得我这样子就比较有安全感. 你是说, 明天我就看到你就要讲江院长好, 是吗? 

搞清楚庄子说的小小鸟是谁, 于是你非常困惑, 天地君亲师耶, 君子有三畏耶, 大人是其中一畏耶, 为什么在儒家经典里面, 位阶这么高的这些人, 在庄子的譬喻里面, 是他飞禽系列里面最小只的. 那谁够格当大只的? 

\subsection{\kaishu 犹有所待--- 你的梦想, 是否还待天时地利或他人的协助才能完成?}

我们往下看, 他说啊.

\begin{center}
	{\color{magenta} 而宋荣子犹然笑之. 且举世而誉之而不加劝, 举世而非之而不加沮, 定乎内外之分, 辩乎荣辱之境, 斯已矣.}
\end{center}

\vspace{-0.5cm}

{\color{blue} 而宋荣子犹然笑之}, 这犹然这两个字就是笑的样子, 所以宋荣子就用笑得样子, 笑了. 


{\color{blue} 且举世而誉之而不加劝}, 那为什么宋荣子笑, 其实各位同学你要注意, 金圣叹说 <<庄子>> 是天下第一才子书, 我觉得那是挺有意思的, 其实有些东西你要埋那个伏笔, 埋得越不着痕迹, 其实是非常困难的. 那我要讲的是, 包括宋荣子出场的表情, 他绝对不会说, 智人犹然笑之. 不是, 你们再去读读 <<老子>>. 

<<老子>> 说什么, 上士闻道, 最高阶的知识分子他听到道, 喜乐而行之. 非常勤奋, 奋勉就开始实践了, 第一等的知识分子, 听到什么? 朝闻道, 夕死可矣还不够, 在 <<庄>> 学认为: 朝闻道, 朝就要开始实践, 这是最高境界. 

中士闻之, 然后中等程度的读书人会怎么样? 若存若亡, 好像听过, 每次问同学, 老师上到哪啊? 有的同学说, 老师这段还没上, 然后一两个同学跟我讲上了. 有的觉得还没上, 有的人觉得上了. 若存若亡, 好像有听过, 好像没听过. 忘了差不多了, 又过了一个礼拜了, 可是这还不是最低阶的.

最低阶, <<老子>> 怎么说, 下士闻道, 大笑之. 哈哈哈, 好笑这是什么道, 不笑不足以为道. 那你看, 你还会讥笑, 这也不是很高的境界.

所以当然后面还有, 我们先讲宋荣子, 宋荣子是什么样的境界呢? 我在去年好好研究一下宋荣子, 觉得庄子真是解人, 如果有一天, 别人问我, 你死后的传记由谁来帮你写? 如果里面有庄子, 我绝对挑庄子. 庄子净拣好的说, 他很会看一个人优点, 所以他的世界很美好. 我看他写的宋荣子, 我当年第一次读 <<庄子>> 的时候, 觉得好欣赏这个人啊. 

怎么样? {\color{blue} 举世而誉之而不加劝}, 全世界的人都赞美他, 他也不会更加劝勉, 这很像一个印度拿诺贝尔奖的矿物学家, 他在他获奖的那一天贺客盈门, 整个新闻媒体都到了, 凑在外面. 他穿着实验服, 缓缓走出来, 只跟大家打了声招呼, 他说得奖之前的我跟今天, 其实完全一样, 得奖之前, 你们没有来找我, 得奖之后, 你们今天这么多人来, 可是我还是跟昨天一样, 我要做实验去了. 他就向后转, 那么多的赞美, 没有让他觉得很开心,觉得很虚荣,很炫耀.

{\color{blue} 举世而非之而不加沮}, 全世界的人都批判他, 他也不在意, 他知道在做一些对的事, 这是非常困难的. 但如果你把宋荣子考察了一番, 我去年在课堂讲了, 我举得我要讲, 我讲了我学生宋荣子梦碎, 这样太没有美感了. 其实宋荣子一生, 怎么说呢, 他就是设计了一些不太理想的理论, 然后到处游说, 然后带着他的学生们. 然后君王都不接受, 读书人也不接受, 他还是坚持到底, 然后把嘴巴讲烂了, 然后他跟他的学生都非常的穷困, 连三餐都不继, 大概就是这样一个学派, 你看了不甚了了, 他根本没有{\color{blue} 举世而誉之而不加劝}, 他只是{\color{blue} 举世而非之而不加沮}. 那庄子因为对仗嘛, 把他写一个 {\color{blue} 举世而誉之而不加劝, 举世而非之而不加沮}, 这样听起来不得了.

{\color{blue} 定乎内外之分}, 什么是重要的, 什么是不重要的, 那宋荣子的理论很简单, 什么是重要? 就是让每个人都吃饱, 那除了这个以外, 什么事情都不要太计较嘛. 他大概他整个理论就这样, 不要讲什么公平正义, 让大家有得吃, 你吃少点没关系, 反正吃饱了嘛, 大概就是这样子. 所以他得理论你看了以后是: 咦, 怎么就这样子而已哦, 宋钘(xing)尹文的宋钘.

{\color{blue} 辩乎荣辱之境}, 他分别荣辱之境的理论, 就是什么叫做 {\color{blue} 辩乎荣辱之境} 呢? 其实那个时候是战乱的时代, 只要被攻打的国家, 都觉得受辱了, 那他就说, 其实你只要不要觉得自己受辱, 你就没有受辱, 这是什么东西呀, 懂吧? 因为这理论别人真的是哀哀无告,受苦, 你跟他说你只要感受你没有受辱就好, 所以他的理论你看实在欠缺周详, 所以他真的就没办法像法家, 像墨家乃至于像道家,儒家他们的思想, 被这么多后世的注疏家或学者研究, 其实真的有他的道理.

好, 就这样子. 可是他也了不起, 为什么? 因为他的价值跟一般人不一样, 坦白讲, 你不要管他的理论周不周全, 他为了自己坚持的理想, 率领这批学生然后不追求人间的温饱, 不图个人的富贵, 其实已经很困难了. 你看现在有一个人, 很多人跟着的, 那个前面没有赚头的, 这绝对是少数, 你懂吧, 这就是宋荣子.

\begin{center}
	{\color{magenta} 彼其于世, 未数数然也. 虽然, 犹有未树也.}
\end{center}

\vspace{-0.5cm}

可是庄子接下来要说的是, {\color{blue} 彼其于世未数数然也}, 可是一般人, 赚了钱, 就想图个人生活富贵, 图一家生活富贵, 最好让你的小孩一生出来, 银行里的存款就很多, 可是他不然. 他不汲汲于世俗之人这些追求, 虽然他已经比世俗之人在庄子的评价更高了. 

可是呢, 庄子说: {\color{blue}犹有未树也}, 我们说他踽(ju)踽独行到处劝说的身影, 他还是有他还没有达到的更高的境界, 于是乎, 我们就要问, 那更高的境界是什么呢? 

%{\color{blue} 夫列子御风而行, 泠然善也, 旬有五日而后反. 彼于致福者, 未数数然也. 此虽免乎行, 犹有所待者也. 若夫乘天地之正, 而御六气之辩, 以游无穷者, 彼且恶乎待哉! 故曰: 至人无己, 神人无功, 圣人无名.}

\begin{center}
	{\color{magenta} 夫列子御风而行, 泠然善也,}
\end{center}

\vspace{-0.5cm}

{\color{blue}  夫列子御风而行, 泠然善也}, 更高的境界, 他讲了一个人, 他叫列子, 列子他怎么样呢, 他乘风而行. 各位同学, 你想想风是什么? 风是机会也可能是考验, 因为机会你如果抓不住, 它可能落空了, 你就很难过, 所以御风而行, 列子他不计较是不是一定要往前走, 他没有这样的固执, 但是如果机会来了, 他就乘风而行. {\color{blue} 泠然善也}, 他的姿态是这么地轻灵美妙.

\begin{center}
	{\color{magenta} 旬有五日而后反. 彼于致福者, 未数数然也.}
\end{center}

\vspace{-0.5cm}

{\color{blue} 旬有五日而后反}, 十五天后, 这个风向转了向, 他又跟着吹了回来, {\color{blue} 彼于致福者, 未数数然也}.

各位同学, 假如刚刚的宋荣子, 他心目中有一个跟别人不一样的福气, 什么叫做人间的福, 幸福是什么? 我们在这块土地成长, 可能不部分的家庭, 家长都会跟你说, 不要管外面的事情, 赶快读书就好, 不用做家事, 赶快读书就好. 好多我台大的学生, 爸爸妈妈对他们最大的希望就是, 念完学士念硕士,念完硕士念博士, 爸爸妈妈就跟孩子说: 你拿到博士学位, 爸爸妈妈就死而无憾了. 他们听到死就觉得非常沉重, 所以一步一步地往上念.

可是如果你不是出生在这样的家庭, 你多看看风声,雨声,读书声, 声声入耳, 家事,国事,天下事,事事关心. 你会看到一群人, 我每次经过台北市下雨的日子, 我很怕经过一条路, 因为那一条路, 会有很多人, 在路边护树, 他们不喜欢台北市政府乱砍树, 那他们知道只要风雨大了, 一般人不注意, 就是砍树的大好时机, 所以他们就爬到树上, 抱住那个树, 那我是一个很爱树的人, 所以我很怕看到这样的声音, 我觉得好残忍, 那这些人不觉得下雨天在外面淋雨很痛苦吗? 这是他们认为的幸福, 他们觉得这个地球就是要永续经营, 拥有很多的绿树才幸福, 那这样的幸福显然跟要开 Party 花比较多的钱, 这种比较世俗价值的幸福是不一样的, 想要吃好穿好享受好, 又是不一样的.

可是我们刚刚讲的那个护树, 这譬喻还不错, 因为他们是真的有点可怜, 日以继夜. 那如果管理众人之事的人越轻忽他们, 处理得越粗糙,越残暴, 他们身心受的伤就越重, 所以这时候就显得列子聪明了. 列子看机会来了, 他才去护树, 懂吗? 

他觉得他今天做这件事情会有用, 而且他自己非常安全, 他才会去做. 他不强求, 在过去儒家思想里面, 我们觉得不强求太没有美感. 在儒家思想, 我们就要怎么样? 不信春风唤不回, 知其不可而为之. 生亦我所欲也, 义亦我所欲也, 二者不可得兼, 舍生而取义. 就是一定要壮烈成仁, 才是完美的结局, 可是庄子的疑问是这样吗? 他觉得保身全生, 爱养自己的生命, 爱养这个社群的生命, 管理众人之事者, 爱养更多人的生命, 这不是最基本的吗? 

{\color{blue} 彼于致福者, 未数数然也}. 这里即使对于他跟别人不同定义的人间的幸福, 他也不汲汲追求, 所以他就比宋荣子不辛苦, 懂吧.

\begin{center}
	{\color{magenta} 此虽免乎行, 犹有所待者也.}
\end{center}

\vspace{-0.5cm}

{\color{blue}此虽免乎行, 犹有所待者也}. 你觉得他好像一个圣之时者也, 风来了, 我乘风而去, 然后风回来了, 我跟着回来. 没有, 我家里蹲就算了, 好像是个圣之时者也, 对不对? 可是为什么, 虽然他做到这样, 我也在上学期好好研究列子一番, 然后就发现, 列子这个人, 他确实有些特殊的技能. 

可是呢, 讲到他非常会射箭, 在平地是个神射手. 可是有一天呢, 他的老师, 一个得道高人就要他到山上射箭, 列子有惧高症. 他一到山上, 哎呀, 他的这个脚后跟有点跑到快要到悬崖外了, 然后三分之二的脚掌在地面上, 他整个人发都冒汗, 整个就趴了下来了. 这就是列子, 真正的列子的故事.

上学期为了好好讲列子, 就深入每一个故事, 然后就发现了什么, 这就是列子还没有办法达到最高境界的地方. 因为庄子讲的最高境界, 就是在心上做工夫, 所以, {\color{blue}此虽免乎行, 犹有所待者也}, 因为你今天还是要等风来嘛, 你才能完遂你要达到的目标, 你还是有个目标在外面, 只是你不强求而已. 好像政治清明了你做, 那你觉得政治太混乱了那就算了, 那你的目标还是要在外面做些什么呀, 不是吗? 所以, 更高一阶到底是什么?

\subsection{\kaishu 乘风御辩---有一种能力,无论顺境,逆境都能逍遥驾御}

{\color{blue} 若夫乘天地之正, 而御六气之辩, 以游无穷者}, 各位同学, 太重要了, 在这边我们得要字斟句酌地讲.

\begin{center}
	{\color{magenta} 若夫乘天地之正, 而御六气之辩, 以游无穷者, 彼且恶乎待哉!}
\end{center}

\vspace{-0.5cm}

其实, {\color{blue} 乘天地之正, 而御六气之辩}, 这是我曾经的一篇升等论文, 就是写这两句. 可是这两句我不是从 <<庄>> 学里面研究, 因为你们知道, 我另一个兴趣嗜好, 另一个专业是医学经典. 在我阅读医学经典的经验当中, 发现里面会谈五运六气.天地之间的运气, 或者讲节气, 有它正常的时候,有不正常的时候, 这个你可以把它讲得非常玄妙, 当然觉得不可解, 可是也可以讲得非常简单.

简单地讲, {\color{blue} 天地之正}, 就是天气正常的时候, 各位同学, 现在不是秋天吗? 所以如果正常的节候, 一定要有点凉爽对不对? 如果热到大家穿没袖的衣服来上学, 那就表示是{\color{blue} 六气之辩}
了, 对不对? 那当然我们都读过 <<窦娥冤>>, 六月雪嘛, 或者我们读过 <上邪>, 冬雷震震, 夏雨雪, 那这都是 {\color{blue} 六气之辩}嘛.

各位同学, 如果就一个心灵的工夫来讲, 你们觉得 {\color{blue} 天地之正} 容易乘御还是{\color{blue}六气之辩} 啊? 我们今天来上学的时候有没有下雨? 没有, 那这几天有下雨, 对不对? 所以我们很自然, 我们都带伞, 对不对? 那如果, 我们带伞了, 待会下雨, 我们觉得太稀罕, 我们就不会受到什么伤害嘛. 可是如果我们待会下课时候下雪了呢? 你不要跟我讲好开心, 搞不好你就觉得我心脏不舒服, 这边不舒服, 我腰酸背痛我不舒服. 

我要讲的是说, 一个异常状态就是我们比较难面对的, 我要讲的是, 如果今天你的情人就在固定打电话的时候打电话给你, 就在固定该约会的时候约会, 那你是不是很容易就心情很好, 可是他如果说: 记得今天晚上九点, 台大傅钟下集合. 然后你在傅钟下怎么等,怎么等, 他都没有来就让你想到新生宿营, 听到那个台大傅钟的鬼故事, 你整个人发毛, 觉得非常不舒服, 那就{\color{blue} 六气之辩}了嘛, 那你今天当然比较不容易开心嘛, 所以, 在天气, 是 {\color{blue} 天地之正}, 我们容易乘御, {\color{blue} 六气之辩}艰难.

情感的对待人与人之间也一样, 如果老师今天一走进来就说: 对不起, 老师今天不舒服, 一个人发一张纸, 请默写 <逍遥游> 第一段. 你一定非常愤怒嘛, 因为这是太严重了, 没有教, 不教而杀, 不教背而背, 这太严重了哦. 这时候你还能气定神闲然后在纸上写: 我知道你在测试我的乘御功力, 马上 $ A^{+} $ 就发回去了, 懂吗? 因为这是一个 <<庄子>> 的考试, 不是考这个嘛, 所以庄子最后讲的最高境界到底是什么? 

你说, 大鹏鸟那么奋发地飞到九万里外, 或者这个大鸟已经飞到百里外,千里外, 已经{\color{blue} 三月聚粮}了, 也不容易耶. 你怎么说那个最小只得才是这个我们伟大得马英九总统或江宜桦院长. 

怎么会这样呢? 那更高的人呢? 

他提出来的第一个宋荣子, 哇塞, 有这样一个跟世俗不一样的标准, 然后还能这么奋力地做, 你觉得不容易, 然后又觉得有点太执着了, 因为你有可能自己的身心会受伤. 然后就讲一个列子, 飞吹来了, 我乘风而去, 我做一番事业, 风吹回来了, 好, 我收摊, 我退休这样也不行吗? 那庄子说, 你还是在外面嘛, 你的目标还是在外面嘛.

最后揭牌了, 最高境界是什么? 原来最高境界就是任何的天气都不会感冒, 可是那个不会感冒的不只是你的身体, 还有你的心灵, 你都很开心, 你都能够好像冲浪一样, 都能乘御这些风浪, {\color{blue} 以游无穷}, 各位同学, 无穷两个字非常困难, 无穷表示任何时间,任何地方, 你的心都能乘御一切, {\color{blue} 彼且恶乎待哉!} 假若这就是你的人生目标, 那你还需要等待什么吗? 如果今天你把所有的逆境都当成考验, 把所有的顺境当成值得感谢的幸福, {\color{blue} 生物之以息相吹}, 那你其实不必盼着一定要有什么好事降临你的头上, 懂吧? 因为你都会品味出不同的滋味, 你觉得它都可以强化你的生命. 所以, 庄子说: {\color{blue} 彼且恶乎待哉!} 这样的人还需要等什么呢? 

那最后呢, 他就用一个结尾来收尾.

\begin{center}
	{\color{magenta} 故曰: 至人无己, 神人无功, 圣人无名.}
\end{center}

\vspace{-0.5cm}

{\color{blue} 至人无己, 神人无功, 圣人无名}, 达到人的最高境界的, 不再执着于自我了.

什么叫不再执着于自我? 如果你觉得读书读得非常好, 是一件很棒,很有意义的事, 而拿书卷奖反映这件事的话, 那虽然你这这学期没有完成, 班上另一位同学完成了, 你也替他高兴, 为什么那个人一定要是你? 

如果你觉得这个女孩非常好, 你跟他都追她了, 她最后接受了他的感情, 而没有接受你的, 你觉得这个我这么爱的女孩, 她能在心智自由而不是拿刀被强迫的情况下做出她的选择, 那我应该为她做出选择而感到开心, 那我又怎么知道谁是我的真命天子,天女呢? 所以你祝福她, 你也很开心, 所以你没什么一定要是你得到, 你才觉得过瘾你才觉得应该要.

{\color{blue} 神人无功}, 你觉得最了不起的人也不一定要是你打开所有的周刊,所有的媒体, 整天聚光灯就跟着他跑. 我们常会在台湾发现一些人, 在媒体披露之前你从来不知道有这个人, 什么一个卖菜老太太, 她怎么样捐钱给一些图书馆,一些单位, 甚至于她就是经营一个小小的生意, 造福一些人, 她没有什么显赫的功绩, 但是她达到很高的境界, 也许你怎么样跟她交谈, 她都心怀感激, 她都不会动怒.

{\color{blue} 圣人无名}, 我们相信有一种人, 他在这个世界, 是默默无名之士, 但是他给这个世界的贡献, 也许比领非常高的薪水占着非常重要位置的人, 还要巨大, 只是我们不知道他的名字, 这样的人.

那庄子讲的最高境界, 其实就是前面讲的 {\color{blue} 乘天地之正, 而御六气之辩}, 而这样一个境界的具象化, 会在下面三段有具体的名字来论述它.

各位同学, 当我们一起读完 <<逍遥游>> 的 <北冥有鱼> 单元, 让我们一起来进行以下的问题及思考: 

{\Large {\color{purple} 问题与思考}}
\vspace{-0.5cm}

\begin{enumerate}
	\item 鸟与树. 每天醒来, 如果人生舞台上的角色可以随意扮演, 各位同学, 你今天要化身不断外逐的飞鸟, 还是棵扎根日深的树?
	
	\item 各位同学, 请在一张纸上, 试着写下这几年来你最重视,付出最多心思的三件事. 然后想想这三件事的成败是否全然操之在己, 还是需待外在人,事,物的配合才能成就? 
\end{enumerate}

%%%%%%%%%%%%%%%%%%%%%%%%%%%%%%%%%%%%%%%%%%%%%%%%%%%%%%%%%%%%%%%%%%%%%%%%%%%%%%%%%%%%%%%%%%%%
\beginrefs
\bibentry{1}{ 蔡璧名}, 
``正是时候读庄子: 庄子的姿势,意识与感情'',
{\it 天下杂志出版 },
2015.
\endrefs

\begin{flushright}
	\tiny \kaishu \today  \  北京.
\end{flushright}

\end{document}

              
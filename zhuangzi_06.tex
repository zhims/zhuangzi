%%%%%%%%%%%%%%%%%%%%%%%%%%%%%%%%%%%%%%%%%%%%%%%%%%%%%%%%%%%%%%%%%%%%%%%%%%%%%%%%%%%%%%%%%%

%%%%%%%%%%%%%%%%%%%%%%%%%%%%%%%%%%% Author:Yao Zhang  %%%%%%%%%%%%%%%%%%%%%%%%%%%%%%%%%%%%
%%%%%%%%%%%%%%%%%%%%%%%%%%%%% Email: jaafar_zhang@163.com  %%%%%%%%%%%%%%%%%%%%%%%%%%%%%%%

%%%%%%%%%%%%%%%%%%%%%%%%%%%%%%%%%%%%%%%%%%%%%%%%%%%%%%%%%%%%%%%%%%%%%%%%%%%%%%%%%%%%%%%%%%

\documentclass[11pt]{article}
\usepackage[utf8]{inputenc} 
\usepackage[table]{xcolor}
\usepackage[most]{tcolorbox}
\usepackage[left=2.50cm, right=1.50cm, top=2.0cm, bottom=2.50cm]{geometry}
\usepackage{xcolor,url,cite}
\usepackage{amsmath,amsthm,amsfonts,amssymb,amscd,multirow,booktabs,fullpage,calc,multicol}
\usepackage{lastpage,enumitem,fancyhdr,mathrsfs,wrapfig,setspace,cancel,amsmath,empheq,framed}
\usepackage[retainorgcmds]{IEEEtrantools}
\usepackage{algorithm}
\usepackage{algorithmic}
\newlength{\tabcont}
\setlength{\parindent}{0.0in}
\setlength{\parskip}{0.05in}
\colorlet{shadecolor}{orange!15}
\parindent 0in
\parskip 12pt
\geometry{margin=1in, headsep=0.25in}
\usepackage{subfig,graphicx,framed}
\graphicspath{ {img1/} }
\usepackage{ctex}
\usepackage{multirow, tabularx}
%%%%%%%%%%%%%%%%%%%%%%%%%%%%%%%%%%%%%%%%%%%%%%%%%%%%%%%%%%%%%%%%%%%%%%%%%%%%%%%%%%%%%%%%%%

\newtheorem{theorem}{Theorem}[section]
\newtheorem{definition}{Definition}[section]
\newtheorem{exercise}{Exercise}[section]
\newtheorem{note}{Note}[section]
\newtheorem{notation}{Notation}
\newtheorem{lemma}{Lemma}[subsection]
\newtheorem{proposition}{Proposition}[section]
\newtheorem{example}{Example}[section]
\newtheorem{homework}{Homework}[section]
\newtheorem{summary}{Summary}[section]
\newtheorem{corollary}{Corollary}[section]
\newtheorem*{remark}{Remark}
\makeatletter
\@addtoreset{equation}{section}
\makeatother
\renewcommand{\theequation}{\arabic{section}.\arabic{equation}}
\usepackage[colorlinks,linkcolor=blue, anchorcolor=green,citecolor=red,urlcolor=blue]{hyperref}


%%%%%%%%%%%%%%%%%%%%%%%%%%%%%%%%%%%%%%%%%%%%%%%%%%%%%%%%%%%%%%%%%%%%%%%%%%%%%%%%%%%%%%%%%%

\def\beginrefs{\begin{list}%
		{[\arabic{equation}]}{\usecounter{equation}
			\setlength{\leftmargin}{0.8truecm}\setlength{\labelsep}{0.4truecm}%
			\setlength{\labelwidth}{1.6truecm}}}
	\def\endrefs{\end{list}}
\def\bibentry#1{\item[\hbox{[#1]}]}

\def\UrlBreaks{\do\A\do\B\do\C\do\D\do\E\do\F\do\G\do\H\do\I\do\J
	\do\K\do\L\do\M\do\N\do\O\do\P\do\Q\do\R\do\S\do\T\do\U\do\V
	\do\W\do\X\do\Y\do\Z\do\[\do\\\do\]\do\^\do\_\do\`\do\a\do\b
	\do\c\do\d\do\e\do\f\do\g\do\h\do\i\do\j\do\k\do\l\do\m\do\n
	\do\o\do\p\do\q\do\r\do\s\do\t\do\u\do\v\do\w\do\x\do\y\do\z
	\do\.\do\@\do\\\do\/\do\!\do\_\do\|\do\;\do\>\do\]\do\)\do\,
	\do\?\do\'\do+\do\=\do\#}

\renewcommand{\algorithmicrequire}{\textbf{Input:}}  % Use Input in the format of Algorithm
\renewcommand{\algorithmicensure}{\textbf{Output:}}
\renewcommand{\figurename}{\kaishu 图}
%%%%%%%%%%%%%%%%%%%%%%%%%%%%%%%%%%%%%%%%%%%%%%%%%%%%%%%%%%%%%%%%%%%%%%%%%%%%%%%%%%%%%%%%%%
\begin{document}
\kaishu 
	
\setcounter{section}{5}
\title{title}
\thispagestyle{empty}

\begin{center}
	{\Large \kaishu 齐物论 \ \ 莫若以明 \ 上}
	
	{\large \kaishu  告别负面情绪的方法}
	
	{\vspace{-0.2cm}}
	
	{\kaishu 蔡璧名}
\end{center}

%\section{}
% \subsection{\kaishu 告别负面情绪的方法}

{\Large {\color{purple} 前言}}

你可曾有过:正要与至亲的家人,要好的朋友,至爱的情人, 由面对而对峙,由对峙而对敌的一瞬. 内心忽然喊停,进而转怒为喜,转危为安的经验?

\begin{center}
	
	因为能够理解体谅,因为瞬间同情共感.
	
	站在几楼的高度俯看这座城市,才能无所偏蔽,完整照见? 
	
	登上哪个山头看人间,才能跳脱立场,公平照看?
\end{center}

心随境转,于是,记挂,悬念,牵萦,纠葛, 原本争鸣于心的杂音, 远了. 只像城市街头,路边夏树的一阵婉转. 

此时倾耳静听,啁啾(zhou jiu)再悦耳,节拍再纷杂,不再因此乱气,动心. 昨日赴之一怒的,而今可以赴之一笑.

\begin{center}
	
	无妨了.
	
	可憎,也可怜. 可恼, 也可爱.
	
	敌人不复对敌,是手足.
	
	太爱太可悲,执手也可放.
	
	眷恋不复执迷,天涯亦若咫尺.
\end{center}

当衡量,评比高下的尺,从功成名就,财源广进(或居仁由义,取义舍生?) 转换成心宽身适,健步如飞. 则人间世的标杆,历史的典范,也将随之翻转,易位.

\begin{center}
	巨大的,渺小了. 非要不可的,淡然了.
	
	你有工作吗?  有啊!
\end{center}

你做什么? 我做建筑师, 给人设计安全舒适的房子. 那你呢? 我做面摊啦, 下面给深夜未归,饿肚子的人.

\begin{center}
	我的肠胃因你温暖. 
	
	你的屋舍因我而得以避雨遮风.  
	
	我们的生活,因彼此,因芸芸众生而完整. 
	
	一己生命才得以在其中, 常养真气,静定心神. 
	
	鸟躁枝头,一树仍多异议...
	
	夜鹭群聚, 众啾一词: 非征讨那些白昼乱飞的异类不行...
\end{center}

{\Large {\color{purple} 自师成心---你没有成见吗?}}

第三段开始喔, 好, 同学念一下. <<齐物论>> 第三段, 第三大段莫若以明. 

{\color{blue} 夫随其成心而师之, 谁独且无师乎? 悉必知代,而心自取者有之,愚者与有焉. 未成乎心而有是非, 是今日适越而昔至也. 是以无有为有. 无有为有, 虽有神禹且不能知, 吾独且奈何哉! 夫言非吹也, 言者有尝有言邪? 其以为异于彀(gou)音, 亦有辩乎? 其无辩乎? 道恶乎隐而有真伪? 言恶乎隐而有是非? 道恶乎往而不存? 言恶乎存而不可? 道隐于小成, 言隐于荣华. 故有儒墨之是非, 以是其所非而非所是. 欲是其所非而非其所是, 则莫若以明.}

好, 我们来看, <<齐物论>> 很长啊. 

我们在第一段提出一个人叫什么? 南郭子綦(qi), 然后提出一个理想得境界, 那这个境界到底要怎样达成? 我们要去了解为什么这个境界达成不了. 为什么我们不能和南郭子綦一样整天心情很好, 没有任何负面情绪. 为什么我们得身体不能跟他一样, 觉得这么轻松, 没有任何得不适. 

那庄子在第二段为我们解析答案是什么, 是莫知所萌. 我不知道为什么我今天就特别背,很多得灾难就横在我得眼前, 所以觉得非常地苦恼, 你会觉得我被气得要死, 我被伤了,我被骗了. 可是<<庄子>> 莫知所萌这段告诉我们什么? 到底这样得负面情绪是谁导致的? 让我们抓出凶手, 一个负面的事件要抓出凶手并不容易, 尤其如果你永远觉得凶手当中没有你自己的话, 可能你就只能随人宰割了. 

然后第三大段我们讲莫若以明, 其实庄子要告诉我们一个方法, 这个方法可能可以让你脱离那种你觉得你被宰割,被伤害,被气得要死得状态, 所以我们现在进入第三大段.

\newpage 

{\begin{center}
		{\color{green} 夫随其成心而师之, 谁独且无师乎?}
\end{center}}

\vspace{-0.5cm}

首先庄子为我们解析, 他说{\color{blue} 夫随其成心而师之,} 什么叫成心? 就是成见. 	你说老师, 那你从开始教我们, 就是希望我们把 <<庄子>>学好, 这算不算是一种成见呢? 所以我们要了解什么叫成见.成见是一家之偏见, 就是它没有一个放诸四海而皆准的价值.

你说,老师, 这个多元价值的时代, 哪来的放诸四海而皆准的价值? 当然有啊, 不信你去参加2015的跨年, 大家会一起喊 Happy New Year. 有一个人喊祝你倒霉, 揍他一巴掌, 对不对? 可见 Happy New Year 我们希望每个人都很快乐很幸福, 就是一个放诸四海皆准的价值嘛. 

所以呢, 他这里讲的成见不是说你要有一个定见, 有一个想法就是成见, 不是, 懂吧. 就是你有偏见, 更可怕的是, 我不知道同学你们生命中是否有经历过这样的岁月? 就是你回到家, 爸爸妈妈说什么话, 你就会说, 可是老师说什么话, 你这辈子说过一次这个话的举手, 哎, 好多人, 我们都尊敬老师啊.

可是什么叫成见呢? 就是那个老师不是别人, 就是你心里的成见.

然后庄子问了, 这世上谁能独独没有成见? 于是我们想, 有一种人可能没有喔, 就在上一段才出现. 

{\begin{center}
		{\color{green} 奚必知代而心自取者有之? 愚者与有焉. 未成乎心而有是非, 是今日适越而昔至也. 是以无有为有. }
\end{center}}

\vspace{-0.5cm}

我们说, 喜怒哀乐忧叹变执姚佚启态. 乐出虚, 蒸成菌, 日夜相代乎前. 

这些负面的情绪有一个人能看透,这些情绪就在你心里跌荡搅扰, 那能看穿这一点的人, 是不是就没有成见? 庄子说, No No No. {\color{blue}奚必知代,} 何必一定要. 知代就是刚刚讲的知化, 知道人的情绪会这样起伏变化的人, 哪里是这样的人才有成见? 他知道他自己有这样的起伏变化, 不是只有他有成见, {\color{blue} 而心自取者有之.} 什么叫自取? 就是你自己, 这个自就有偏的意思, 取就是执. 你偏执一己之见的人, 也有成见. 

你说, 老师, 那人一种人他很傻, 很傻的人是不是就不会拜自己为师了? 庄子说, 错了, 即便是傻瓜, 也会拜他那傻瓜成见为师. 所以, 你会发现, 原来成见这东西人人都有.

真的是这样吗? 庄子更斩订截铁地说, {\color{blue} 未成乎心而有是非}, 如果说有人心中没有成见, 这句话就像什么? 就像太阳从西边出来, 就像地球变方了. 你说, 啥? 古人也这样说吗?

不不不, 古人不这样说, 他们说, 今日适越而昔至也. 那就好像今天我才出发到越国去, 你告诉我, 我昨天就到了, 那才怪. 那是古人用来描述我们讲太阳从西边出来. 简单讲, 这是不可能的事, 所以人人都有成见.

\newpage 

{\begin{center}
		{\color{green} 无有为有, 虽有神禹且不能知, 吾独且奈何哉!}
\end{center}}

\vspace{-0.5cm}

如果今天你要说, 把没有当作有, 是我诬赖别人, 这是不可能的事. {\color{blue} 无有为有,} 要把没有说成有, 这件事有多困难? {\color{blue} 虽有神禹且不能知},  就算神圣英明的大禹, 这个夏大禹, 我们知道中国古代的圣王嘛, 大禹治水, 三过家门而不入, 我们都知道的, 就算夏大禹这么英明, 他也不会知道没有怎么变成有. 那我, {\color{blue} 吾独且奈何哉}, 更何况是这样的平凡人呢? 我怎么有办法去了解没有变成有?

那你这里觉得很奇怪, 你说大家都没有成见就好了, 为什么要转这句话? 其实转这句话我就嗅到一种好像 <<老子>> 快要开始挨骂的味道, 因为天下万物生于有, 有生于无, 道可道, 非常道, 名可名, 非常名, 无名天地之始, 有名万物之母, 这老子的学说嘛, 对不对. 你要把没有说成有, 说天下万物生于有, 有生于无, 他说这个再聪明也听不出来, 我隐隐然嗅到这样的味道. 

\vspace{0.25cm}

{\Large \color{purple} 异于彀(gou)音---你今天说的话与枝头的鸟叫声, 来日回首, 意义可有不同? }

\vspace{-0.25cm}

接着庄子说啊.

\vspace{-0.25cm}

{\begin{center}
		{\color{green} 夫言非吹也, 言者有言, 其所言者特未定也. 果有言邪? 其未尝有言邪? 其以为异于彀者, 亦有辩乎? 其无辩乎? }
\end{center}}

\vspace{-0.5cm}

{\color{blue} 夫言非吹也}, 他说, 我们讲的话啊, 就是跟风不一样, 你说风有什么特色?    

各位同学, 风有它一定的方向. 如果你很注意你的家, 它是座哪里朝哪里, 你就知道太阳都从那边升起, 冬天都那边最冷. 你大概会知道, 风是有一定的型态, 今天是龙卷风, 今天是微风, 今天是暴风, 可是言语不一样, 言语哪里不一样? {\color{blue} 言者有言, 其所言者特未定也.} 言语非常地不确定, 它可以一下像东风,一下像北风,一下像南风, 它随时都可能改变, 各位同学, 没有统一的定论.

如果你到今天还没有感受到人的言语的不确定性. 我只能说, 你在感情上一定缺乏历练. 我听过很多, 总是要年过半百吧, 可能是我母亲的朋友, 然后我就会听他们讲, 这个男人要变天是不会挑礼拜几的, 所以大概只有二十岁的人会执迷于一句话, 会来问老师说: 老师, 他变心了, 你可以告诉我, 他为什么要变心吗? 

因为她眼泪就快要掉下来了, 所以我不能告诉她我心里的 OS(off screen, 画外音, 旁白之意). 你可不可以告诉我, 他为什么不会变心呢? 但我不能说. 她接着问:老师, 他今天如果甩了我, 他那时为何要追我?

我在想, 我要怎么样回答她, 我还是不能让她听到我的 OS, 因为他如果不追你, 他今天没办法甩了你, 这一切都是这么地自然.

可是你有成见地人, 你又觉得有一个人爱上我, 就应该像那个瞬间胶,及时胶就这样地黏住了,而且永远不能拆散. 那种胶很少, 可是你有这样的成见所以你很痛苦, 直复看尽洛城花, 始共春风容易别.

不知道我母亲当年我比较年轻的时候, 听那些年过五十的女人,总是这么悠悠地讲起人变心这件事, 而且她们不会觉得怎么样, 就好像给你说, 昨天下雨了, 这么地自然. 可能是有点人生历练吧, 了解了人的言语的不确定性, 然后终于明白, 当一个人跟你说, 我永远爱着你, 这句话表示什么? 这句话有可能是真的, 就是哪一秒钟他曾经想过要永远爱着你, 你这样理解是非常精准的, 懂吧? 这不愧为是一个修过 <<庄子>> 的人了.

这时候回头你会问, 我问我的学生, 每个人发一张白纸. 各位同学, 请写下你这辈子听过最感动的一句话. 没有, 没有人对我说感动的话, 觉得自己过得非常地不好. 好, 那这些想不出来的同学, 请你写下你这辈子对别人说过, 你觉得最动人的一句话, 想一想, 好像也没有. 

{\color{blue} 果有言邪? 其未尝有言邪?} 那如果真的是这样, {\color{blue} 其以为异于彀者, 亦有辩乎? 其无辩乎?} 你会以为你你说的话, 跟外面鸟叫声不一样吗? 	可是庄子居然说{\color{blue} 其果有言邪? 其未尝有言邪}, 你说的话你真的说过吗? 还是你说过就好像没说?

有一天你的话被别人回忆起来, 他也是写不出一句从你嘴巴吐露过的动人的话. 于是庄子就问了, 你以为你的声音跟小鸟鸣叫不同吗? 各位同学, 在这里用功的同学, 有复习<<逍遥游>>的同学是不是感觉到了? 蜩(tiao)与学鸠(jiu)笑之曰, 斥鶠(yan)笑之曰. 那些小小鸟, 自以为自己地位很崇高的人, 自认为自己最有学问的人, 也许后代的人已经尊奉你为先秦诸子了. 可是庄子要问 你们说过的话真的跟外面树上的一阵鸟鸣, 或是笼里的一支鸟鸣, 有所不同吗? {\color{blue} 亦有辩乎? 其无辩乎?}, 还是没有什么差别? 这时候我们就要从人的话, 这时人的话和鸟叫声, 转移到很多人士的一些话, 或者很多人自以为自己说过的话 是宇宙中最珍贵的道理.

{\begin{center}
		{\color{green} 道恶乎隐而有真伪? 言恶乎隐而有是非? 道恶乎往而不存? 言恶乎存而不可? 道隐于小成, 言隐于荣华. }
\end{center}}

\vspace{-0.5cm}

庄子要问了 如果这个宇宙有真正的, 最成功的道理, {\color{blue} 道恶乎隐}, 它为什么被隐蔽了? {\color{blue} 而有真伪}. 然后这世间上的言论, 居然出现真真假假, 而这么多不同的 <<庄子>> 诠释, 到底谁的是真的谁的是假的?

为什么真正的 <<庄子>> 要被隐藏在这么多说法完全的不同的 <<庄子>> 书里?

{\color{blue} 言恶乎隐而又是非 ?}这时候可能就会有一些作者开始吵架. 如果他们有机会见面的话或是面对读者的质询的话. 唉, 蔡璧名, 你怎么缘督以为经的诠释, 天之生是使独也的诠释跟目前市场上的诠释不一样啊? 怎么跑出一些身体的操作呢? 你是不是骗人呢? 哇, 于是我就跟大家吵了起来. 

为什么会有这些争执?如果真的跟 <<柯南>> 讲的, 真相只有一个, 那为什么它会被隐蔽了呢?


{\color{blue} 道恶乎往而不存,} 甚至于真正的道理哪儿去了呢? 为什么它应该是无所不在的啊, 真理.

为什么我们今天诠释 <<庄子>>? 有时候我们看注疏家写的, 好像点到了, 可是你要拿来操作又觉得不够, 于是你开始思考, 这些 <<庄子>> 的注疏家就是历史上真的最理解 <<庄子>> 的人吗?  还是可能是陶渊明,是李白,是苏东坡? 这些人在他们作品里面出现了读 <<庄子>> 的一些心得的札记, 会不会比这些注疏家更贴近庄子的原意? 这些都是我们要思考的喔.
	

{\color{blue} 言恶乎存而不可?} 为什么这个世界上有时候留下一些言论, 让我们觉得实在不太能认可?

我听到一段广播, 非常有意思喔. 它是说这个小朋友他在学校学游泳. 然后老师呢, 就给小朋友绑的带子颜色不一样, 你游得特别好, 你就是红带高手; 你游得不错, 你就是黄带中手; 你游得不好, 你就是黑带低手了.这个家长非常地愤怒,当他的孩子告诉他这样地一种分别, 他就想要去跟老师吵架.

这个家长代表的就是当代的价值, 你们懂吗? 就不可以让小朋友知道他比别人差, 你要告诉他, 孩子你是最强的. 你今天如果成绩不好, 你肯定是脑子好但不用功. 你今天如果家政做的不好, 那你肯定是会打篮球. 所以我的孩子, 你们都是全世界最好的孩子. 这是我们现在很喜欢的, 鼓励的教育嘛. 可是想不到当这个妈妈跟她的孩子说, 我要去学校骂你们老师, 他怎么可以这样伤害你的自尊?

她的孩子说, 啥? 妈, 这样为什么会伤害我的自尊呢? 你知道我一个礼拜就只去泳池练习一次, 可那些拿黄带,红带的, 他们有的天天去练习耶. 我觉得这样好公平喔, 没关系啊. 我只是不会游泳, 我会做家事嘛.

他妈妈忽然觉得自己很肤浅, 她不应该找老师抗议. 我要讲的是, 有时候这个世界上剩下的一些言论, 它叫我们要这样鼓励孩子, 当然和我的家庭教育不太一样, 可是我觉得我会这样子鼓励别人, 那个对象不是孩子, 是我养的狗, 我每一次去遛狗, 有一阵子它有点, 胃肠不好, 有点便秘, 所以它只要大小便顺利, 我就跟我母亲很不自觉在旁边拍手说, 好棒,好棒, Yuri, 好棒, Yuri, 尿尿了, 然后讲了这句话, 我跟我母亲两个就会相视大笑. 就为什么它今天尿尿或便便, 是一件这么值得鼓励的事, 可是如果你今天说, 孩子优缺点都不能跟他讲, 你也不能跟他说人可以更好, 你只要鼓励他, 你就这样, 就好了. 妈妈就爱你原来的样子

然后等你谈恋爱以后, 我们现在的时代氛围, 我们要告诉所有你爱的人说, 我爱你, 我爱你一切的优点跟一切的缺点, 请你千万不要改变你的缺点, 这句话是什么意思你们知道吗? 如果你今天爱上一个人,他的体脂肪非常地高. 但你和他说, 我爱你的全部, 那他只好完全不动, 维持一二三木头人不动的状态, 永远不要运动. 就让他肥死, 让他变得不健康, 这太不合理嘛, 懂吧. 

可是为什么有些言论似是而非, 好像就在这个时代广泛流传了? {\color{blue} 道隐于小成}, 庄子说, 因为真正的道理, 被有限的成就隐蔽了, 不是没有成就, 他说的不是完全没有道理, 有的人真的需要鼓励. 我是搞传统医学, 我非常了解, 那就好像如果有一个人, 他真的阳虚, 阴气虚, 他很容易怕冷, 你和他说, 不要怕, 直接冲冷水澡, 越冲越勇, 那他就完蛋了, 你懂吧. 可是有的人真的已经很不怕冷,
你说你要再锻炼的话, 可以早上冲冷水. 你要对待不同的体能的人, 不同的心情的人讲不同的话, 而不是一味地批判, 也不是一味地赞美.

{\color{blue} 道隐于小成}, 所以有点道理的话, 使得真正的道理被隐蔽, 还有呢, {\color{blue} 言隐于荣华}, 什么叫荣华, 就是好听的话啊. 什么叫好听的话? 就你听了会注意的话啊, 那听了会注意的话, 通常就是离经叛道, 你才会注意嘛, 对不对? 如果今天跟你说, 各位同学, 你们都要好好做好份内的工作, 不归你的就别想要, 有谁会想重复这句话? 好无聊, 听了转头就睡了, 可是如果现在, 我们那个年代, 我年轻的时候, 电视上忽然出现一句话, 一句广告, 只要你喜欢, 有什么不可以? 大家听了好乐, 解放了, 自由了. 可是这句话的背后, 只要你喜欢, 卖毒油有什么不可以? 反正你变得很有钱. 只要你喜欢, 稍微害人一下, 只要不被抓到, 有什么不可以? 其实这句话有很多的流弊, 可是因为它很好听, 因为它很酷, 所以我们完全完全就被它蒙蔽了, {\color{blue} 言隐于荣华}, 因为好听嘛, 所以老子呢, 老子讲过一句话, 信言不美, 美言不信, 他说真话都不好听, 好听都不是真话, 可是我们常常会不想听真话, 老套. 你要说别人老套的时候, 老生常谈, 你要去想, 为什么这句话会被重复这么多次? 有没有可能它是真话?只是它被重复太多次, 你就不重视了啊, 接着接着, 接着庄子要接多可怕的话? 他要接, 就在他刚说完, 有些人的话其实只是一阵鸟鸣, 有些人的学说只是小成.

{\begin{center}
		{\color{green} 故有儒,墨之是非. 以是其所非, 而非其所是.}
\end{center}}

\vspace{-0.5cm}

他讲到这, 他接的不是任何一个老百姓的学说, 不是你我在坊间的议论, 是故有, 所以才会有儒家跟墨家的是非争辩. 儒家说爱有亲疏,爱有差等, 墨家说兼爱, 这两种语言我绝对都可以把它说得非常有道理.
你可以说, 你怎么可以爱自己的老婆跟爱隔壁邻居的老婆一样? 那不是一场灾难? 哇, 好有道理啊, 爱有亲疏. 那墨家怎么说有道理? 今天有两个人需要你帮助, 一个是你的亲人, 一个是路人, 可是去帮助他,去为善的你, 你去做这件善事, 都是用你短暂生命中的一个小时,两个小时, 因为你爱你的生命, 所以你都全力以赴. 哎, 好像也应该兼爱, 对不对? 所以嘛, 他们就吵起来了.当然我可能把墨家的兼爱解释到最理想的状态了啊.

然后呢, 儒家很重视音乐的教育, 墨家说, 大家都穷死了, 先有得吃吧. 先吃安全得油, 再来听音乐吧, 哎, 好像也合理, 对不对? 可是音乐又可以陶冶人性情, 哎, 好像真的都很有道理喔. 然后, 儒家厚葬, 子生三年, 然后免于父母之怀, 所以自己得至亲死了, 古人他守丧三年, 而且要厚葬, 多么有道理, 一个人如果连父母之恩都不懂, 那他如何感念天地间的恩情? 可是节葬呢? 假使我们接受了庄子或一些宗教的灵魂观, 我们觉得生命是永恒的. 今天父母亲的魂魄已经离开这个身体了, 那我们很尊敬地埋葬他, 我们地棺木真的要耗费这么多地木材吗? 还是金属? 哎, 你会发现其实一个理论两边说, 都有它说得通得地方, 更不用讲儒家是怎么样? 远鬼, 敬鬼神而远之. 墨家, 明鬼, 你说理性主义啊, 西方科学我们这时代不谈鬼神的, 可是你随便去看古今中外任何什么破案档案, 然后这是什么根据真实的事情描述, 好像多多少少都会出现鬼魄喔. 那是古今中外的这个警政单位一起联手欺骗我们吗? 还是灵魂真的存在? 

所以好像每一个角度我们去看一件事, 它都可说, 那如果都有可说得通的地方, 你在批判的对象都有说得通得地方, 又可以被批判, 会不会真的他们像庄子讲的一样, 他们得学说还没有大成. 什么叫大成, 无可挑剔, 还没有到无可挑剔的境界, 那怎么办呢? 那我们身为一个后生小子, 我们读这么多古人的智慧, 读这么多古今中外的经典, 我们是都不要读吗? 当然不是, 那我们要怎样去取舍呢?

{\begin{center}
		{\color{green} 欲是其所非而非其所是, 则莫若以明.}
\end{center}}

\vspace{-0.5cm}

庄子说呀, {\color{blue} 欲是其所非而非其所是}, 如果你今天自己觉得不对的事物, 你想要看到它对的那一面, 或者你觉得对的事物, 你想要看到它不对的那一面, 那要怎么办? 最好的办法, {\color{blue} 莫若以明}, 明这个字又出现了, 各位同学, 它在 <<庄子>>书会不断出现, 什么叫明, 太阳加月亮叫做明. 太阳跟月亮有什么不同? 它很高, 它很亮, 它可以照见更广阔, 幅员更完整的大地, 是不是这样呢? 它可以无所偏执地完整照见, 让世间万物呈现它更清晰的自己, 所以庄子要我们到太阳跟月亮的高度去, 这句话是什么意思? 他会继续说下去

{\Large {\color{purple} 得其环中---你可以尝试着去倾听,了解,体谅, 原本反对得一方?}}

同学念下一段

{\color{blue} 物无所彼, 物无非是. 自彼则不见, 自喻则知之. 故曰彼出于是, 是亦因彼, 彼是方生之说也. 虽然, 方生方死, 方死方生; 方可方不可, 方不可方可; 因是因非, 因非因是.是以圣人不由而照之于天, 亦因是也, 是亦彼也, 彼亦是也. 彼亦一是非, 此亦一是非. 果且有彼是乎哉? 果且无彼是乎哉? 彼是莫得其偶, 谓之道枢. 枢始得其环中, 以应无穷. 是亦一无穷, 非亦一无穷也. 故曰莫若以明.}

好, 先停在这儿, 因为我们又看到莫若以明了, 所以我们要赶快停下来, 看看是什么意思喔. 

{\begin{center}
		{\color{green} 物无非彼, 物无非是.}
\end{center}}

\vspace{-0.5cm}

他说啊, {\color{blue} 物无非彼}, 各位同学, 我们每一个人都是别人眼中的他嘛. {\color{blue} 物无非是}, 可是我们每一个人又是自己眼中的我们这一边, 自己这一方.

{\begin{center}
		{\color{green} 自彼则不见,自知(喻)则知之.}
\end{center}}

\vspace{-0.5cm}

{\color{blue} 自彼则不见}, 当我们要去看别人那一方的时候, 我们都看不到他的好处, 我们也无法非常深刻地了解, 因为他就在远远地那里, 你没有近距离地观照.

可是呢, 当你看自己地东西地时候, 大家都说自己文章好嘛. 我记得到中国旅行的时候, 还学了一个顺口溜, 他们说, 天下文章属三将, 说全中国文章最好的人集中在三江. 三江文章属我乡, 哎呀, 三江文章最好的刚好在我的故乡. 我乡文章属老表, 我们这个故乡文章最好的就是我表哥. 老表请我改文章, 那你知道最好的是谁了吧. 哎, 人面对作品, 总觉得自己最好, 就像妈妈看自己的孩子一样啊, 你容易明白自己的长处.

\begin{center}
	{\color{green} 故曰彼出于是, 是亦因彼, 彼是方生之说也. 虽然, 方生方死, 方死方生.}
\end{center}

\vspace{-0.5cm}

故曰, 所以我们知道了, {\color{blue} 彼出于是}, 我们今天会说他们, 是因为有我们才产生了他们, 而我们的概念也因为有他们的存在才得以成立嘛. 如果一个空间里面只有这群人, 那有什么他们好说? 别吓人. {\color{blue} 彼是方生之说也}, 什么叫方生, 这个方就是并, 同时并起, 相对而起. 他说, 他方跟我方, 这是同时产生的概念, 他会一起产生的, 他不是单一存在的. 怎么说呢? 

继续解释了{\color{blue} 虽然, 方生方死}. 他说同时出现会同时消失, 这什么意思? 你会知道这是什么意思的, 我昨晚上就知道了. 我昨晚和我学生去吃饭, 那个小餐厅的门口就贴了, 真食物, 真石材, 无添加. 唉, 食物怎有真假? 那就因为有假的嘛, 对不对, 所以这叫做方生, 就是因为有假食物出现, 才会有个商店要强调真食物嘛. 如果这天底下再也没有假油, 那谁还说我这油是真油啊, 对不对, 所以{\color{blue} 方生方死}. {\color{blue} 方死方生,} 可是有一天, 唉? 又有人做假东西了. 唉, 那什么是真的? 这个概念就又出来了, 同时消失又同时出现.

\begin{center}
	{\color{green} 方可方不可, 方不可方可.}
\end{center}

\vspace{-0.5cm}

还有呢, 很多事情我们觉得这是可以的, 很多事情我们说这是不可以的. 

我们举一个当代男人觉得最不公平的事情, 我们文学院的男生, 每次读到东坡传, 心里都不是滋味啊, 凭什么苏东坡家里有个王氏. 明月夜, 短松岗, <<江城子>> 这么恩爱的情感, 凭什么这人就是他老婆? 而且他家里还有小妾, 朝如行云, 暮如行雨, 朝朝暮暮, 其情如故, 那朝云跟暮雨还在东坡旁边走来走去. 唉, 东坡凭什么坐享齐人之福啊?

这所有在民国, 从清代要到民国, 你知道所有男人都要抗争, 都要疯了, 所以辜鸿铭说, 什么一夫一妻制啊? 从古到今, 一个茶壶就是配好多个茶杯, 天经地义啊. 可是后来时代变了, 你现在到这个商店去看看, 好多茶壶都只有一个茶杯了, 认命了懂吧. 那你们认命以后, 你们就不抗争了嘛, 所以以前你觉得可以的,对不起, 现在不行, 现在叫劈腿, 所以狗仔可以去追你, 你有一定的位置, 你做事情要更小心.

{\color{blue} 方不可方可}, 可是你才说它不行, 有觉得好像这九把刀劈得不是, 还没结婚, 这劈好像罪就没有已经结婚那么大, 对不对? 干嘛把他讲得这么不堪啊? 可是另一方又说, 虽无夫妻之名, 已行夫妻之实. 那就应该一样, 唉, 那好像又有道理, 所以这人世间的事情啊, 是是非非, 很多跟风俗习惯相关的是会因时代而改变, 因时代地域而改异的.


\begin{center}
	{\color{green} 因是因非, 因非因是.}
\end{center}

\vspace{-0.5cm}


{\color{blue} 因是因非}, 可是你活在台湾, 今天的台湾, 你就习惯了一夫一妻制是合理的, 你觉得你不认同的, 你就不认同了. 可是有一天, 以前我记得李登辉总统跟曾文惠女士, 他们到我们的邦交国非洲去考察. 一下飞机, 你们知道元首跟元首握手, 对不对. 这我们的曾文惠女士总统夫人可忙着了, 为什么? 因为这非洲国王的太太排成一排, 轮流跟她握手, 你懂吧. 可你到我们的总统夫人不会说, 这哪成啊? 你们猜拳, 其他人都休了. 这绝对不可以嘛,对不对. 那又是一套价值标准了, 那怎么办呢?

\newpage

\begin{center}
	{\color{green} 是以圣人不由, 而照之于天, 亦因是也.}
\end{center}

\vspace{-0.5cm}

{\color{blue} 是以圣人不由, 而照之于天,} 所以圣人哪, 他就不随这世间的是是非非起舞了, 而{\color{blue} 而照之于天}. 他跑到太阳跟月亮的高度, 太阳跟月亮的高度很了不起耶, 不只可以照见当代的是是非非. 秦时明月汉时关, 它都从秦代看到民国了, 它啥没见过啊? 所以这个时候你的心境会更具包容性, 对于是是非非, 越能参透, 越能同情了解, 越能了然.

\begin{center}
	{\color{green} 是亦彼也, 彼亦是也. 彼亦一是非, 此亦一是非.}
\end{center}

\vspace{-0.5cm}

要不, {\color{blue} 是亦彼也}, 今天是己方, 换个角度就变他方了. 小时候我参加辩论比赛, 所有的冠亚军赛绝对是同一个题目, 正反双方交换, 就是这三十分钟, 我是这个立场, 安乐死应该合法化, 下三十分钟, 我是这个立场, 安乐死绝对不能合法化. 那在辩论场我们就了解了一件事, 其实每一件事两边可能都有道理能讲. 

{\color{blue} 彼亦一是非,} 所以呢他们有一套他们的是非标准, 你看这个画片, 如果我们到非洲去生活, 可能我们每个人要学的就是怎么样保护自己的安全. 就像我的学生喜欢户外活动的, 他说, 老师, 这个假期要参加营队. 我说, 你们参加什么营队? 我们这次要去学习去辨识野生动物的粪便, 因为你认得野生动物的粪便才会知道是黑熊来了吗? 还是只是一只羊咩, 对不对? 那你觉得这个对生活是太重要了.

可是呢, 如果你到了所谓的华夏之邦, {\color{blue} 此亦一是非}, 在中国古代的礼节, 有访客到主人家要怎么样敲门呢? 然后你要走哪一边的阶梯呢? 到了中庭要怎么样跟主人打躬作揖? 这都有一定的, 如果你不懂, 那你这个野蛮人根本不能在这生活. 唉, 各位同学, 你们乍听一下觉得好像这样的冲突不会发生, 你只要把场景辐辏在同一个时空, 它就马上发生.

我记得我第一次1989年我到中国北京去玩, 然后呢, 在长安大街的街边, 就有些人特别, 他们的行为很特别. 他们就像坐在地上围了一圈好像开始吃餐盒, 然后那衣服打扮不太像城里人, 那因为这样的光景很特别. 我遇到我在北京, 住在北京的中国朋友, 我就问了, 我说, 哎呀, 我今天遇到什么样的人, 啊, 外地人, 只有外地来的会这样. 

那你想象刚刚这群人, 如果到了这个地方, 是不是就会有这样的光景呢? 可是他们在不同的时空环境里, 你都觉得恰到好处, 可是换一个地方, 你忽然又觉得他们不对了.

\begin{center}
	{\color{green} 果且有彼是乎哉? 果且无彼是乎哉? 彼世莫得其偶,谓之道枢. 枢始得其环中,以用无穷.}
\end{center}

\vspace{-0.5cm}

于是我们就要问了, {\color{blue} 果且有彼是乎哉}, 这世界上真的有你执着说是对的那个道理吗? {\color{blue} 果且无彼是乎哉}, 还是其实没有那个是非的定论呢?

{\color{blue} 彼世莫得其偶,谓之道枢.} 庄子说啊, 那我们就不要站在彼此对立得两端吧, 那我们要站在哪儿? 

偶是相对,是对立, 道枢,门枢, 这门, 下课同学都要打开来的, 这时, 门轴是不动的, 只有门片转动, 门轴跟门片上每一点的距离都是一样的, 所以如果你站在门的轴心, 你就能公平地对待门片上的每一点. 就像你立在圆心, 你到园的周边的每一点是等距的, 你就能公平客观地面对你遇到的所有的事. 

各位同学, 有人说人与人之间最不合适谈论的话题, 第一个叫政治, 第二个叫宗教, 那我们就来谈谈吧, 为什么这么说呢? 我们看看它是不是真的不可谈嘛, 对不对. 我们都知道, 台湾这一年来我们为了气爆的问题
或者我们为了真假油的问题. 哎呀, 这个蓝绿两党真的是骂来骂去的, 有人说这是绿营执政时发生的, 只是在蓝营的时代发现了, 有人说这是中央管理不当, 有人说是这是地方政府的缺失, 但是他们说的话我们都可以简单地化约成一句话, 都是敌营地错. 这句话摆在哪一个党团都合用, 你赞成吗? 可是今天如果你不要站在这两端, 对立的两端, 我们就去问, 为什么高雄气爆?为什么新北市也气爆? 虽然规模大小不同, 报道的日数不同
. 为什么油有假油? 我们一起来针对这个问题, 我们该了解世界其它文明先进国家为什么没有这个问题? 我们一起来解决, 我们一起让它不要再发生, 那我们就是道枢了吗? 所以这是非常重要的. 如果你希望永远只是你这一方赢了, 大获全胜, 那你常常就不能针对事情的本质跟核心的问题处理了, 所以希望我们今天虽然不从政, 可是我们也从政啊, 你家里就是一个小政府, 你要管理众人之事, 管理一家子的人, 我们就必须要了解, 怎么样能把人与人之间相处的事情处理好, 那就是应该公平客观, 无所偏袒, 不要是你跟谁距离比较近, 你就说他对, 你就为他的利益着想, 这样呢, 你才能应接你这一生无穷的对象与事物.

\begin{center}
	{\color{green} 是亦一无穷, 非亦一无穷也. 故曰莫若以明. }
\end{center}

\vspace{-0.5cm}

庄子说, 否则啊, 你坚持你是对的, 那你永远都是对的.

很多同学问我, 老师, 我们现在二十岁谈恋爱, 你觉得对方的条件最重要的是什么啊? 然后有二分之的学生告诉我, 他最重要的是她的外貌, 因为这张脸要看很久, 不能太早看腻喔. 我倒觉得, 如果是我的话. 我会从我的人生经验去看, 有时候可能有一张脸其实他并不丑, 甚至于他很帅. 可是他任何事情都觉得自己是对的, 别人是错的, 你不觉得这样的脸, 就越看越不帅, 越看越丑恶吗? 相对的, 可有一个老婆婆, 她也不一定长得多美, 可是她很仁慈, 很有爱心, 你就觉得越美, 连皱纹你都觉得是岁月的光辉, 对不对? 那如果这样你跟一个人在一起, 要是我, 我会注意, 当发生事情的时候, 这个人是会百分之百地反省别人, 还是百分之多少地反省自己? 我倒觉得这是你未来跟他相处开不开心的一个很重要的因素. 嘿, 我希望修过 <<庄子>> 课的同学, 你将来找对象的时候, 可以比别人顺利, 这个顺利不是别人追不上, 你追得上, 而是你会选对人, 你会选对一个让你一辈子过得非常开心的人. 

{\color{blue} 非亦一无穷}, 有时候呢, 我们今天觉得那个想法是错的, 那个作法是错的, 可是你换了位置, 你换了位置忽然觉得蛮值得体谅的嘛. 自从我开始到醉月湖边打拳, 我就看到台大的校规, 喂食禽鸟, 就校方会派一定的人喂食, 所以我每次看到有人在喂食这些禽鸟, 我就觉得, 我就很想看他到底是不识字还是怎么样? 这家长是想带坏小孩还是怎么样? 可是当我在备这个课的时候, 忽然想到如果我是那个小孩的奶奶, 或是我是他的母亲, 然后我们带了一包要有机的谷类, 然后我的孙子想要喂鸟, 我会说, 哎呀, 这个校规虽然说不能喂食禽鸟, 可是这可怜的禽鸟可没机会吃这么好, 这么纯净的米, 你懂吧. 孙子, 喂吧. 我会不会觉得可以是个例外?

然后呢, 我每天搭乘电梯到五楼来上课, 不是懒得爬楼梯, 因为希望开始讲课, 你就脸不红气不喘的, 对不对, 然后这时候我就看到, 尤其是时间有点赶的时候, 我就不知道为什么这么多同学, 忽然都要假扮成教师与残障同学? 因为学校的电梯旁边贴着, 本电梯专供教师与残障同学专用, 好吧, 你要装扮教师与残障同学也就罢了, 你们还真残障, 还进来得这么慢, 慢慢走, 一边聊天, 一边吃东西, 哈啰, 哈啰, 然后慢慢讲, 慢慢地走进来, 让在里面想赶快赶到教室上课的老师非常焦心. 可是一个教 <<庄子>> 的老师, 要怎么样让自己不焦心吗? 我开心反省我自己, 这张这个亚克力板, 真的是我当老师以后才贴上的吗? 还是我当学生的时候它就贴上了? 可是我真的没看见, 所有的学生都看不到这张字, 都是隐形的, 懂吧, 只有傻瓜看得见, 我又反省了, {\color{blue} 是亦一无穷, 非亦一无穷也. 故曰莫若以明}. 所以说啊, 我们才得到太阳跟月亮的地方嘛, 所以说你可以想象, 学习庄子的我, 我只要看谁有一点不顺眼, 或对谁有一点意见, 我就开始想怎么会有意见? 是我站在对立的那一方, 我就赶快站到他的立场想想, 那你本来觉得这个人有点可恶的, 其实也蛮可怜的, 也就没事了, 其实庄子叫我们{\color{blue} 莫若以明}, 不是叫我们没有是非, 是教我们学会包容跟体谅.


{\color{blue} 以指喻指之非指, 不若以非指之非指也; 以马喻马之非马, 不若以非马喻马之非马也. 天地一指也, 万物一马也.}

\begin{center}
	{\color{green} 以指喻指之非指, 不若以非指之非指也; 以马喻马之非马, 不若以非马喻马之非马也. }
\end{center}

\vspace{-0.5cm}

接着呢, 庄子说{\color{blue} 以指喻指之非指}, 你期望我们用这根手指, 去说明那根手指不是手指, 那我想这个我们配上图片是很容易了解啊, 那会画这个图是因为我养猫, 你们知道在这样的天气, 人跟猫的距离是非常近的, 对不对, 我和我的猫好像会牵着手一起睡觉似的. 然后你摸它的手, 觉得猫的手太神奇了, 这也叫手掌吗? 怎么有肉垫呢, 对不对? 我就会觉得这怎么是手指呢? 这不像人的手指, 不符合人对手指的定义, 如果我的猫咪知道我这么想, 它一定想, 你的才不是手指呢, 怎么说呢? 这猫的指甲是怎么样? 它可以伸缩的对不对? 它喜欢你, 它不想攻击你就缩起来, 它想攻击你就啪伸出来, 那你们人的指甲怎么那么笨啊, 只能呆呆的一定的长度, 那能叫手指吗? 于是呢, 两个都觉得彼此不是手指, 那用这个譬喻是要说明什么?

庄子又用了另一个譬喻, {\color{green} 以马喻马之非马, 不若以非马喻马之非马也.} 他说呀, 同样的道理, 我们用黑马的角度去看白马, 如果有的地方的马全是黑的, 现在有匹白马出现了, 他们说, 嘿, 怪物来了, 那肯定不是马, 兄弟咱们都是黑的, 一定会这么讲, 对不对. 可是呢, 用白马的标准, 如果一个地方都是白马, 现在来了一匹黑马, 你会问它真的是马吗? 

其实我们就是这样看待世界的, 我记得又一回我学生到我家, 看我养了一只柴犬, 他说, 老师你的柴犬叫什么? 我说因为柴犬日本狗吧, 所以我爸给它取了日文名叫 Yuri. 我学生那天给我的柴犬拍了一些照片, 他回到家把照片上传给我, 他说全天底下最不合逻辑的一个名字, 因为我的柴犬是黑色的, 百合是白色的. 他说, 我怎么遇见了一朵黑百合呢? 这太不合理了, 我相信还真不合理, 那好像一个黑人叫白雪公主一样, 懂吧. 可是你知道它为什么叫百合吗? 我父亲取这个名字啊, 是因为, 我学生告诉我人间有个活动叫宠物展, 我就去看了. 我第一年去看回来就带了一只乌龟回来, 象龟. 我第二年又去参观宠物展了, 我带来一只柴犬回来, 唉, 我妈看了受不了. 她说呀, 这谁要养的? 我说, 我. 是你要养的没错喔. 我说, 对. 她说, 请你不要让它离开你的房间, 我就愣住了. 然后当天我赶快去乖乖地去找个木匠, 在我的房门口做了一个隔栅, 因为我们三楼基本上是整个是同一个屋顶,是可以互通的, 就让它不要离开我的房间就对了. 我父亲知道了, 就给它取个名叫百合, 他希望我因为这只狗啊, 跟我妈妈能够百事合乐. 后来很有意思啊, 你知道女人有时候都会这样, 说不要结果都要. 后来我母亲不知道为什么越看这只狗越顺眼, 她偶尔一两次帮我遛狗就遛上隐了. 有一天呢, 我脚撞到了, 想搬到学校这边的宿舍来住, 我母亲就很不舍, 她说你这样害我没运动, 本来帮你遛狗是我最好的运动, 果然这只狗让我跟我母亲百事皆合啊. 因为疼爱这只狗很多共同的话题, 我母亲后来疼爱到坚持要出一半的狗的钱, 因为她不愿意让我说那是我的狗, 所以我要说的是, 如果你听了这只狗的身世, 你觉得它不叫百合, 谁适合叫百合啊? 对不对? 所以你会觉得那个叫百合的花, 它合了啥吗? 它够格叫百合吗?

我为什么要跟你们讲这个.

\begin{center}
	{\color{green} 天地一指也, 万物一马也.}
\end{center}

\vspace{-0.5cm}

就是我们在人世间, 如果都能用这样的态度, 来看待这个世界的是非, 就A说B错, 那你就想, 其实这就像人的手指跟猫咪的手指说, 你那不是手指, 就像白马跟黑马说, 你不是马, 是有一样的可能性的, 你如果有这样心得余地, 你就会容受更多得事物喔, 更多得事物.


{\Large {\color{purple} 厉与西施---你可曾有昨日认定得噩运, 今日欲成美好机缘得经验?}}

下一段:

{\color{blue} 可乎可, 不可乎不可. 道行之而成, 物谓之而然. 恶乎然? 然于然. 恶乎不然? 不然于不然. 物固有所然, 物故有所可. 无物不然, 无物不可. 故举莛与楹, 历与西施, 恢诡(gui)谲(jue)怪, 道同为一. 其分也, 成也; 其成也, 毁也. 凡物无成与毁, 复通为一. 唯达者知通为一, 为是不用, 而寓诸庸. 庸也者, 用也; 用也者, 通也; 通也者, 得也; 适得而几矣.} 

\begin{center}
	{\color{green} 可乎可, 不可乎不可. 道行之而成, 物谓之而然.}
\end{center}

\vspace{-0.5cm}

{\color{blue} 可乎可, 不可乎不可,} 什么意思呢? 这世界上, 我们都是根据可以得理由来说可以嘛, 比方说我们要发展经济, {\color{blue} 不可乎不可}, 我们根据不可以得理由来说不可以, 比方说影响生态


{\color{blue} 道行之而成}, 这世界上的路, 都是人走出来的, 是人开辟的, 它才叫一条路. {\color{blue} 物谓之而然}, 东西叫什么, 也是我们这样的称呼它, 它才有那个名字. 如果你了解到这个道理, 你就不会对很多东西那么地坚持, 那你说坚持对的不好吗? 其实我们生命中有太多的坚持, 其实是固执, 也无关乎真正的对错. 

\newpage 

\begin{center}
	{\color{green} 恶乎然? 然于然. 恶乎不然? 不然于不然. 物固有所然, 物固有所可. 无物不然, 无物不可. }
\end{center}

\vspace{-0.5cm}

{\color{blue} 恶乎然? 然于然.} 为什么我们今天说它对? 因为我们根据它对的理由来说它对嘛. 又为什么说它不对? 因为我们根据它不对的地方说它不对.

就拿在学校学习吧, 我看我的学生很多人, 他们可能实验室有什么事情, 或是课程上很紧凑, 就开始熬夜, 然后搞得好像大一,大二,大三,大四, 四年下来, 倒不是亭亭玉立了, 而是憔悴不少. 那你就会想, 到底是那书卷重要, 还是身体重要? 可是站在现在就业这么困难, 竞争这么地激烈, 你不好好表现, 你怎么能拿到入场券呢? 人生胜利组的入场券呢? 所以好像这么用功, 熬个夜也无妨, 反正年轻嘛, 身体还好. 又不是老师那种已经治疗出来的, 剩下的 quota(配额) 很少的人, 我么何必要那么谨慎呢?

{\color{blue} 然乎然}, 可是呢, 你又觉得好像不那么谨慎不行, 不然的话, 也不过就是生日嘛, 几个朋友这样跳到醉月湖游一圈, 他们绝对不是第一次跳的人. 我的学生告诉我, 常常有人在那里庆生就跳进去了, 可是他却失去他的生命, 那你就会想到, 如果这个人有每天练穴道导引, 每天做运动 或者那天要跳下去以前有先做暖身操, 那么这一切是否就可避免? 那我们去重视身体健康跟生命, 难道比一个书卷不重要吗? 你忽然会觉得好重要, 对不对? 那就赶快睡吧. 

我永远记得我以前小时候, 其实我的家长很少叫我说每天晚上不准睡, 他们都跟我说, 不要念了就去睡吧. 我母亲总说, 睡饱了, 明天精神好, 考试就可以头脑很清楚. 我就回答我母亲说, 妈, 可是我什么都没念, 我就算脑子很清楚, 坐在那里也只能发愣, 因为太晚开始读了, 所以我们说这所有的是非, 都有是非可说.


说到最后, {\color{blue} 物固有所然}, 每一样事物都有它对的那一部分. 我说十几年前吧, 你跟我, 你们那时候出生了对不对? 你们知道一个叫陈进兴案吗? 我们以前在电视前看了一个终极保镖,警匪枪战对不对?
可是很多人都在最后那一幕, 他躲在房子里快要被抓到的那一幕, 大家忽然发现, 当他跟他的儿女对话, 跟他妻子对话, 其实看起来还蛮正常,蛮慈祥,蛮温柔的,对不对? 那这个人怎么会对待外人这样呢?

那我们就要想, 有一个圣者, 看到有人就要上绞刑台了. 他会这么想, 他会说:感谢上帝, 感谢上苍, 如果不是你们对我的爱, 不是我拥有的这一切, 也许今天上绞刑台的就是我.

刚刚下课有一位同学, 我想他真是有用心在修这个课的同学, 他来问老师一个问题, 他说: 在照之于天, 这样的一个立场, 要怎样去看一个在捷运杀人的事件? 我沉思片刻之后跟他说: 就我一个教育工作者, 我看待这个事件, 我会觉得教育工作者, 还需要更加地努力, 更加地用心才能让这样地事件减到最少. 我说但是我不只是个老师, 我同时是个大众运输的乘客, 我可能我觉得我每次搭车的时候, 要带一个长一点的, 坚固一点的雨伞, 来护卫我自己的安全. 至于这个人, 这样一个犯罪事件吧, 我觉得这是非常非常多因素造成的. 也许有很多的人都必须负这个责任, 但当然包括他自己, 他的功过最后是交给法律去断定. 

那我们看待这件事, 真的是就好像我刚引用的那个故事喔, 有时候我们面对这些社会犯罪, 我都会想, 他是在什么样的家庭长大? 如果我跟他从小交换家庭, 我会不会走上这条路? 我通常会这样想. 我不知道, 因为我们都只有机会去过我们自己的人生. 你这样想的时候, 你对一些罪无可赦的或你觉得真是不要脸的, 会多一点同情喔. {\color{blue} 无物不然, 无物不可} 所以你就不会看得非常生气, 或是觉得非常难以忍受之类的.

\begin{center}
	{\color{green} 故举莛(ting)与楹, 厉与西施, 恢诡谲怪, 道通为一. }
\end{center}

\vspace{-0.5cm}

{\color{blue} 故举莛与楹,} 庄子觉得最小的木片叫莛, 莛这个字是小木片, 也可说是女子的发髻(zan), 跟什么呢, 跟楹, 楹是最大的梁柱. 这两种东西感情如果质料一样, 一定是梁柱贵很多, 对不对? 因为木材贵, 量体大嘛. 可是各位同学, 假使我今天正在上课, 我前面一撮头发不断掉下来,哎呀, 下课同学借我一个小小的发夹, 夹起来, 是木头材质做的, 我感激极了. 这时有个同学说,老师, 那有什么, 我们家一根柱子送给你, 你可以做出一千个发夹. 我想, 你有病吗? 我要一根大柱子干嘛? 这时候这个发夹显得非常地大, 梁柱显得非常地小, 你在需要的时候你才会觉得哪一个好, 没有高下, 无法分辨.你说, 老师下一个呢, {\color{blue} 厉与西施}, 历可是有名长癞病的丑女啊, 可是西施这个美女就不用说了, 那怎么会没有高下呢? 

这时我忍不住诉说一个我自己成长遭遇的故事. 我在高中的时候有两个很好的朋友, 当然我们交朋友都是因为这个人的质地或者机缘, 你非常喜欢她, 或者刚好变成非常好的朋友. 你不会说, 啊, 这个人好漂亮, 我跟她做朋友. 喔, 这个人不够漂亮, 我们通常选择朋友不会这样嘛, 对不对. 我有两个很好的朋友, 其中一个朋友是我心目中我们高中的校花, 很巧, 我们就住在同一个小镇, 我们就常常会一起坐公车上下学, 然后放学一起去吃点心, 变成好朋友. 

然后另一个好朋友呢, 她是非常喜欢我这个好朋友, 想跟她做好朋友, 所以我们就变成三个人是好朋友了, 那个朋友其实当然长得也是很端正啦. 可是女生从来不要求自己只是长得端正嘛, 可是她的个性还真特别, 其实我细看她, 有时觉得她也挺美的, 可是呢, 她个性还蛮吓人的. 我永远记得在高中我们女生打篮球, 然后有别班男同学来教我们打篮球, 有一个我们学校公认的帅哥来教我们班打篮球, 那个人好像是那个时候爱乐交响乐团电视台的片头, 就是这个人. 那我这个同学就走到他的面前就说, 莫某某, 你为什么可以迟到, 你自以为帅吗? 让我告诉你, 你丑毙了. 哎呀, 我们所有胆小斯文的女生站在旁边觉得真不好意思, 人家来教我们篮球, 我们这样对人家大声嚷嚷, 所以我虽然是她的好朋友, 不要讲高中不谈恋爱了, 就是大学四年我也不敢帮她介绍男朋友, 你知道吗? 那个人哪天会被骂成什么样, 我从来不敢介绍. 现在话说孔雀展羽毛, 不就是为了要吸引异性吗?我就来说, 说我这个校花朋友, 我说她老公跟我现在可也是好朋友啊, 那当然也是长相端正啊, 个性, 那也是不错啊, 可是我就不会觉得特别怎么样. 

可我刚刚讲的这个第一个朋友, 那我可奇了, 我从参加她婚礼我就百思不解, 哇, 这个人气质真好, 这个人真是斯文, 这个人看起来真是知书达礼,谦谦君子啊. 他还不只看起来, 他真的是, 后来他们结婚了. 好朋友嘛, 一定要去拜访的啊, 所以我去拜访我的好朋友的时候, 那时候她已经有孩子了, 然后她就说, 走吧, 我们去吃饭. 我说, 那你老公跟孩子呢. 喔, 我们待会见, 包一点回来给他们吃就行了. 我说: 喔喔好, 我们就出去了, 我们开始吃饭了. 我们就出去了, 我们开始吃饭了. 唉, 第二道菜了, 我说, 是不是跟餐厅老板要盒子, 赶快把你老公要吃的, 因为宝宝可能要喝牛奶嘛. 把它夹起来, 夹什么? 我们不是好姐妹一家人吗? 我们吃剩的我包回去就好, 我们那时候开始觉得有点不好意思, 我们就很开心地吃,很放怀地吃, 那最后有的东西剩多, 有的东西剩少, 就塑胶袋包一包. 好, 回到她家了. 她好像在给予恩典一样, 老公你的食物回来了, 她老公充满欢喜地接下, 孩子还抱在怀里睡得很好. 我们有点惊讶地离开了, 后来我才知道, 她老公跟我同行, 是某大学医学院的教授, 我们同学呢, 嫁给他以后就当个家庭主妇, 很无聊, 日子很无聊, 因为闲闲的. 有一天呢, 她就告诉我了, 她以前学商的, 她的老板很信任她, 因为我这个朋友人很端正, 操守很好, 老板希望她回去当会计, 她就考虑了, 月薪那么高. 然后她就问她老公说, 唉, 某某大公司要我回去, 你觉得怎么样? 她说我老公怎么不发一语啊? 绕到他的前面去看怎么不说话? 她老公就抱着孩子, 眼泪就这样掉下来, 一句话都不敢吭, 他心里不愿意但不敢说, 懂吧? 因为个性很好. 我们同学一看有点不忍, 好吧, 就当你的家庭主妇吧, 所以我同学是过这种无法无天的日子, 你以为她在家里做家事? 没有, 她到我家来, 我就这样边聊天, 边下面做菜给她们吃. 她说: 蔡璧名, 我以前因为觉得做菜很累, 不想做家事, 来到你家, 看你做菜还蛮轻松的, 我就也买了一些材料来做做看, 唉, 后来发现有调理包, 整个做好, 我就买那个了. 我说: 喔, 我不买调理包, 那当然是比较贵啊, 对不对? 自己做便宜啊, 当然做久了也好吃嘛. 你就知道她过什么样的生活

我为什么讲这里忽然提到她们两个, 一切尽在不言中, 很多女生觉得: 唉, 好多男生都外貌协会, 你看长得好看的人多吃香啊? 人家好几年每一个人追, 她一年就十几二十个人追, 再多人追也只能嫁一个, 懂吧. 可是很多人追, 心乱, 你懂吧. 那以前我们去图书馆看书, 三个女生一起去图书馆, 那个读书读到一半, 有男生丢纸条过来, 那要完全不动声色看完那一张, 要相当的定力, 要读 <<庄子>>才行, 你懂吧. 可是反观那个同学, 她在图书馆日以继夜, 从来没有字条丢过来, 所以非常专心喔, 所以很能充实自己, 专业能力很强, 可以学习很多的才艺, 所以各位同学, 你读完这一段, 忽然觉得长得美真是一种幸福, 长得不够美那更是另一种充实, 另一种好运喔. 

{\color{blue} 恢诡谲怪,} 恢是大, 诡呢, 诡是变, 谲呢, 谲是权诈, 怪, 那就是怪了, 不管有多大的, 多怪的, 多变来变去的, 在道的眼光都能通为一. 通为一不是告诉你们他都是一样的, 当然他们也都是一样的. 生命中有很多一样, 禅宗的白骨观, 要我们在椰林大道, 不要被美色, 这美色包括女色跟男色迷惑, 它要你有人走过来, 就看成是, 哇, 一排骷髅头这样晃过来了. 所以你能看到他的百年之后, 你就不会被美色迷惑. 

人有些是一样的, 每次我在跟我学生讲时间利用, 这世界最公平的就是不管你贫富贵贱, 你活着的那一天, 拥有的每一个人一天都是二十四小时, 很公平的, 还有最公平的是每一个人都会死亡. 那这里要讲的不是每个人都一样, 他要讲的是, 每一个人都是可以欣赏的, 因为有的人他可能有点懒, 可是他可能很放松, 他可能过得很开怀, 那也是值得你学习的啊. 可有的人, 你觉得这种人真是罪恶啊, 他怎么可以给那么多人下毒, 他怎么可以杀那么多人? 你知道有一种杀是快速的, 一下杀十几个. 有一种杀, 不用负法律责任的, 或负的法律责任, 因死无对证, 你到底吃了多少有毒的东西加在一起累积变成这样的癌症,这样的疾病? 到时候他可能不必负责的, 这种人很难欣赏, 但你可以同情. 如果有一个人, 他已经活到良心都没了, 那个新黑到只剩下要自己家族的钱, 或是自己的钱, 所有人都可以死, 那他活在这个世界上, 他已经失去人性. 人之所以人的光辉啊, 就是他已经不是人了, 他连宠物的价值也没有, 因为宠物还可以带给人欢乐, 对不对? 也有很多不错的个性, 像你们知道动物很不会记仇, 你们知道吗? 它不高兴, 它就发泄出来, 然后结束就结束了, 它不会记得你昨天是不是拿了一个小竹签戳了它一下, 它不会记得, 非常善良. 

可是一个人已经失去一个人最基本的人心跟价值, 他是值得同情的, 如果你相信人生命是永恒的, 人可能是有魂魄的, 那最后他就像托尔斯泰小说里面, 掉到地洞里, <<伊索寓言>>里面的一只小鬼,那是值得同情的, 这是{\color{blue} 道通为一}. 


{\Large {\color{purple} 知通为一, 你可曾在挫折中遇见成长?}}

\begin{center}
	\color{green} 其分也, 成也. 其成也, 毁也.
\end{center}

\vspace{-0.5cm}

庄子又说了: {\color{blue} 其分也, 成也}, 你不要觉得你失去了, 你不要觉得这个东西毁坏了, 说不定它是完成呢. 我昨天晚上收到一封我学生的信, 他说: 老师, 我发现啊, 你每次说完蛋了, 过不到几小时, 就有一个很热心的专业人士来帮助你, 完成你本来想完成的事. 我昨天当然有一些遭遇, 会觉得非常地感动, 就是有时候觉得你遇到一件很不好地事情, 可是因为那件不好的事情你必须要个变化, 可是没想到这个变化结果好像比原来还要好. 那就好像羊毛, 你今天剃了羊毛, 对于羊那是一种生命中不可承受之轻, 你可曾倾听过羊的心声? 我没有听过羊的心声, 但我知道, 因为我是养猫的人, 很多养猫的人, 你为了夏天夏天不要让它热, 你给它剃毛, 猫很气, 因为它觉得它的自尊都没了, 它变得那么丑, 它没办法面对世界, 你懂吗. 可是当羊咩咩觉得它失去这么多的时候, 有一件毛衣被织好了, 温暖了另一个寒冬里夜归的人.

{\color{blue} 其成也, 毁也.} 你以为你做好了一件家具, 哇, 这家具太精彩了. 可是你想过, 如果像印度跟日本的研究是真的, 植物也是有知觉的, 有所谓的生命科学探测仪, 去做一个家具的同时你想过一棵树的痛吗? 所以成就事物的背后, 为了建立, 其实你有毁毁. 所以在完成毛衣的时候呢,羊失去它的毛, 完成家具的时候, 树被砍下了. 



\begin{center}
	\color{green} 凡物无成与毁, 复通为一. 唯达者知通为一, 为是不用而寓诸庸.
\end{center}

\vspace{-0.5cm}

各位同学, 这并不是一个跟你我不相干的事情, 我们常常生命中真的是有得有失, 如果拿今天你们家长的立场吧, 你们家长现在都还年轻, 好多家长在孩子求学的阶段就不断跟他讲, 建中,台大, American, 建中, 台大, 美利坚. 这是以前 <<建中青年>> 的一个题目, 你们知道, 你的儿女现在都很成材, 一个一个都出国了. 后来, 出国就在那边嫁人了, 在那边定居了, 好像也没说要把你接去, 你懂吧. 还好, 你生得多, 最后留下一个没能力出国的, 然后这孩子就特别地孝顺, 就把你的晚年照顾得很好. 你忽然想到小时候你在念他他们得时候, 都说: 哎呀, 这些考第一名的就是光耀你啊, 让你觉得很开心, 这每一次都考坏的, 让你烦得要死. 可到后来, 你忽然觉得到底什么是成, 什么是毁? 谁是你最亲爱得儿子? 谁是你最亲爱得女儿? 这想法是会改变的喔, 就好像说, 你今天觉得, 哎呀, 太帅了, 今年果然所有的努力没有白费, 书卷拿到手了, 拿到手了怎么样? 暑假也检查检查身体吧, 肝脏指数出了问题, 这都是在台大很容易遇到的例子. 所以什么是成, 什么是毁? 其实庄子促使我们再去思考生命中的这些问题, {\color{blue} 凡物无成与毁, 复通为一.} 最后我们看万事万物, 我们发现有一方完成, 就有一面毁坏, 有得就有失. 你能看到万物都是这样的有成有毁可说. 这只有谁看得到啊?


通达得人才会明白, 这种有得有失的道理, 他明白了为是不用, 他人生在设定目标的时候, 就不会去设定那一种, 大家说最光耀的, 能拿到最多俸禄的工作证,专业执照, 因为他觉得那不是他要的人生. 唉, 我觉得在台大教书的老师, 应该说在台大教 <<庄子>> 的老师, 尤其教过工学院,医学院,法学院的老师, 最能体会这些话. 以前我还没有在台大教书喔, 我就这样念文科以中文系为第一志愿, 然后就一直留在文学哲学领域发展. 我如果没有来台大接触那么多工学院的学生, 我不知道工学院还真的还蛮辛苦的. 他们可能在帮老师做实验, 有时候做这个生物科技方面的, 有时候一天要站七,八个个小时. 到了科学园区工作吧, 朝九晚十是最人道的, 通常还要更晚. 那你说: 老师, 那别念工, 念商吧, 我前两个才帮一个商学院同学写推荐函. 我说你去深造啊? 不是深造吗? 也是. 我说: 刚才为什么说不是? 老师, 不是为了要深造, 因为太累, 想出国念书歇口气, 先休息一下. 我说: 是加班吗? 疯狂加班. 加班有加薪吗? 没有, 责任制哪来的加薪? 我说: 喔, 只有你们公司这样吗? 念商都这样. 

喔, 那听起来念商好像也不太好. 好吧, 那念医科好了, 在台湾万般皆下品, 只有医生高, 日据时代以来都是如此, 念医科好不好? 一阵 SARS 来就知道, 你懂吧, 你能请假吗? 你能今天关门吗? 

说来说去, 还是念文学院更农学院好, 对不对? 

答对了, 每天与大自然为伍, 古今中外, 中国古代最有智慧的人每天对话. 我们的工作, 不需要太多的压榨自己的身体. 

可是如果你说: 唉, 你儿子念哪? 台大中文系. 唉, 我记得当年, 我去我外公家, 我外公说: 你念什么啊? 那时候我妈说: 璧名今年考上博士班. 喔, 博士耶, 哪个学校? 台大是吗? 喔, 我外公很高兴, 什么系啊? 我妈说: 中文. 就在那一刹那我外公, 啊, 看了我一眼, 一幅那种我明天就要上街当乞丐似的, 你知道吗? 在那种西医师的心目中, 中文系大概就是这样的存在. 那如果你说: 你儿子念哪? 念农学院, 在十年前可能: 当农夫是吧? 喔, 现在不一样了, 是生物科技是吧? 不得了了对不对? 唉, 你们现在是两边占便宜啊, 是又不得了了, 又是生物科技又热门, 然后又没有什么害啊. 真的很不错, 以后粮食有什么灾荒还可以自给自足, 不要卖自己吃喔, 可是一般人不知道啊, 很多的家长还是叫小孩子最好是台大医科, 不然就电机系, 不然资讯系, 资讯系能念吗? 不是不能念, 要顶头悬哪, 随时打电脑, 不要熬夜啊, 你去和资工系的人讲这两句, 他们一定苦笑. 

{\color{purple} 于是呢, 一个学庄子的人, 他的人生不是以选择那一种自己能发挥最大工具价值的工作为人生的目标, 而只是把自己寄托在一个工作里面, 一个你喜欢的工作, 你做起来觉得开心的工作.} 你为什么寄托在这个工作里? 其实坦白讲, 我现在就我生病以后, 我现在的工时其实也算不少, 我一天如果工作超过十个小时应该很频繁,很常有, 可是我觉得我很开心, 因为我不觉得我在工作, 我觉得我在做我很感兴趣的事情. 

那你不要说, 每个人都想当大学教授. 我有个朋友, 一个女性朋友, 她先生是某所大学的校长, 她是家庭主妇. 我说: 如果当年你有工作, 你最想做的是什么? 因为她是服装设计专业. 她说啊: 我说了你绝对不信. 我说: 什么? 她说: 当计程车司机, 或是任何的司机. 她说: 她不知道为什么, 她这辈子最爱的就是开车, 好特别的一个嗜好. 她说: 可是她是校长夫人, 她先生又始终不相信她的开车技术, 所以很少能当司机.  

\begin{center}
	{\color{green} 庸者也, 用也; 通也者, 得也; 适得而几矣. 因是已. 已而不知其然谓之道.}
\end{center}

\vspace{-0.5cm}

所以我说: 我们就找一份你爱的工作, 然后你在世界上就是一个有用的人了. 是个有用的人, 对别人也有意义. 

你, 那你跟别人的生命不就流通了吗? 你的生命跟别人的生命互相往来, 那就对了, 我们是因为这样, 活在这个世界上的吧.

如果我们呱呱落地在婴儿房, 呱呱落地那天开始, 我们就自己围个堡, 把自己包起来或躲在桌下再也不出来, 我们应该不是为了这样来到这世界, 我们要认识很多的人, 跟很多人往来的, 你能跟这么多人沟通往来, 那就对了. 对了又能怎样呢?


{\color{blue} 适得而几矣.} 这个几是接近,庶几,差不多, 你能做到这样, 你就接近道了. 天哪, 这么容易耶, 再往下看. 



已经这样做了, 欲不知道为什么要这样做, 

也就是说, 当你这样做的时候, 你不是想: 唉, 圣人喔, 唉, 得道了, 嘿, 高人一等了. 不是这样, 你觉得我就是喜欢做, 我就是应该这样做, 这就是道了.

我在这边想讲一个我生命中的小故事, 是我在教这一段的时候, 我回想起来的喔. 我记得小学我第一次拿到学校成绩单的时候, 我好开心啊, 那当然是不小心拿了第一名喽, 然后很开心地跑回家. 然后给我妈妈看我的成绩单, 我妈妈说: 嗯, 第一名, 很好, 然后当然我妈妈一定会奖一些利多啦, 比方说: 哇, 下一个什么, 明天醒来的时候, 我妈妈最喜欢放两个袜子在我们的床头喔, 然后就假装会有什么天使,神仙之类的, 会放很多礼物在我的袜袋里, 我就会非常开心. 

当然要一个个炫耀嘛, 对不对, 第一名多久才考一次, 不知下一次还能不能拥有. 再来是跑到我爸爸面前, 我就说: 爸爸, 第一名耶. 我爸说: 喔喔. 没听到吗? 第一名耶, 爸. 嗯, 第一名. 不够惊讶, 再讲一次, 爸, 我全班第一名耶. 我爸这时候就放下手边的工作跟我说, 璧名, 爸爸给你说, 学生也是一种职业, 懂吗? 你不是很喜欢吃我们家附近的面摊吗? 是啊. 那你觉得他每碗煮得很好吃给你吃, 是不是很应该? 嗯, 他就是很好吃我才跟他买啊. 他说:爸爸是药剂师, 配药给病人, 配对了是不是很应该? 对啊. 那你觉得需要鼓掌或礼物吗? 不需要. 所以啊, 你的职业是学生, 考第一名或成绩好这都是应该的, 好, 就这样, 很好. 我从那天开始我就知道拿了第一名在家里静静的就好, 应该的, 煮面的嘛, 配药的嘛, 不都是这样吗? 

所以在我的家庭教育里面, 我真的很少听到我的父母说: 哎呀, 好棒好棒什么, 没有, 没有. 反而有时候我拿了什么第一名回来要讲话还要小心, 会不会让别人觉得我太骄傲了? 因为我父亲觉得骄傲是非常糟糕的德性, 所以我都很安静地在家里成长. 好, 唉, 我觉得这会养成习惯, 有一天你就会习惯这么热情地对待你的工作, 对待你所做的事情, 你觉得这就是应该的, 学校莫名其妙地颁了个奖项给我, 好像还有几次, 因为我忘记了要去领奖, 学校以为我很淡薄. 其实忘了, 懂吗? 就是记错时间了, 后来那些娃娃, 那些什么优良教师, 什么杰出教师拿回家, 这娃娃放在哪里都不是, 本来放在一个小矮几上, 我的猫, 咕噜, 就把它踢倒了, 有一个娃娃头就断了, 感觉不太吉利, 懂吧. 后来我就想想, 看来看去有个非常好地地方, 就是我那个壁柜转角, 刚好有一个地方很不方便, 我就五个娃娃一起摆进去, 我平常开柜子是看不到的, 大概两三年我打扫房子会看到, 喔, 拿过一些奖项. 我过两天可能到某一个老师家拜访, 嘿, 发现他的娃娃就摆在客厅最亮眼的地方, 那个微笑我就觉得, 这应该就是我读 <<庄子>>以后导致的不同, 就是面对一些荣誉你觉得有点害羞, 觉得这也没什么, 最好藏起来, 大概就是这样. 

所以你最后会觉得, 人, 你在你的份位里你应该做好, 就是很喜欢这工作就把它做好, 这不是应该的吗? 那是因为有别人偷懒, 那才显得你好吗, 还是怎样? 其实也不用, 那如果每一个人做好份内的事, 他当一个官, 他不贪污, 他当一个面包师, 他用好的材料来做给顾客吃, 都是应该的, 那有什么荣誉可说? 所以你都把自己份内的事做好, 你就没有这些烦恼.


%
\begin{flushright}
	 \tiny \kaishu \today  \  北京.
\end{flushright}
	 



%%%%%%%%%%%%%%%%%%%%%%%%%%%%%%%%%%%%%%%%%%%%%%%%%%%%%%%%%%%%%%%%%%%%%%%%%%%%%%%%%%%%%%%%%%%%
%\beginrefs
%\bibentry{1}{ J. B. Wilker}, 
%``Rings of Sets are Really Rings,''
%{\it The American Mathematical Monthly },
%1982, Vol.~89(3), pp.~211.
%\bibentry{2}{ A. Friedman }, 
%``Foundations of Modern Analysis.''
%\endrefs
\end{document}

              
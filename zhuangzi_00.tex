%%%%%%%%%%%%%%%%%%%%%%%%%%%%%%%%%%%%%%%%%%%%%%%%%%%%%%%%%%%%%%%%%%%%%%%%%%%%%%%%%%%%%%%%%%

%%%%%%%%%%%%%%%%%%%%%%%%%%%%%%%%%%% Author:Yao Zhang  %%%%%%%%%%%%%%%%%%%%%%%%%%%%%%%%%%%%
%%%%%%%%%%%%%%%%%%%%%%%%%%%%% Email: jaafar_zhang@163.com  %%%%%%%%%%%%%%%%%%%%%%%%%%%%%%%

%%%%%%%%%%%%%%%%%%%%%%%%%%%%%%%%%%%%%%%%%%%%%%%%%%%%%%%%%%%%%%%%%%%%%%%%%%%%%%%%%%%%%%%%%%

\documentclass[11pt]{article}
\usepackage[utf8]{inputenc} 
\usepackage[table]{xcolor}
\usepackage[most]{tcolorbox}
\usepackage[left=2.50cm, right=1.50cm, top=2.0cm, bottom=2.50cm]{geometry}
\usepackage{xcolor,url,cite}
\usepackage{amsmath,amsthm,amsfonts,amssymb,amscd,multirow,booktabs,fullpage,calc,multicol}
\usepackage{lastpage,enumitem,fancyhdr,mathrsfs,wrapfig,setspace,cancel,amsmath,empheq,framed}
\usepackage[retainorgcmds]{IEEEtrantools}
\usepackage{algorithm}
\usepackage{algorithmic}
\newlength{\tabcont}
\setlength{\parindent}{0.0in}
\setlength{\parskip}{0.05in}
\colorlet{shadecolor}{orange!15}
\parindent 0in
\parskip 12pt
\geometry{margin=1in, headsep=0.25in}
\usepackage{subfig,graphicx,framed}
\graphicspath{ {img1/} }
\usepackage{ctex}
\usepackage{multirow, tabularx}
%%%%%%%%%%%%%%%%%%%%%%%%%%%%%%%%%%%%%%%%%%%%%%%%%%%%%%%%%%%%%%%%%%%%%%%%%%%%%%%%%%%%%%%%%%

\renewcommand{\cite}[1]{[#1]}
\makeatletter
\@addtoreset{equation}{section}
\makeatother
\renewcommand{\theequation}{\arabic{section}.\arabic{equation}}
\usepackage[colorlinks,linkcolor=blue, anchorcolor=green,citecolor=red,urlcolor=blue]{hyperref}


%%%%%%%%%%%%%%%%%%%%%%%%%%%%%%%%%%%%%%%%%%%%%%%%%%%%%%%%%%%%%%%%%%%%%%%%%%%%%%%%%%%%%%%%%%

\def\beginrefs{\begin{list}%
		{[\arabic{equation}]}{\usecounter{equation}
			\setlength{\leftmargin}{0.8truecm}\setlength{\labelsep}{0.4truecm}%
			\setlength{\labelwidth}{1.6truecm}}}
	\def\endrefs{\end{list}}
\def\bibentry#1{\item[\hbox{[#1]}]}

\def\UrlBreaks{\do\A\do\B\do\C\do\D\do\E\do\F\do\G\do\H\do\I\do\J
	\do\K\do\L\do\M\do\N\do\O\do\P\do\Q\do\R\do\S\do\T\do\U\do\V
	\do\W\do\X\do\Y\do\Z\do\[\do\\\do\]\do\^\do\_\do\`\do\a\do\b
	\do\c\do\d\do\e\do\f\do\g\do\h\do\i\do\j\do\k\do\l\do\m\do\n
	\do\o\do\p\do\q\do\r\do\s\do\t\do\u\do\v\do\w\do\x\do\y\do\z
	\do\.\do\@\do\\\do\/\do\!\do\_\do\|\do\;\do\>\do\]\do\)\do\,
	\do\?\do\'\do+\do\=\do\#}

%%%%%%%%%%%%%%%%%%%%%%%%%%%%%%%%%%%%%%%%%%%%%%%%%%%%%%%%%%%%%%%%%%%%%%%%%%%%%%%%%%%%%%%%%%
\begin{document}
\kaishu 
	
\setcounter{section}{-1}
\title{title}
\thispagestyle{empty}

\begin{center}
	{\Large \kaishu 正是时候读庄子}
	
	{\Large \kaishu 导读}
\end{center}

\vspace{-0.5cm}

他这么活过他的一生, 留下一本书, <<庄子>>. 这本书影响了陶渊明的一生. 影响了李太白的一生. 影响了白居易的一生. 影响了苏东坡的一生. 唐玄宗下诏称此书为南华真经, 尊庄子为南华真人. 清初名评论家金圣叹, 评定这本书是天下第一才子书.

才子必读, 华人欲成才者必读, 如果你醉心于技进于道,技道合一的职人文化, 追本溯源, 请读庄子. 如果你不想成材,不想在优胜劣败的竞走中疲惫一生, 也请读庄子. 如果想处于才与不才之间, 想在人生的惊涛骇浪间存活,无伤, 更练就日益精进的乘御之力, 就请订阅庄子.

福轻乎羽, 莫之知载. 祸重乎地, 莫之知避. (<<庄子$ \cdot $人间世>>)

庄子所处的战国时代, 平民百姓能拥有的福分比羽毛还要轻薄, 飘忽不定不知道要怎样才能承接,拥有; 可是灾难祸患欲比山河大地还要沉重, 想要闪避欲不知道有什么方法能全身而退. 当时, 一次战争里被斩首,杀害的士卒多达数万,数十万人.

而庄子, 就在这个布满罗网,暗藏凶器的时代社会里, 担任一个小小漆树园的小小吏.

必须承受,最能感受时代之殇的, 莫过于金字塔底层. 是战国中期的庄子, 也是当代的你我. 萧条异代都同样涌动着如风浪翻滚,层出不穷的普世之殇. 

与庄子为友, 在李太白之后,东坡之后, 在诗人,哲人,职人与成千上万因此改变姿势,意识与用情的华人之后, 蔡老师的课程将陪我们, 逐字逐句读完庄子亲笔内七篇, 而庄周在此中断续铺陈,辐辏的三个主题: 姿势,意识与感情, 也讲在逐字逐句的讲解中豁然胸次. 

我们会发现, 原来只要把注意力收回自身, 心就可以不烦,不乱,不痛. 原来只要掌握正确的姿势原则, 身体竟可以如此轻盈放松. 原来感情可以不执著,不陷溺, 只要懂得深情而不滞于情. 原来身心的安定, 是面对混乱的时代最有力量的武器.

时局板荡, 暧昧天光, 半部论语治天下, 半部庄子治身心, 正是时候, 好好读庄子. 


以上这段文字摘自 {\color{blue} \cite{2}}.


点击 \hspace{-0.25cm }\url{http://ocw.aca.ntu.edu.tw/ntu-ocw/ocw/cou/103S123}, 然后选择 NTU video, 可观看蔡璧名老师主讲的<<正是时候读庄子>>的课程录影.

\vspace{-0.5cm}

%%%%%%%%%%%%%%%%%%%%%%%%%%%%%%%%%%%%%%%%%%%%%%%%%%%%%%%%%%%%%%%%%%%%%%%%%%%%%%%%%%%%%%%%%%%%
\beginrefs
\bibentry{1}{ 蔡璧名}, 
``正是时候读庄子: 庄子的姿势,意识与感情'',
{\it 天下杂志出版 },
2015.
\bibentry{2}{台大开放式课程}, \hspace{-0.25cm}
{ \url{http://ocw.aca.ntu.edu.tw/ntu-ocw/ocw/cou_intro/103S123}}.
\endrefs

\begin{flushright}
	\tiny \kaishu \today  \  北京.
\end{flushright}

\end{document}

              